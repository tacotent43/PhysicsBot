\documentclass[a4paper,12pt]{article}
\usepackage{fancyhdr}   % Для настройки заголовков и футеров
\usepackage{graphicx}    % Для работы с изображениями (например, логотип)
\usepackage{lipsum}      % Для примера текста
\usepackage{geometry}    % Для настройки полей страницы
\usepackage{mathtext}
\usepackage[T2A]{fontenc}
\usepackage[utf8]{inputenc}
\usepackage[english, russian]{babel}
\usepackage{titlesec}
\usepackage{xcolor}
\usepackage{graphicx}
\usepackage{caption}
\usepackage{subcaption}
\usepackage{wrapfig}
\usepackage{gensymb}
\usepackage{amsmath}
\usepackage{amssymb}
\usepackage{tikz}
\usepackage{circuitikz} 
\usepackage{physics}
\usepackage{enumitem}

\geometry{top=2cm, bottom=2cm, left=1cm, right=1cm}

% Настройка заголовков и футеров
\pagestyle{fancy}
\fancyhf{} % Убираем все стандартные элементы
\fancyfoot[C]{\rule[20pt]{\textwidth}{0.4pt}}
\fancyhead{} % Убираем верхнюю линию заголовка
\fancyfoot[L]{Page \thepage}  % Нумерация страниц по центру
\fancyfoot[R]{\copyright\ Бот для подготовки к ЕГЭ по физике. 2025г}  % Копирайт в правом нижнем углу


\titleformat{\section}[block]{\normalfont\Large\bfseries}{\thesection}{3em}{}
\setlist[itemize]{label=\scriptsize\textbullet}


%~%~%~%~%~%~%~%~%~%~%~%~%~%~%~%~%~%~%~%~%
%          document starts here         %
%~%~%~%~%~%~%~%~%~%~%~%~%~%~%~%~%~%~%~%~%

\begin{document}

\begin{center}
\scalebox{2}{\textbf{Законы сохранения}}
\end{center}

\vspace{-2em}



% % 4.1
% \section*{4.1 Импульс тела}

% \vspace{-9pt}
% \subsection*{Определение:}
% \vspace{-3pt}
% \textbf{Импульс тела} ($p$): Векторная физическая величина, являющаяся мерой механического движения тела. Определяется как произведение массы тела на его скорость.

% \vspace{-9pt}
% \subsection*{Формула:}
% \vspace{-3pt}
% $$ p = m \cdot v $$
% \begin{enumerate}[itemsep=0pt, topsep=0pt, parsep=0pt]
%   \item $p$ – импульс тела;
%   \item $m$ – масса тела;
%   \item $v$ – скорость тела.
% \end{enumerate}


% % 4.2
% \section*{4.2 Импульс системы тел}

% \vspace{-9pt}
% \subsection*{Определение:}
% \vspace{-3pt}
% \textbf{Импульс системы тел} ($P$): Векторная сумма импульсов всех тел, входящих в систему.

% \vspace{-9pt}
% \subsection*{Формула:}
% \vspace{-3pt}
% $$ P = p_1 + p_2 + ... + p_n = m_1v_1 + m_2v_2 + ... + m_nv_n, $$
% где: 
% \begin{enumerate}[itemsep=0pt, topsep=0pt, parsep=0pt]
%   \item $P$ – импульс системы тел;
%   \item $p_n$ – импульс $n$-го тела;
%   \item $m_n$ – масса $n$-го тела;
%   \item $v_n$ – скорость $n$-го тела.
% \end{enumerate}



% % 4.3
% \section*{4.3 Закон сохранения импульса}

% \vspace{-9pt}
% \subsection*{Определение:}
% \vspace{-3pt}
% \textbf{Закон сохранения импульса} --- в замкнутой системе (системе, на которую не действуют внешние силы или сумма внешних сил равна нулю) полный импульс системы остается постоянным с течением времени.

% \vspace{-9pt}
% \subsection*{Формула:}
% \vspace{-3pt}
% \begin{enumerate}[itemsep=0pt, topsep=0pt, parsep=0pt]
%   \item $P_{\text{начальное}}$ – импульс системы до взаимодействия;
%   \item $P_{\text{конечное}}$ – импульс системы после взаимодействия;
%   \item $v_i$ – скорости тел до взаимодействия;
%   \item $u_i$ – скорости тел после взаимодействия.
% \end{enumerate}

% \vspace{-9pt}
% \subsection*{Важно:}
% \vspace{-3pt}
% \begin{enumerate}[itemsep=0pt, topsep=0pt, parsep=0pt]
%   \item Закон выполняется только для замкнутых систем.
%   \item Применяется для анализа столкновений, реактивного движения и других процессов.
% \end{enumerate}



% % 4.4
% \section*{4.4 Работа силы}

% \vspace{-9pt}
% \subsection*{Определение:}
% \vspace{-3pt}
% \textbf{Работа силы ($A$)} --- скалярная физическая величина, являющаяся мерой действия силы на тело, в результате которого происходит перемещение тела.

% \vspace{-9pt}
% \subsection*{Формула (для постоянной силы и прямолинейного движения):}
% \vspace{-3pt}
% $$ A = F \cdot s \cdot \cos{\alpha}, $$
% где: 
% \begin{enumerate}[itemsep=0pt, topsep=0pt, parsep=0pt]
%   \item $A$ --- работа силы;
%   \item $F$ --- модуль силы;
%   \item $s$ --- формула перемещения силы;
%   \item $\alpha$ --- угол между направлением силы и направлением перемещения.
% \end{enumerate}

% \vspace{9pt}
% \subsection*{Частные случаи:}
% \vspace{-3pt}
% \begin{enumerate}[itemsep=0pt, topsep=0pt, parsep=0pt]
%   \item Если сила и перемещение сонаправлены ($\alpha=0\degree$), то $A = F \cdot s$ (работа положительная);
%   \item Если сила и перемещение противоположно направлены ($\alpha=180\degree$), то $A = -Fs$ (работа отрицательная);
%   \item Если сила перпендикулярна перемещению ($\alpha=90\degree$), то $A = 0$ (работа равна нулю).
% \end{enumerate}



% % 4.5
% \section*{4.5 Мощность}

% \vspace{-9pt}
% \subsection*{Определение:}
% \vspace{-3pt}
% \textbf{Мощность ($P$)} --- cкалярная физическая величина, характеризующая скорость совершения работы.

% \vspace{-9pt}
% \subsection*{Формула:}
% \vspace{-3pt}
% $$ P = \frac{A}{t}, $$
% где:
% \begin{enumerate}[itemsep=0pt, topsep=0pt, parsep=0pt]
%   \item $P$ --- мощность;
%   \item $A$ --- работа;
%   \item $t$ --- время, за которое совершена работа.
% \end{enumerate}



% \vspace{-9pt}
% \section*{Импульс тела:}
% \vspace{-3pt}
% Импульс тела \(\vec{p}\) — векторная величина, равная произведению массы тела \(m\) на его скорость \(\vec{v}\):
% $$
% \vec{p} = m \vec{v}.
% $$

% \vspace{-9pt}
% \section*{Импульс системы тел:}
% \vspace{-3pt}
% Импульс системы тел равен векторной сумме импульсов всех тел, входящих в систему:
% $$
% \vec{p}_{\text{системы}} = \sum_{i=1}^n \vec{p}_i = \sum_{i=1}^n m_i \vec{v}_i.
% $$

% \vspace{-9pt}
% \section*{Закон сохранения импульса:}
% \vspace{-3pt}
% Если сумма внешних сил, действующих на систему, равна нулю, то импульс системы сохраняется:
% $$
% \vec{p}_{\text{системы}} = \text{const}.
% $$

% \vspace{-9pt}
% \section*{Работа силы:}
% \vspace{-3pt}
% Работа \(A\) силы \(\vec{F}\) на перемещении \(\vec{s}\) равна скалярному произведению вектора силы на вектор перемещения:
% $$
% A = \vec{F} \cdot \vec{s} = F s \cos \alpha,
% $$
% где \(\alpha\) — угол между вектором силы и вектором перемещения.

% \vspace{-9pt}
% \section*{Мощность:}
% \vspace{-3pt}
% Мощность \(P\) — скалярная величина, равная отношению работы \(A\) ко времени \(t\), за которое она совершена:
% $$
% P = \frac{A}{t}.
% $$

% \vspace{-9pt}
% \section*{Работа как мера изменения энергии:}
% \vspace{-3pt}
% Работа силы равна изменению кинетической энергии тела:
% $$
% A = \Delta E_k = E_{k2} - E_{k1}.
% $$

% \vspace{-9pt}
% \section*{Кинетическая энергия:}
% \vspace{-3pt}
% Кинетическая энергия \(E_k\) тела массой \(m\), движущегося со скоростью \(v\):
% $$
% E_k = \frac{m v^2}{2}.
% $$

% \vspace{-9pt}
% \section*{Потенциальная энергия:}
% \vspace{-3pt}
% Потенциальная энергия \(E_p\) тела в гравитационном поле на высоте \(h\):
% $$
% E_p = m g h,
% $$
% где \(g\) — ускорение свободного падения.

% \vspace{-9pt}
% \section*{Закон сохранения механической энергии:}
% \vspace{-3pt}
% Если система замкнута и консервативна, то полная механическая энергия системы сохраняется:
% $$
% E_{\text{мех}} = E_k + E_p = \text{const}.
% $$



\section*{4.1 Импульс тела}
\textbf{Импульс тела $\vec{p}$} — векторная величина, равная произведению массы тела \(m\) на его скорость \(\vec{v}\):
\vspace{-0.3em}
$$\vec{p} = m \vec{v},$$
где:
\begin{itemize}[itemsep=0pt, topsep=0pt, parsep=0pt]
  \setlength\itemsep{0em}
  \item \(m\) — масса тела (кг),
  \item \(\vec{v}\) — скорость тела (м/с).
\end{itemize}


\section*{4.2 Импульс системы тел}
Импульс системы тел равен векторной сумме импульсов всех тел, входящих в систему:
\vspace{-0.3em}
$$\vec{p}_{\text{системы}} = \sum_{i=1}^n \vec{p}_i = \sum_{i=1}^n m_i \vec{v}_i,$$
где:
\begin{itemize}[itemsep=0pt, topsep=0pt, parsep=0pt]
  \setlength\itemsep{0em}
  \item \(m_i\) — масса \(i\)-го тела (кг),
  \item \(\vec{v}_i\) — скорость \(i\)-го тела (м/с),
  \item \(n\) — количество тел в системе.
\end{itemize}


\section*{4.3 Закон сохранения импульса}
Если сумма внешних сил, действующих на систему, равна нулю, то импульс системы сохраняется:
\vspace{-0.3em}
$$\vec{p}_{\text{системы}} = \text{const},$$
где:
\begin{itemize}[itemsep=0pt, topsep=0pt, parsep=0pt]
  \setlength\itemsep{0em}
  \item \(\vec{p}_{\text{системы}}\) — импульс системы (кг·м/с),
  \item \(\text{const}\) — постоянная величина.
\end{itemize}


\section*{4.4 Работа силы}
Работа \(A\) силы \(\vec{F}\) на перемещении \(\vec{s}\) равна скалярному произведению вектора силы на вектор перемещения:
\vspace{-0.3em}
$$A = \vec{F} \cdot \vec{s} = F s \cos \alpha,$$
где:
\begin{itemize}[itemsep=0pt, topsep=0pt, parsep=0pt]
  \setlength\itemsep{0em}
  \item \(F\) — модуль силы (Н),
  \item \(s\) — модуль перемещения (м),
  \item \(\alpha\) — угол между вектором силы и вектором перемещения.
\end{itemize}


\section*{4.5 Мощность}
Мощность \(P\) — скалярная величина, равная отношению работы \(A\) ко времени \(t\), за которое она совершена:
\vspace{-0.3em}
$$P = \frac{A}{t},$$
где:
\begin{itemize}[itemsep=0pt, topsep=0pt, parsep=0pt]
  \item \(A\) — работа (Дж),
  \item \(t\) — время (с).
\end{itemize}


\section*{4.6 Работа как мера изменения энергии}
Работа силы равна изменению кинетической энергии тела:
\vspace{-0.3em}
$$A = \Delta E_k = E_{k2} - E_{k1},$$
где:
\begin{itemize}[itemsep=0pt, topsep=0pt, parsep=0pt]
  \setlength\itemsep{0em}
  \item \(E_{k1}\) — начальная кинетическая энергия (Дж),
  \item \(E_{k2}\) — конечная кинетическая энергия (Дж).
\end{itemize}


\section*{4.7 Кинетическая энергия}
Кинетическая энергия \(E_k\) тела массой \(m\), движущегося со скоростью \(v\):
\vspace{-0.3em}
$$E_k = \frac{m v^2}{2},$$
где:
\begin{itemize}[itemsep=0pt, topsep=0pt, parsep=0pt]
  \setlength\itemsep{0em}
  \item \(m\) — масса тела (кг),
  \item \(v\) — скорость тела (м/с).
\end{itemize}


\section*{4.8 Потенциальная энергия}
Потенциальная энергия \(E_p\) тела в гравитационном поле на высоте \(h\):
\vspace{-0.3em}
$$E_p = m g h,$$
где:
\begin{itemize}[itemsep=0pt, topsep=0pt, parsep=0pt]
  \setlength\itemsep{0em}
  \item \(m\) — масса тела (кг),
  \item \(g\) — ускорение свободного падения (\(g \approx 9.81 \, \text{м/с}^2\)),
  \item \(h\) — высота над нулевым уровнем (м).
\end{itemize}


\section*{4.9 Закон сохранения механической энергии}
Если система замкнута и консервативна, то полная механическая энергия системы сохраняется:
\vspace{-0.3em}
$$E_{\text{мех}} = E_k + E_p = \text{const},$$
\vspace{-0.3em}
где:
\begin{itemize}[itemsep=0pt, topsep=0pt, parsep=0pt]
  \setlength\itemsep{0em}
  \item \(E_k\) — кинетическая энергия (Дж),
  \item \(E_p\) — потенциальная энергия (Дж),
  \item \(\text{const}\) — постоянная величина.
\end{itemize}




\end{document}