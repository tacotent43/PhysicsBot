\documentclass[a4paper, 12pt]{article} 
\usepackage{fancyhdr}   % Для настройки заголовков и футеров
\usepackage{graphicx}    % Для работы с изображениями (например, логотип)
\usepackage{lipsum}      % Для примера текста
\usepackage{geometry}    % Для настройки полей страницы
\usepackage{mathtext}
\usepackage[T2A]{fontenc}
\usepackage[utf8]{inputenc}
\usepackage[english, russian]{babel}
\usepackage{titlesec}
\usepackage{xcolor}
\usepackage{graphicx}
\usepackage{caption}
\usepackage{subcaption}
\usepackage{wrapfig}
\usepackage{gensymb}
\usepackage{amsmath}
\usepackage{amssymb}
\usepackage{tikz}
\usepackage{circuitikz} 
\usepackage{physics}
\usepackage{enumitem}

\geometry{top=2cm, bottom=2cm, left=1cm, right=1cm}

% Настройка заголовков и футеров
\pagestyle{fancy}
\fancyhf{} % Убираем все стандартные элементы
\fancyfoot[C]{\rule[20pt]{\textwidth}{0.4pt}}
\fancyhead{} % Убираем верхнюю линию заголовка
\fancyfoot[L]{Page \thepage}  % Нумерация страниц по центру
\fancyfoot[R]{\copyright\ Бот для подготовки к ЕГЭ по физике. 2025г}  % Копирайт в правом нижнем углу


\titleformat{\section}[block]{\normalfont\Large\bfseries}{\thesection}{3em}{}
\setlist[itemize]{label=\scriptsize\textbullet}


%~%~%~%~%~%~%~%~%~%~%~%~%~%~%~%~%~%~%~%~%
%          document starts here         %
%~%~%~%~%~%~%~%~%~%~%~%~%~%~%~%~%~%~%~%~%

\begin{document}

\begin{center}
\scalebox{2}{\textbf{Динамика}}
\end{center}


% % 2.1
% \section*{2.1 Инерциальные системы отсчета. Первый закон Ньютона}
% \vspace{-9pt}
% \subsection*{Определение:}
% \vspace{-3pt}
% \textbf{Инерциальная система отсчета (ИСО)} --- cистема отсчета, в которой тело, на которое не действуют силы (или действие сил скомпенсировано), находится в состоянии покоя или равномерного прямолинейного движения.

% \vspace{-9pt}
% \subsection*{Формулировка первого закона Ньютона:}
% \vspace{-3pt}
% Существуют такие системы отсчёта, называемые инерциальными, относительно которых материальные точки, когда на них не действуют никакие силы (или действуют силы взаимно уравновешенные), находятся в состоянии покоя или равномерного прямолинейного движения.

% \vspace{-9pt}
% \subsection*{Историческая формулировка:}
% \vspace{-3pt}
% Всякое тело продолжает удерживаться в своём состоянии покоя или равномерного и прямолинейного движения, пока и поскольку оно не понуждается приложенными силами изменить это состояние.

% \vspace{-9pt}
% \subsection*{Почему историческая формулировка считается неудовлетворительной?:}
% \vspace{-3pt}
% \textbf{Во-первых}, термин \textit{«тело»} следует заменить термином \textit{«материальная точка»}, так как тело конечных размеров в отсутствие внешних сил может совершать и вращательное движение. 
% \par \noindent
% \textbf{Во-вторых}, Ньютон в своём труде опирался на существование абсолютной неподвижной системы отсчёта, то есть абсолютного пространства и абсолютного времени, а это представление современная физика отвергает. С другой стороны, в произвольной (например, вращающейся) системе отсчёта закон инерции неверен, поэтому ньютоновская формулировка была заменена постулатом существования инерциальных систем отсчёта.




% % 2.2
% \section*{2.2 Принцип относительности Галилея}
% \vspace{-9pt}
% \subsection*{Формулировка:}
% \vspace{-3pt}
% \textbf{Принцип относительности Галилея:} \textit{"Для предметов, захваченных равномерным движением, это последнее как бы не существует и проявляет своё действие только на вещах, не принимающих в нём участия."}.

% \par \noindent
% \textbf{Трансформации Галилея} выражают связь координат и времени между двумя ИСО, движущимися друг относительно друга равномерно и прямолинейно.
% \begin{enumerate} [itemsep=0pt, topsep=0pt, parsep=3pt]
%   \item \textbf{Координаты:} $x' = x - vt$, $y' = y$, $z' = z$ (если движение вдоль оси x).
%   \item \textbf{Время:} $t' = t$.
%   \item $(x, y, z, t)$ - координаты и время в одной ИСО.
%   \item $(x', y', z', t')$ - координаты и время в другой ИСО.
%   \item $v$ - скорость движения второй ИСО относительно первой.
% \end{enumerate}


% % 2.3
% \section*{2.3 Масса тела}
% \vspace{-9pt}
% \textbf{Масса ($m$):} Физическая величина, являющаяся мерой инертности тела и его способности сохранять свою скорость. Также масса является мерой гравитационных свойств тела.
% \vspace{-3pt}

% \textbf{Свойства массы:}
% \begin{enumerate} [itemsep=0pt, topsep=0pt, parsep=3pt]
%   \item Масса – скалярная величина.
%   \item Масса – аддитивная величина: масса системы тел равна сумме масс отдельных тел.
%   \item Масса – инвариантна: не меняется при переходе из одной ИСО в другую (в классической механике).
% \end{enumerate}
% \textbf{Единица измерения:} килограмм (кг).


% % 2.4
% \section*{2.4 Плотность вещества}
% \vspace{-9pt}
% \textbf{Плотность ($\rho$):} Физическая величина, равная отношению массы тела к его объему.
% \vspace{-3pt}

% Формула: $\rho = \frac{m}{V}$, где
% \begin{enumerate} [itemsep=0pt, topsep=0pt, parsep=3pt]
%   \item $\rho$ - плотность,
%   \item $m$ - масса,
%   \item $V$ - объем.
% \end{enumerate}
% Единица измерения: $\frac{кг}{м^3}$.


% % new page
% \newpage


% % 2.5
% \section*{2.5 Сила}
% \vspace{-9pt}
% \textbf{Сила ($F$):} Векторная физическая величина, являющаяся мерой механического воздействия на тело, в результате которого тело получает ускорение или деформируется.
% \vspace{-3pt}

% \textbf{Характеристики силы:}
% \begin{enumerate} [itemsep=0pt, topsep=0pt, parsep=3pt]
%   \item Величина (модуль).
%   \item Направление.
%   \item Точка приложения.
% \end{enumerate}
% Единица измерения: ньютон (Н).


% % 2.6
% \section*{2.6 Принцип суперпозиции сил}
% \vspace{-9pt}
% \textbf{Принцип суперпозиции сил:} Если на тело одновременно действует несколько сил, то их действие эквивалентно действию одной силы, равной векторной сумме всех действующих сил.
% \vspace{-3pt}

% Формула: $\sum{\vec{F_R}} =\vec{F_1} + \vec{F_2} + ... + \vec{F_n}$.


% % 2.7
% \section*{2.7 Второй закон Ньютона}
% \vspace{-9pt}
% \textbf{Второй закон Ньютона:} Ускорение, которое получает тело, прямо пропорционально равнодействующей силе, действующей на тело, и обратно пропорционально его массе.
% \vspace{-3pt}

% Формула: $\vec{F_R} = m\vec{a}$, где
% \begin{enumerate} [itemsep=0pt, topsep=0pt, parsep=3pt]
%   \item $\vec{F_R}$ – векторная равнодействующая сила,
%   \item $m$ – масса тела,
%   \item $a$ – вектор ускорения тела.
% \end{enumerate}

% В импульсной форме: $F_R\Delta{t} = \Delta{p}$, где
% \begin{enumerate} [itemsep=0pt, topsep=0pt, parsep=3pt]
%   \item $\Delta{p}$ - изменение импульса тела,
%   \item $\Delta{t}$ - время действия силы.
% \end{enumerate}

% \newpage


\section*{2.1 Инерциальные системы отсчета. Первый закон Ньютона}

\vspace{-9pt}
\subsection*{Определение:}
\vspace{-3pt}
\textbf{Инерциальная система отсчета} --- это система отсчета, в которой тело, не подверженное действию внешних сил, движется равномерно и прямолинейно или покоится.

\vspace{-9pt}
\subsection*{Формулировка первого закона Ньютона:}
\vspace{-3pt}
Существуют такие системы отсчёта, называемые инерциальными, относительно которых материальные точки, когда на них не действуют никакие силы (или действуют силы взаимно уравновешенные), находятся в состоянии покоя или равномерного прямолинейного движения.

\vspace{-9pt}
\subsection*{Историческая формулировка:}
\vspace{-3pt}
Всякое тело продолжает удерживаться в своём состоянии покоя или равномерного и прямолинейного движения, пока и поскольку оно не понуждается приложенными силами изменить это состояние.

\vspace{-9pt}
\subsection*{Почему историческая формулировка считается неудовлетворительной?}
\vspace{-3pt}
\textbf{Во-первых}, термин \textit{«тело»} следует заменить термином \textit{«материальная точка»}, так как тело конечных размеров в отсутствие внешних сил может совершать и вращательное движение. 
\par \noindent
\textbf{Во-вторых}, Ньютон в своём труде опирался на существование абсолютной неподвижной системы отсчёта, то есть абсолютного пространства и абсолютного времени, а это представление современная физика отвергает. С другой стороны, в произвольной (например, вращающейся) системе отсчёта закон инерции неверен, поэтому ньютоновская формулировка была заменена постулатом существования инерциальных систем отсчёта.


\section*{2.2 Принцип относительности Галилея}

\vspace{-9pt}
\subsection*{Принцип относительности Галилея}
\vspace{-3pt}
Принцип относительности Галилея утверждает, что все механические процессы протекают одинаково во всех инерциальных системах отсчета. Математически это выражается в инвариантности законов механики относительно преобразований Галилея:
$$ x' = x - vt, \quad y' = y, \quad z' = z, \quad t' = t $$
где $x, y, z $ --- координаты в одной системе отсчета, $x', y', z' $ --- координаты в другой системе, движущейся со скоростью $v $ относительно первой.


\section*{2.3 Масса тела}

\vspace{-9pt}
\subsection*{Определение:}
\vspace{-3pt}
\textbf{Масса тела} --- это физическая величина, характеризующая инертные и гравитационные свойства тела. Единица измерения массы в системе СИ --- килограмм (кг).


\section*{2.4 Плотность вещества}

\vspace{-9pt}
\subsection*{Определение}
\vspace{-3pt}
\textbf{Плотность вещества} --- это физическая величина, равная отношению массы тела к его объему:
$$ \rho = \frac{m}{V} $$
где $\rho $ --- плотность, $m $ --- масса, $V $ --- объем. Единица измерения плотности в системе СИ --- килограмм на кубический метр (кг/м$ $).


\section*{2.5 Сила}

\vspace{-9pt}
\subsection*{Сила}
\vspace{-3pt}
\textbf{Сила} --- это векторная величина, характеризующая взаимодействие тел и вызывающая изменение их скорости или деформацию. Единица измерения силы в системе СИ --- ньютон (Н).


\section*{2.6 Принцип суперпозиции}

\vspace{-9pt}
\subsection*{Формулировка:}
\vspace{-3pt}
\textbf{Принцип суперпозиции:} если на тело действуют несколько сил, то их действие эквивалентно действию одной силы, равной векторной сумме всех действующих сил:
$$ \vec{F} = \vec{F}_1 + \vec{F}_2 + \dots + \vec{F}_n $$


\section*{2.7 Второй закон Ньютона}

\vspace{-9pt}
\subsection*{Формулировка второго закона Ньютона:}
\vspace{-3pt}
Ускорение тела прямо пропорционально действующей на него силе и обратно пропорционально его массе:
$$ \vec{a} = \frac{\vec{F}}{m} $$
где $\vec{a}$ --- ускорение, $\vec{F}$ --- сила, $m$ --- масса тела.


\section*{2.8 Третий закон Ньютона}

\vspace{-9pt}
\subsection*{Формулировка третьего закона Ньютона:}
\vspace{-3pt}
Силы, с которыми два тела действуют друг на друга, равны по модулю и противоположны по направлению:
$$ \vec{F}_{12} = -\vec{F}_{21} $$
где $\vec{F}_{12} $ --- сила, действующая на первое тело со стороны второго, $\vec{F}_{21} $ --- сила, действующая на второе тело со стороны первого.


\section*{2.9 Закон всемирного тяготения. Искусственные спутники Земли}

\vspace{-9pt}
\subsection*{Формулировка}
\vspace{-3pt}
\textbf{Закон всемирного тяготения:} сила гравитационного притяжения между двумя телами прямо пропорциональна произведению их масс и обратно пропорциональна квадрату расстояния между ними:
$$ F = G \frac{m_1 m_2}{r^2} $$
где $F $ --- сила притяжения, $G $ --- гравитационная постоянная ($G \approx 6.674 \times 10^{-11} \, \text{Н} \cdot \text{м}^2/\text{кг}^2 $), $m_1 $ и $m_2 $ --- массы тел, $r $ --- расстояние между центрами масс тел.

\vspace{-9pt}
\subsection*{Искусственные спутники Земли}
\vspace{-3pt}
Искусственные спутники Земли движутся по орбитам под действием силы гравитационного притяжения Земли. Для круговой орбиты скорость спутника определяется формулой:
$$ v = \sqrt{\frac{GM}{r}} $$
где $v $ --- скорость спутника, $G $ --- гравитационная постоянная, $M $ --- масса Земли, $r $ --- радиус орбиты.


\section*{2.10 Сила тяжести}

\vspace{-9pt}
\subsection*{Определение:}
\vspace{-3pt}
Сила тяжести --- это сила, с которой Земля притягивает тело:
$$ F = mg $$
где $F $ --- сила тяжести, $m $ --- масса тела, $g $ --- ускорение свободного падения ($g \approx 9.81 \, \text{м/с}^2 $).


\section*{2.11 Вес и невесомость}

\vspace{-9pt}
\subsection*{Определение:}
\vspace{-3pt}
\textbf{Вес тела} --- это сила, с которой тело действует на опору или подвес:
$$ P = mg $$
где $P $ --- вес, $m $ --- масса тела, $g $ --- ускорение свободного падения.

\vspace{-9pt}
\subsection*{Определение:}
\vspace{-3pt}
\textbf{Невесомость} --- это состояние, при котором вес тела равен нулю. Это происходит, когда тело движется только под действием силы тяжести (например, в свободном падении или на орбите).


\newpage


\section*{2.12 Сила упругости. Закон Гука}

\vspace{-9pt}
\subsection*{Определение:}
\vspace{-3pt}
Сила упругости --- это сила, возникающая при деформации тела и направленная в сторону, противоположную деформации.

\vspace{-9pt}
\subsection*{Формулировка:}
\vspace{-3pt}
Закон Гука: Сила упругости пропорциональна величине деформации:
$$ F = -kx $$
где $F $ --- сила упругости, $k $ --- коэффициент упругости (жесткость), $x $ --- величина деформации.


\section*{2.13 Сила трения}

\vspace{-9pt}
\subsection*{Определение:}
\vspace{-3pt}
Сила трения --- это сила, возникающая при движении одного тела по поверхности другого и направленная против направления движения. Сила трения скольжения определяется формулой:
$$ F_{\text{тр}} = \mu N $$
где $F_{\text{тр}} $ --- сила трения, $\mu $ --- коэффициент трения, $N $ --- сила нормального давления.


\section*{2.14 Давление}

\vspace{-9pt}
\subsection*{Определение:}
\vspace{-3pt}
Давление --- это физическая величина, равная отношению силы, действующей перпендикулярно поверхности, к площади этой поверхности:
$$ p = \frac{F}{S} $$
где $p $ --- давление, $F $ --- сила, $S $ --- площадь. Единица измерения давления в системе СИ --- паскаль (Па).



\end{document}