\documentclass[a4paper, 12pt]{article} 
\usepackage{fancyhdr}   % Для настройки заголовков и футеров
\usepackage{graphicx}    % Для работы с изображениями (например, логотип)
\usepackage{lipsum}      % Для примера текста
\usepackage{geometry}    % Для настройки полей страницы
\usepackage{mathtext}
\usepackage[T2A]{fontenc}
\usepackage[utf8]{inputenc}
\usepackage[english, russian]{babel}
\usepackage{titlesec}
\usepackage{xcolor}
\usepackage{graphicx}
\usepackage{caption}
\usepackage{subcaption}
\usepackage{wrapfig}
\usepackage{gensymb}
\usepackage{amsmath}
\usepackage{amssymb}
\usepackage{tikz}
\usepackage{circuitikz} 
\usepackage{physics}
\usepackage{enumitem}

\geometry{top=2cm, bottom=2cm, left=1cm, right=1cm}

% Настройка заголовков и футеров
\pagestyle{fancy}
\fancyhf{} % Убираем все стандартные элементы
\fancyfoot[C]{\rule[20pt]{\textwidth}{0.4pt}}
\fancyhead{} % Убираем верхнюю линию заголовка
\fancyfoot[L]{Page \thepage}  % Нумерация страниц по центру
\fancyfoot[R]{\copyright\ Бот для подготовки к ЕГЭ по физике. 2025г}  % Копирайт в правом нижнем углу


\titleformat{\section}[block]{\normalfont\Large\bfseries}{\thesection}{3em}{}
\setlist[itemize]{label=\scriptsize\textbullet}


%~%~%~%~%~%~%~%~%~%~%~%~%~%~%~%~%~%~%~%~%
%          document starts here         %
%~%~%~%~%~%~%~%~%~%~%~%~%~%~%~%~%~%~%~%~%

\begin{document}

\begin{center}
  \scalebox{2}{\textbf{Механические колебания и волны}}
\end{center}


\section*{5.1 Гармонические колебания}
\vspace{-9pt}
\subsection*{Определение}
\vspace{-3pt}
\textbf{Гармонические колебания} — это колебания, при которых физическая величина изменяется по закону синуса или косинуса. Уравнение гармонических колебаний имеет вид:
$$ x(t) = A \cos(\omega t + \phi_0),$$
где:
\begin{itemize}[itemsep=0pt, topsep=0pt, parsep=0pt]
  \item $ x(t)$ — значение колеблющейся величины в момент времени $ t$;
  \item $ A$ — амплитуда колебаний;
  \item $ \omega$ — циклическая частота;
  \item $ \phi_0$ — начальная фаза.
\end{itemize}

\vspace{-9pt}
\subsection*{Формула}
\vspace{-3pt}
Скорость и ускорение при гармонических колебаниях:
$$ v(t) = \frac{dx}{dt} = -A \omega \sin(\omega t + \phi_0),$$
$$ a(t) = \frac{dv}{dt} = -A \omega^2 \cos(\omega t + \phi_0).$$

\section*{5.2 Амплитуда и фаза колебаний}
\vspace{-9pt}
\subsection*{Определение}
\vspace{-3pt}
\textbf{Амплитуда $A$} — это максимальное отклонение колеблющейся величины от положения равновесия. Фаза $ \phi(t) = \omega t + \phi_0$ определяет состояние колебательной системы в данный момент времени.

\section*{5.3 Период колебаний}
\vspace{-9pt}
\subsection*{Определение}
\vspace{-3pt}
\textbf{Период $T$} — это время, за которое система совершает одно полное колебание. Связь с циклической частотой:
$$ T = \frac{2\pi}{\omega}.$$

\section*{5.4 Частота колебаний}
\vspace{-9pt}
\subsection*{Определение}
\vspace{-3pt}
\textbf{Частота $\nu$} — это количество полных колебаний, совершаемых за единицу времени. Связь с периодом:
$$ \nu = \frac{1}{T}.$$
Циклическая частота $\omega$ связана с частотой $\nu$ соотношением:
$$ \omega = 2\pi \nu.$$

\section*{5.5 Свободные колебания}
\vspace{-9pt}
\subsection*{Определение}
\vspace{-3pt}
\textbf{Свободные колебания} — это колебания, происходящие в системе под действием внутренних сил после выведения её из положения равновесия. Уравнение свободных колебаний:
$$ \frac{d^2x}{dt^2} + \omega_0^2 x = 0,$$
где $ \omega_0$ — собственная частота системы.

\section*{5.6 Вынужденные колебания}
\vspace{-9pt}
\subsection*{Определение}
\vspace{-3pt}
\textbf{Вынужденные колебания} — это колебания, возникающие под действием внешней периодической силы. Уравнение вынужденных колебаний:
$$ \frac{d^2x}{dt^2} + 2\beta \frac{dx}{dt} + \omega_0^2 x = F_0 \cos(\omega t),$$
где:
\begin{itemize}[itemsep=0pt, topsep=0pt, parsep=0pt]
  \item $ \beta$ — коэффициент затухания;
  \item $ F_0$ — амплитуда внешней силы;
  \item $ \omega$ — частота внешней силы.
\end{itemize}

\section*{5.7 Резонанс}
\vspace{-9pt}
\subsection*{Определение}
\vspace{-3pt}
\textbf{Резонанс} — это явление резкого увеличения амплитуды колебаний при совпадении частоты внешней силы $ \omega$ с собственной частотой системы $ \omega_0$. Условие резонанса:
$$ \omega = \omega_0.$$

\section*{5.8 Длина волны}
\vspace{-9pt}
\subsection*{Определение}
\vspace{-3pt}
\textbf{Длина волны $\lambda$} — это расстояние, на которое распространяется волна за один период колебаний. Связь с скоростью $ v$ и частотой $ \nu$:
$$ \lambda = \frac{v}{\nu}.$$

\section*{5.9 Звук}
\vspace{-9pt}
\subsection*{Определение}
\vspace{-3pt}
\textbf{Звук} — это механические колебания, распространяющиеся в упругой среде. Скорость звука $ v$ зависит от свойств среды:
$$ v = \sqrt{\frac{E}{\rho}},$$
где:
\begin{itemize}[itemsep=0pt, topsep=0pt, parsep=0pt]
  \item $ E$ — модуль упругости среды;
  \item $ \rho$ — плотность среды.
\end{itemize}

\vspace{-9pt}
\subsection*{Характеристики звука}
\vspace{-3pt}
\begin{enumerate}[itemsep=0pt, topsep=0pt, parsep=0pt]
  \item \textbf{Громкость} — зависит от амплитуды колебаний;
  \item \textbf{Высота тона} — зависит от частоты колебаний;
  \item \textbf{Тембр} — определяется спектром звука.
\end{enumerate}


\end{document}