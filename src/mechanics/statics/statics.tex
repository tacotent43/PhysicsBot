\documentclass[a4paper,12pt]{article}
\usepackage{fancyhdr}   % Для настройки заголовков и футеров
\usepackage{graphicx}    % Для работы с изображениями (например, логотип)
\usepackage{lipsum}      % Для примера текста
\usepackage{geometry}    % Для настройки полей страницы
\usepackage{mathtext}
\usepackage[T2A]{fontenc}
\usepackage[utf8]{inputenc}
\usepackage[english, russian]{babel}
\usepackage{titlesec}
\usepackage{xcolor}
\usepackage{graphicx}
\usepackage{caption}
\usepackage{subcaption}
\usepackage{wrapfig}
\usepackage{gensymb}
\usepackage{amsmath}
\usepackage{amssymb}
\usepackage{tikz}
\usepackage{circuitikz} 
\usepackage{physics}

\usepackage{enumitem}

\geometry{top=2cm, bottom=2cm, left=1cm, right=1cm}

% Настройка заголовков и футеров
\pagestyle{fancy}
\fancyhf{} % Убираем все стандартные элементы
\fancyfoot[C]{\rule[20pt]{\textwidth}{0.4pt}}
\fancyhead{} % Убираем верхнюю линию заголовка
\fancyfoot[L]{Page \thepage}  % Нумерация страниц по центру
\fancyfoot[R]{\copyright\ Бот для подготовки к ЕГЭ по физике. 2025г}  % Копирайт в правом нижнем углу


\titleformat{\section}[block]{\normalfont\Large\bfseries}{\thesection}{3em}{}
% делает itemize точки более тонкими
\setlist[itemize]{label=\scriptsize\textbullet}


%~%~%~%~%~%~%~%~%~%~%~%~%~%~%~%~%~%~%~%~%
%          document starts here         %
%~%~%~%~%~%~%~%~%~%~%~%~%~%~%~%~%~%~%~%~%

\begin{document}

\begin{center}
\scalebox{2}{\textbf{Статика}}
\end{center}

\vspace{-2.5em}

\section*{3.1 Момент силы}
\vspace{-9pt}
\subsection*{Определение:}
\vspace{-3pt}
\textbf{Момент силы (M)} — это физическая величина, характеризующая вращательное действие силы на твердое тело.

\vspace{-9pt}
\subsection*{Формула:}
\vspace{-3pt}
Момент силы:
\vspace{-0.05em}
$$ M = F \cdot d \cdot \sin(\alpha) $$
где:
\begin{enumerate} [itemsep=0pt, topsep=0pt, parsep=0pt]
  \item $F$ — величина силы, приложенной к телу.
  \item $d$ — плечо силы (расстояние от оси вращения до линии действия силы).
  \item $\alpha$ — угол между силой и плечом.
\end{enumerate}

\vspace{-9pt}
\subsection*{Единица измерения:}
\vspace{-3pt}
\begin{itemize} [itemsep=0pt, topsep=0pt, parsep=0pt]
  \item $Н \cdot м$ (Ньютон-метр).
\end{itemize}

\vspace{-9pt}
\subsection*{Направление:}
\vspace{-3pt}
\textbf{Момент силы} — векторная величина, направление которой определяется правилом правого винта (если вращение против часовой стрелки, то направление "от нас", если по часовой — "к нам").


\vspace{-9pt}
\subsection*{Значение:}
\vspace{-3pt}

\textbf{Момент силы} показывает, насколько эффективно сила может вращать тело вокруг определенной оси.




\section*{3.2 Условия равновесия твердого тела}
\vspace{-9pt}
\subsection*{Первое условие (равенство нулю суммы сил):}
\vspace{-3pt}
Сумма всех сил, действующих на твердое тело, должна быть равна нулю. Это гарантирует отсутствие поступательного движения тела:
\begin{enumerate}[itemsep=0pt, topsep=0pt, parsep=2pt]
  \item Сумма сил:
    \vspace{-0.05em}
    $$ \sum{\vec{F}} = 0 $$
  \item Проекции сил:
    \vspace{-0.05em}
    $$ \sum F_x = 0; \quad \sum F_y = 0; \quad \sum F_z = 0 $$
\end{enumerate}

\vspace{-9pt}
\subsection*{Второе условие (равенство нулю суммы моментов сил):}
\vspace{-3pt}
\textbf{Сумма моментов всех сил}, действующих на твердое тело относительно любой произвольной оси вращения, \textbf{должна быть равна нулю}. Это гарантирует отсутствие вращательного движения тела:
$$\sum{\vec{M}} = 0$$

\vspace{-9pt}
\subsection*{Вместе:}
\vspace{-3pt}
Оба условия должны выполняться одновременно для полного равновесия твердого тела.




\section*{3.3 Давление жидкости}
\vspace{-9pt}
\subsection*{Определение:}
\vspace{-3pt}
\textbf{Давление ($P$)} — это физическая величина, характеризующая силу, действующую на единицу площади поверхности перпендикулярно этой поверхности.

\vspace{-9pt}
\subsection*{Формула давления в жидкости на глубине $h$:}
\vspace{-3pt}
\textbf{Единица измерения:} паскаль (Па) или $\frac{H}{M^2}$.
\vspace{-0.05em}
$$ P = \rho \cdot g \cdot h, $$
где:
\begin{enumerate}[itemsep=0pt, topsep=0pt, parsep=2pt]
  \item $\rho$ — плотность жидкости.
  \item $g$ — ускорение свободного падения.
  \item $h$ — глубина погружения в жидкость (расстояние от поверхности жидкости до рассматриваемой точки).
\end{enumerate}



\section*{3.4 Закон Паскаля}
\vspace{-9pt}
\subsection*{Формулировка:}
\vspace{-3pt}
Давление, производимое на жидкость или газ, передается одинаково во всех направлениях.

\vspace{-9pt}
\subsection*{Применение:}
\vspace{-3pt}
Лежит в основе работы гидравлических машин (прессов, подъемников и т.д.).

\vspace{-9pt}
\subsection*{Суть:}
\vspace{-3pt}
Изменение давления в одной точке жидкости мгновенно передается во все другие точки жидкости.



\section*{3.5 Закон Архимеда}
\vspace{-9pt}
\subsection*{Формулировка:}
\vspace{-3pt}
На тело, погруженное в жидкость или газ, действует выталкивающая сила, равная весу жидкости или газа, вытесненного этим телом.

\vspace{-9pt}
\subsection*{Формула:}
\vspace{-3pt}
\vspace{-0.05em}
$$ F_A = \rho \cdot g \cdot V $$
где:
\begin{enumerate} [itemsep=0pt, topsep=0pt, parsep=0pt]
  \item $F_A$ — сила Архимеда.
  \item $\rho$ — плотность жидкости или газа.
  \item $g$ — ускорение свободного падения.
  \item $V$ — объем погруженной части тела (объем вытесненной жидкости или газа).
\end{enumerate}
\vspace{10pt}
Направлена всегда вертикально вверх.


\newpage


\section*{3.6 Условия плавания тел}
\vspace{-9pt}
\subsection*{Условие плавания (тело плавает на поверхности):}
\vspace{-3pt}
Если средняя плотность тела меньше плотности жидкости, то тело плавает на поверхности. В этом случае сила тяжести, действующая на тело, равна силе Архимеда.

\vspace{-9pt}
\subsection*{Условие плавания (тело плавает в толще жидкости):}
\vspace{-3pt}
Если средняя плотность тела равна плотности жидкости, то тело плавает в толще жидкости. Сила тяжести, действующая на тело, равна силе Архимеда.

\vspace{-9pt}
\subsection*{Условие утопления (тело тонет):}
\vspace{-3pt}
Если средняя плотность тела больше плотности жидкости, то тело тонет. Сила тяжести, действующая на тело, больше силы Архимеда.

\vspace{-9pt}
\subsection*{Формулы:}
\vspace{-3pt}
\begin{enumerate}[itemsep=0pt, topsep=0pt, parsep=2pt]
  \item Плавание на поверхности:
    \vspace{-0.05em}
    $$ m \cdot g = \rho \cdot g \cdot V_{погруж} $$
    где $V_{погруж}$ — объем погруженной части тела.
  \item Процент погружения:
    \vspace{-0.05em}
    $$ \frac{V_{погруж}}{V_{общ}} = \frac{\rho_{тела}}{\rho_{жидкости}} \cdot 100\% $$
\end{enumerate}

\end{document}
