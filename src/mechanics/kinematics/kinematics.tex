\documentclass[a4paper,12pt]{article}
\usepackage{fancyhdr}   % Для настройки заголовков и футеров
\usepackage{graphicx}    % Для работы с изображениями (например, логотип)
\usepackage{lipsum}      % Для примера текста
\usepackage{geometry}    % Для настройки полей страницы
\usepackage{mathtext}
\usepackage[T2A]{fontenc}
\usepackage[utf8]{inputenc}
\usepackage[english, russian]{babel}
\usepackage{titlesec}
\usepackage{xcolor}
\usepackage{graphicx}
\usepackage{caption}
\usepackage{subcaption}
\usepackage{wrapfig}
\usepackage{gensymb}
\usepackage{amsmath}
\usepackage{amssymb}
\usepackage{tikz}
\usepackage{circuitikz} 
\usepackage{physics}

\usepackage{enumitem}

\geometry{top=2cm, bottom=2cm, left=1cm, right=1cm}

% Настройка заголовков и футеров
\pagestyle{fancy}
\fancyhf{} % Убираем все стандартные элементы
\fancyfoot[C]{\rule[20pt]{\textwidth}{0.4pt}}
\fancyhead{} % Убираем верхнюю линию заголовка
\fancyfoot[L]{Page \thepage}  % Нумерация страниц по центру
\fancyfoot[R]{\copyright\ Бот для подготовки к ЕГЭ по физике. 2025г}  % Копирайт в правом нижнем углу


\titleformat{\section}[block]{\normalfont\Large\bfseries}{\thesection}{3em}{}
\setlist[itemize]{label=\scriptsize\textbullet}


%~%~%~%~%~%~%~%~%~%~%~%~%~%~%~%~%~%~%~%~%
%          document starts here         %
%~%~%~%~%~%~%~%~%~%~%~%~%~%~%~%~%~%~%~%~%

\begin{document}

\begin{center}
\scalebox{2}{\textbf{Кинематика}}
\end{center}

\subsection*{Определение 1}
\vspace{-3pt}
\textbf{Кинематика} - это раздел механики, в котором изучается механическое движение тел без учёта причин, вызывающих это движение.

\vspace{-10pt}

\subsection*{Определение 2}
\vspace{-3pt}
\textbf{Материальная точка} - тело, обладающее массой, размерами которого в данной задаче можно пренебречь, если:
\begin{enumerate}[itemsep=0pt, topsep=0pt, parsep=3pt]
  \item расстояние, которое проходит тело, во много раз больше его размера;
  \item расстояние от данного тела до другого тела много больше его размера;
  \item тело движется поступательно.
\end{enumerate}

\vspace{-10pt}

\subsection*{Определение 3}
\vspace{-3pt}
\textbf{Система отсчёта} - это тело отсчета, связанная с ним система координат и прибор для измерения времени. Траектория - это линия, которую описывает тело при своем движении.

\vspace{-10pt}

\subsection*{Определение 4}
\vspace{-3pt}
\textbf{Путь} - это скалярная величина, равная длине траектории.   

\vspace{-10pt}

\subsection*{Определение 5}
\vspace{-3pt}
\textbf{Перемещение} - это вектор, соединяющий начальное положение тела с его конечным положением за данный промежуток времени

\vspace{10pt}


% 1.1
% Название главы
\section*{1.1 Механическое движение и его виды}

% Оформление для \subsection
\vspace{-9pt}
\subsection*{Определение:}
\vspace{-3pt}
% Конец оформления для \subsection

\textbf{Механическое движение} - это изменение положения тела в пространстве относительно других тел с течением времени.

\vspace{-9pt}
\subsection*{Основные виды:}
\vspace{-3pt}

\begin{enumerate} [itemsep=0pt, topsep=0pt, parsep=3pt]
  \item \textbf{Поступательное движение:} движение, при котором все точки тела движутся одинаково, например, движение автомобиля по прямой дороге.
  \item \textbf{Вращательное движение:} движение, при котором все точки тела движутся по окружностям относительно одной оси, например, вращение колеса.
  \item \textbf{Колебательное движение:} движение, при котором тело многократно повторяет свое движение вблизи определенного положения равновесия, например, качание маятника.
  \item \textbf{Комбинированные движения:} сочетание нескольких видов движения, например, движение катящегося колеса (поступательное и вращательное).
\end{enumerate}


% new page
\newpage


% 1.2
\section*{1.2 Относительность механического движения}

\vspace{-9pt}
\subsection*{Примеры:}
\vspace{-3pt}
\begin{enumerate} [itemsep=0pt, topsep=0pt, parsep=3pt]
  \item Человек в движущемся поезде покоится относительно поезда, но движется относительно земли.
  \item Скорость автомобиля, измеренная относительно наблюдателя на обочине, отличается от скорости этого же автомобиля, измеренной относительно другого движущегося автомобиля.
\end{enumerate}
Система отсчёта включает в себя тело отсчёта, связанную с ним систему координат и часы. 
$$ x = x_0 + S_x $$


% 1.3
\section*{1.3 Скорость}

\vspace{-9pt}
\subsection*{Определение:}
\vspace{-3pt}
\textbf{Скорость} — это физическая величина, характеризующая быстроту изменения положения тела в пространстве.
\vspace{-9pt}
\subsection*{Обозначение и единицы измерения:}
\vspace{-3pt}
\begin{enumerate} [itemsep=0pt, topsep=0pt, parsep=3pt]
  \item Обозначение: $v$
  \item Единица измерения в СИ: $\frac{м}{с}$ (метр в секунду).
\end{enumerate}
\vspace{-9pt}
\subsection*{Виды скорости:}
\vspace{-3pt}
\begin{enumerate} [itemsep=0pt, topsep=0pt, parsep=3pt]
  \item \textbf{Средняя скорость:} отношение полного пройденного пути к общему времени движения.
  \item \textbf{Мгновенная скорость:} скорость тела в данный момент времени в данной точке траектории. Является векторной величиной (имеет направление и величину).
\end{enumerate}


% 1.4
\section*{1.4 Ускорение}

\vspace{-9pt}
\subsection*{Определение:}
\vspace{-3pt}
\textbf{Ускорение} - это физическая величина, характеризующая быстроту изменения скорости тела.
\vspace{-9pt}
\subsection*{Обозначение и единицы измерения:}
\vspace{-3pt}
\begin{enumerate} [itemsep=0pt, topsep=0pt, parsep=3pt]
  \item Обозначение: $a$
  \item Единица измерения в СИ: $\frac{м}{s^2}$ (метр на секунду в квадрате).
\end{enumerate}
Является векторной величиной (имеет направление и величину).
$$ \vec{a} = \frac{\vec{v} - \vec{v_0}}{t} $$


% new page
\newpage


% 1.5
\section*{1.5 Равномерное движение}

\vspace{-9pt}
\subsection*{Определение:}
\vspace{-3pt}
\textbf{Равномерное движение} — это движение, при котором тело за любые равные промежутки времени проходит равные пути, то есть движется с постоянной по модулю скоростью.
\vspace{-9pt}
\subsection*{Характеристики:}
\vspace{-3pt}
\begin{enumerate} [itemsep=0pt, topsep=0pt, parsep=3pt]
  \item Скорость постоянна ($v = const$).
  \item Ускорение равно нулю ($a = 0$).
  \item График зависимости перемещения от времени — прямая линия.
\end{enumerate}
\vspace{-9pt}
\subsection*{Пример:}
\vspace{-3pt}
Движение автомобиля по прямой дороге с постоянной скоростью (если пренебречь колебаниями скорости).
\vspace{-9pt}
\subsection*{Формулы:}
\vspace{-3pt}
\begin{enumerate}
  \item Формула перемещения для равномерного движения:
    \vspace{-0.05em}
    $$ \vec{S} = \vec{v}t $$ 
  \item Формула для скорости при равномерном движении:
    \vspace{-0.05em}
    $$ v = \frac{S}{t} $$ 
  \item Уравнение координаты:
    \vspace{-0.05em}
    $$ x(t) = x_0 + v \cdot t $$ 
\end{enumerate}


% 1.6
\section*{1.6 Прямолинейное равноускоренное движение}

\vspace{-9pt}
\subsection*{Определение:}
\vspace{-3pt}
\textbf{Прямолинейное равноускоренное движение} — это движение по прямой с постоянным ускорением, то есть скорость меняется на одну и ту же величину за равные промежутки времени.
\vspace{-9pt}
\subsection*{Характеристики:}
\vspace{-3pt}
\begin{enumerate}[itemsep=0pt, topsep=0pt, parsep=3pt]
  \item Ускорение постоянно: $a = \text{const}$.
  \item Скорость изменяется линейно со временем.
  \item График зависимости скорости от времени — прямая линия.
\end{enumerate}
\vspace{-9pt}
\subsection*{Формулы:}
\vspace{-3pt}
\begin{enumerate} [itemsep=0pt, topsep=0pt, parsep=3pt]
  \item Уравнение скорости:
    \vspace{-0.05em}
    $$ v = v_0 + at $$ 
  \item Уравнение перемещения:
    \vspace{-0.05em}
    $$ S = v_0 t + \frac{at^2}{2} $$ 
\end{enumerate}
Также можно выразить \textbf{перемещение} через \textbf{скорости}:
\vspace{-0.05em}
\[
S = \frac{v^2 - v_0^2}{2a}
\]


% 1.7
\section*{1.7 Свободное падение}

\vspace{-9pt}
\textbf{Свободное падение} — это движение тела под действием только силы тяжести (без учета сопротивления воздуха).
\vspace{-9pt}
\subsection*{Характеристики:}
\vspace{-3pt}
\begin{enumerate} [itemsep=0pt, topsep=0pt, parsep=3pt]
  \item Является равноускоренным движением.
  \item Ускорение свободного падения ($g$) направлено вертикально вниз (приблизительно равно $9.8 \, м/с^2$ на поверхности Земли).
  \item Скорость увеличивается пропорционально времени.
\end{enumerate}
\vspace{-9pt}
\subsection*{Уравнения:}
\vspace{-3pt}
Уравнения кинематики, применимые к равноускоренному движению, с учетом ускорения свободного падения:
\begin{enumerate} [itemsep=0pt, topsep=0pt, parsep=3pt]
  \item $v = v_0 + gt$
  \item $S = v_0 t + \frac{gt^2}{2}$
  \item $S = \frac{v^2 - v_0^2}{2g}$
  \item $y = y_0 + v_0 t + \frac{at^2}{2}$
\end{enumerate}


% 1.8
\section*{1.8 Движение по окружности с постоянной по модулю скоростью. Центростремительное ускорение.}

\vspace{-9pt}
Движение по окружности с постоянной по модулю скоростью — это движение тела по траектории окружности, при котором величина скорости остается неизменной, но направление скорости постоянно меняется.
\vspace{-9pt}
\subsection*{Центростремительное ускорение:}
\vspace{-3pt}
Ускорение, которое направлено к центру окружности и вызывает изменение направления скорости.
\vspace{-9pt}
\subsection*{Формула:}
\vspace{-3pt}
$$a_ц = \frac{v^2}{r}$$ где $v$ — скорость, $r$ — радиус окружности.
\vspace{-9pt}
\subsection*{Примеры:}
\vspace{-3pt}
\begin{enumerate} [itemsep=0pt, topsep=0pt, parsep=3pt]
  \item Движение спутника вокруг Земли (при условии постоянной по модулю скорости).
  \item Движение шарика на нитке.
\end{enumerate}



\end{document}