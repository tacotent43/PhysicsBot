\documentclass[a4paper,12pt]{article}
\usepackage{fancyhdr}
\usepackage{graphicx} 
\usepackage{lipsum}
\usepackage{geometry}
\usepackage{mathtext}
\usepackage[T2A]{fontenc}
\usepackage[utf8]{inputenc}
\usepackage[english, russian]{babel}
\usepackage{titlesec}
\usepackage{xcolor}
\usepackage{graphicx}
\usepackage{caption}
\usepackage{subcaption}
\usepackage{wrapfig}
\usepackage{gensymb}
\usepackage{amsmath}
\usepackage{amssymb}
\usepackage{tikz}
\usepackage{circuitikz} 
\usepackage{physics}

\usepackage{enumitem}

\geometry{top=2cm, bottom=2cm, left=1cm, right=1cm}

\pagestyle{fancy}
\fancyhf{} 
\fancyfoot[C]{\rule[20pt]{\textwidth}{0.4pt}}
\fancyhead{} 
\fancyfoot[L]{Page \thepage} 
\fancyfoot[R]{\copyright\ Бот для подготовки к ЕГЭ по физике. 2025г} 

\titleformat{\section}[block]{\normalfont\Large\bfseries}{\thesection}{3em}{}
\setlist[itemize]{label=\scriptsize\textbullet}


\begin{document}

\begin{center}
\scalebox{2}{\textbf{Основы специальной теории относительности}}
\end{center}

\vspace{-2.5em}

%%%%%%%%%%%%%%%%%
% \section*{}


% \vspace{-9pt}
% \subsection*{}
% \vspace{-3pt}


% \begin{enumerate} [itemsep=0pt, topsep=0pt, parsep=0pt]
%   \item $p$ – импульс тела;
%   \item $m$ – масса тела;
%   \item $v$ – скорость тела.
% \end{enumerate}


% [itemsep=0pt, topsep=0pt, parsep=0pt]
%%%%%%%%%%%%%%%%%
\section*{4.1 Инвариантность скорости света. Принцип относительности Эйнштейна}
\vspace{-9pt}
\subsection*{Принцип относительности Эйнштейна:}
\vspace{-3pt}
\begin{itemize}
    \item \textbf{Суть:} Законы физики одинаковы во всех инерциальных системах отсчёта.
    \item \textbf{Пояснение:} Инерциальная система отсчёта — это система, которая движется прямолинейно и равномерно (без ускорения) относительно другой инерциальной системы. Проще говоря, если вы находитесь в поезде, который едет с постоянной скоростью по прямой, то все физические законы внутри поезда будут работать так же, как если бы вы стояли на перроне. Никакой физический эксперимент, проведённый внутри поезда, не сможет выявить факт его движения.
    \item \textbf{Важность:} Этот принцип говорит о том, что не существует "абсолютной" системы отсчёта, относительно которой можно было бы определить "абсолютную" скорость объекта. Скорость всегда относительна и зависит от выбора системы отсчёта.
\end{itemize}

\vspace{-9pt}
\subsection*{Инвариантность скорости света:}
\vspace{-3pt}
\begin{itemize}
    \item \textbf{Суть:} Скорость света в вакууме (обозначается как $c$ и приблизительно равна $3 \cdot 10^8$ м/с) одинакова для всех наблюдателей во всех инерциальных системах отсчёта, независимо от скорости источника света или наблюдателя.
    \item \textbf{Пояснение:} Это контринтуитивно. Представьте, что вы едете в машине и включаете фары. Согласно классической физике, скорость света от фар для вас будет равна $c$, а для наблюдателя на обочине дороги — $c +$ скорость вашей машины. СТО утверждает, что и для вас, и для наблюдателя скорость света будет одинакова и равна $c$.
    \item \textbf{Важность:} Этот постулат радикально меняет наше понимание пространства и времени, поскольку он противоречит классической механике Ньютона, в которой скорости складываются. Из этого постулата вытекают такие следствия, как замедление времени, сокращение длины и увеличение массы.
\end{itemize}

\vspace{-9pt}
\subsection*{Влияние двух принципов:}
\vspace{-3pt}
Совмещение этих двух принципов ведет к революционным последствиям, которые противоречат интуитивному пониманию пространства и времени:
\begin{itemize}
    \item \textbf{Одновременность относительна:} События, которые кажутся одновременными одному наблюдателю, могут быть неодновременными для другого, движущегося относительно первого.
    \item \textbf{Замедление времени:} Время течет медленнее в движущейся системе отсчета по сравнению с покоящейся.
    \item \textbf{Сокращение длины:} Длина объекта в направлении движения сокращается в движущейся системе отсчета по сравнению с покоящейся.
\end{itemize}


\section*{4.2 Полная энергия}
\vspace{-9pt}
\subsection*{Полная энергия ($E$):}
\vspace{-3pt}
Это сумма энергии покоя и кинетической энергии. Полная энергия релятивистской частицы определяется уравнением:
\vspace{-0.05em}
$$ E = \frac{mc^2}{\sqrt{1 - \frac{v^2}{c^2}}} $$
где:
\begin{itemize}
    \item $m$ — масса частицы.
    \item $v$ — скорость частицы.
    \item $c$ — скорость света.
\end{itemize}

\vspace{-9pt}
\subsection*{Кинетическая энергия ($E_k$):}
\vspace{-3pt}
Это разница между полной энергией и энергией покоя.


\section*{4.3 Связь массы и энергии. Энергия покоя}
\vspace{-9pt}
\subsection*{Связь массы и энергии ($E = mc^2$):}
\vspace{-3pt}
Самое знаменитое уравнение в физике. Оно выражает глубокую связь между массой и энергией.
\begin{itemize}
    \item \textbf{Суть:} Масса и энергия эквивалентны. Небольшое количество массы может быть преобразовано в огромное количество энергии и наоборот.
    \item \textbf{Пояснение:} Эта формула означает, что любая масса, даже в состоянии покоя, обладает "скрытой" энергией, называемой энергией покоя.
    \item \textbf{Примеры:}
    \begin{itemize}
        \item Ядерные реакции (ядерное оружие, атомные электростанции) используют преобразование массы в энергию.
        \item Энергия Солнца также получается из-за преобразования массы в энергию в процессе термоядерного синтеза.
    \end{itemize}
\end{itemize}

\vspace{-9pt}
\subsection*{Энергия покоя ($E_0$):}
\vspace{-3pt}
Это энергия, которой обладает объект даже в состоянии покоя.
\begin{itemize}
    \item \textbf{Формула:}
    \vspace{-0.05em}
    $$ E_0 = mc^2 $$
    \item \textbf{Пояснение:} Даже неподвижный объект (например, камень) обладает энергией, которая может быть высвобождена (хотя обычно это очень сложно сделать). Эта энергия является энергией, связанной с его массой.
\end{itemize}

\vspace{-9pt}
\subsection*{Основные выводы:}
\vspace{-3pt}
\begin{itemize}
    \item СТО революционизировала наше понимание пространства, времени, массы и энергии.
    \item Скорость света является предельной скоростью, которую может достичь любой объект, имеющий массу.
    \item Масса и энергия — это взаимозаменяемые сущности.
    \item СТО является фундаментальной теорией, которая используется во многих областях науки и техники, особенно в физике высоких энергий и астрофизике.
\end{itemize}
\end{document}