\documentclass[a4paper,12pt]{article}
\usepackage{fancyhdr}
\usepackage{graphicx} 
\usepackage{lipsum}
\usepackage{geometry}
\usepackage{mathtext}
\usepackage[T2A]{fontenc}
\usepackage[utf8]{inputenc}
\usepackage[english, russian]{babel}
\usepackage{titlesec}
\usepackage{xcolor}
\usepackage{graphicx}
\usepackage{caption}
\usepackage{subcaption}
\usepackage{wrapfig}
\usepackage{gensymb}
\usepackage{amsmath}
\usepackage{amssymb}
\usepackage{tikz}
\usepackage{circuitikz} 
\usepackage{physics}

\usepackage{enumitem}

\geometry{top=2cm, bottom=2cm, left=1cm, right=1cm}

\pagestyle{fancy}
\fancyhf{} 
\fancyfoot[C]{\rule[20pt]{\textwidth}{0.4pt}}
\fancyhead{} 
\fancyfoot[L]{Page \thepage} 
\fancyfoot[R]{\copyright\ Бот для подготовки к ЕГЭ по физике. 2025г} 

\titleformat{\section}[block]{\normalfont\Large\bfseries}{\thesection}{3em}{}
\setlist[itemize]{label=\scriptsize\textbullet}

\begin{document}

\begin{center}
\scalebox{2}{\textbf{Молекулярная физика}}
\end{center}

\vspace{-2.5em}

%%%%%%%%%%%%%%%%%
% \section*{}


% \vspace{-9pt}
% \subsection*{}
% \vspace{-3pt}


% \begin{enumerate}[itemsep=0pt, topsep=0pt, parsep=0pt]
%   \item $p$ – импульс тела;
%   \item $m$ – масса тела;
%   \item $v$ – скорость тела.
% \end{enumerate}
%%%%%%%%%%%%%%%%%


% 1.1
\section*{1.1 Модели строения газов, жидкостей и твердых тел}

\vspace{-9pt}
\subsection*{Газы:}
\vspace{-3pt}
\begin{enumerate}[itemsep=0pt, topsep=0pt, parsep=0pt]
  \item Молекулы расположены на больших расстояниях друг от друга (по сравнению с размерами молекул);
  \item Молекулы находятся в хаотическом движении;
  \item Силы взаимодействия между молекулами слабые;
  \item Занимают весь предоставленный объем и легко сжимаемы.
\end{enumerate}

\vspace{-9pt}
\subsection*{Жидкости:}
\vspace{-3pt}
\begin{enumerate}[itemsep=0pt, topsep=0pt, parsep=0pt]
  \item Молекулы расположены близко друг к другу, но не упорядоченно;
  \item Молекулы также находятся в хаотичном движении, но колебательном;
  \item Силы взаимодействия между молекулами значительные, удерживают их в "своем" объеме; 
  \item Сохраняют свой объем, но принимают форму сосуда, обладают текучестью.
\end{enumerate}

\vspace{-9pt}
\subsection*{Твёрдые тела:}
\vspace{-3pt}
\begin{enumerate} [itemsep=0pt, topsep=0pt, parsep=0pt]
  \item Молекулы (атомы, ионы) расположены в определенном порядке, образуя кристаллическую решетку (или аморфную структуру);
   \item Молекулы совершают колебательные движения относительно своих положений равновесия;
    \item Силы взаимодействия между молекулами очень сильные; 
  \item Сохраняют свою форму и объем.
\end{enumerate}


% 1.2
\section*{1.2 Тепловое движение атомов и молекул вещества}

\vspace{-9pt}
\subsection*{Определение:}
\vspace{-3pt}
\textbf{Тепловое движение} --- беспорядочное движение атомов и молекул, составляющих вещество. Интенсивность этого движения зависит от температуры: чем выше температура, тем быстрее движутся частицы.

\vspace{-9pt}
\subsection*{Свойства:}
\vspace{-3pt}
\begin{enumerate}[itemsep=0pt, topsep=0pt, parsep=0pt]
  \item Тепловое движение непрерывно;
  \item Тепловое движение хаотично (беспорядочно).
\end{enumerate}


% 1.3
\section*{1.3 Броуновское движение}

\vspace{-9pt}
\subsection*{Определение:}
\vspace{-3pt}
\textbf{Броуновское движение} - беспорядочное движение мелких частиц, взвешенных в жидкости или газе.

\vspace{-9pt}
\subsection*{Причина:}
\vspace{-3pt}
Толчки со стороны окружающих молекул, находящихся в тепловом движении.

\vspace{-9pt}
\subsection*{Доказательство:}
\vspace{-3pt}
Является наглядным свидетельством хаотического теплового движения молекул.


\newpage


% 1.4
\section*{1.4 Диффузия}

\vspace{-9pt}
\subsection*{Определение:}
\vspace{-3pt}
\textbf{Диффузия} --- явление проникновения молекул одного вещества в другое вследствие их теплового движения.

\vspace{-9pt}
\subsection*{Примеры:}
\vspace{-3pt}
Распространение запаха, растворение сахара в воде.

\vspace{-9pt}
\subsection*{Скорость диффузии:}
\vspace{-3pt}
Зависит от температуры (чем выше, тем быстрее) и агрегатного состояния (в газах диффузия быстрее).


% 1.5
\section*{1.5 Экспериментальные доказательства атомистической теории. Взаимодействие частиц вещества}

\vspace{-9pt}
\subsection*{Экспериментальные доказательства:}
\vspace{-3pt} 
Броуновское движение, диффузия, испарение, явления поверхностного натяжения, химические реакции (закон сохранения массы), электролиз, рентгеновские исследования кристаллических решеток.

\vspace{-9pt}
\subsection*{Взаимодействие частиц:}
\vspace{-3pt}
\begin{enumerate} [itemsep=0pt, topsep=0pt, parsep=0pt]
    \item \textbf{Притяжение} — это сила, которая действует между двумя или более телами, заставляя их приближаться друг к другу. В физике притяжение часто связано с гравитационными или электромагнитными взаимодействиями. Например, гравитационное притяжение между Землёй и объектами на её поверхности вызывает их вес.
    \item \textbf{Отталкивание} — это сила, которая действует между двумя или более телами, заставляя их удаляться друг от друга. Отталкивание может возникать, например, между одноимённо заряженными частицами (электростатическое отталкивание) или между магнитами с одинаковыми полюсами.
    \item \textbf{Результирующая сила} (или равнодействующая сила) — это векторная сумма всех сил, действующих на тело. Если на тело действует несколько сил, то результирующая сила определяется как:
    $$ \vec{F}_{\text{рез}} = \vec{F}_1 + \vec{F}_2 + \dots + \vec{F}_n, $$
    где $\vec{F}_1, \vec{F}_2, \dots, \vec{F}_n$ — силы, действующие на тело. Результирующая сила определяет ускорение тела в соответствии со вторым законом Ньютона:
    $$\vec{F}_{\text{рез}} = m \vec{a},$$
    где $m$ — масса тела, а $\vec{a}$ — его ускорение.
\end{enumerate}
    
    
\newpage


% 1.6
\section*{1.6 Модель идеального газа}

\vspace{-9pt}
\subsection*{Определение:}
\vspace{-3pt}
\textbf{Идеальный газ} - упрощенная модель газа, в которой пренебрегают:
\begin{enumerate} [itemsep=0pt, topsep=0pt, parsep=0pt]
    \item Размерами молекул (считают их материальными точками);
    \item Взаимодействием между молекулами (кроме моментов столкновений);
    \item Столкновения молекул со стенками сосуда считаются абсолютно упругими.
\end{enumerate}
Упрощение позволяет получить простые законы, описывающие поведение газов.


% 1.7
\section*{1.7 Связь между давлением и средней кинетической энергией теплового движения молекул идеального газа}


\vspace{-9pt}
\subsection*{Основное уравнение МКТ идеального газа:}
\vspace{-3pt}
$$ p = \frac{2}{3} \cdot n \cdot E_{k_{средн}}, $$
где:
\begin{itemize} [itemsep=0pt, topsep=0pt, parsep=0pt, label=\textbullet, font=\normalfont]
    \item $P$ – давление газа;
    \item $n$ – концентрация молекул (число молекул в единице объема);
    \item $E_{k_{средн}}$ – средняя кинетическая энергия поступательного движения одной молекулы.
\end{itemize}


% 1.8
\section*{1.8 Абсолютная температура}

\vspace{-9pt}
\subsection*{Абсолютная температура ($T$):}
\vspace{-3pt}
\textbf{Температура, отсчитываемая от абсолютного нуля (0 $К$ или -273.15 $\degree C$).}

\vspace{-9pt}
\subsection*{Шкала Кельвина:}
\vspace{-3pt}
\textbf{Температурная шкала, где за 0 $К$ принята температура абсолютного нуля.}

\vspace{-9pt}
\subsection*{ Перевод в шкалу Цельсия:}
\vspace{-3pt}
$$T(K) = t(\degree C) + 273.15$$


% 1.9
\section*{1.9 Связь температуры газа со средней кинетической энергией его частиц}

\vspace{-9pt}
\subsection*{Средняя кинетическая энергия молекул идеального газа:}
\vspace{-3pt}
$$E_{k_{средн}} = \frac{3}{2} \cdot k\cdot T,$$ 
где:
\begin{enumerate}
    \item $k$ – постоянная Больцмана $\approx 1.38 \cdot 10^{-23}$ Дж/К
    \item $T$ – абсолютная температура.
\end{enumerate}

\vspace{-9pt}
\subsection*{Вывод:}
\vspace{-3pt}
Средняя кинетическая энергия молекул пропорциональна абсолютной температуре.




\newpage


% 1.10
\section*{1.10 Уравнение состояния идеального газа (Клапейрона)}

\vspace{-9pt}
\subsection*{Уравнение Клапейрона:}
\vspace{-3pt}
$$p \cdot V = const$$ 
\vspace{-3pt} 
(для данной массы газа)

\vspace{-9pt}
\subsection*{Связь:}
\vspace{-3pt} Устанавливает связь между давлением, объемом и температурой идеального газа


% 1.11
\section*{1.11 Уравнение Менделеева–Клапейрона}

\vspace{-9pt}
\subsection*{Уравнение Менделеева – Клапейрона:}
\vspace{-3pt}
$$p \cdot V = \nu \cdot R \cdot T$$ 
где:
\begin{itemize}
    \item $p$ – давление газа,
    \item $V$ – объем газа,
    \item $\nu$ – количество вещества (число молей),
    \item $p$ – $R$ – универсальная газовая постоянная $\approx$ 8.31 Дж/(моль$\cdot$К)),
    \item $T$ – абсолютная температура.
\end{itemize}

\vspace{-9pt}
\subsection*{Форма:}
\vspace{-3pt}
$p \cdot V = \frac{m}{M}\cdot R\cdot T$, 
где: 
\begin{enumerate}
    \item $m$ – масса газа,
    \item $M$ – молярная масса газа.
\end{enumerate}

\vspace{-9pt}
\subsection*{Использование:}
\vspace{-3pt}
Позволяет рассчитать параметры состояния газа.



% 1.12
\newpage
\section*{1.12 Изопроцессы}

\vspace{-9pt}
\subsection*{Определения:}
\vspace{-3pt}
\textbf{Изопроцесс} - это термодинамический процесс, происходящий в физической системе при постоянном значении одного из её параметров состояния. В зависимости от того, какой параметр остаётся неизменным, выделяют следующие основные виды изопроцессов:
\begin{enumerate} 
    \item \textbf{Изотермический процесс} --- процесс, происходящий при постоянной температуре (\( T = \text{const} \));
    \item \textbf{Изобарный процесс} --- процесс, происходящий при постоянном давлении (\( p = \text{const} \));
    \item \textbf{Изохорный процесс} --- процесс, происходящий при постоянном объёме (\( V = \text{const} \)). 
\end{enumerate}
Каждый из этих процессов описывается соответствующими законами термодинамики, например:
\begin{itemize}
    \item Для изотермического процесса справедлив \textbf{закон Бойля-Мариотта}: \( pV = \text{const} \).
    \item Для изобарного процесса --- \textbf{закон Гей-Люссака}: \( \frac{V}{T} = \text{const} \).
    \item Для изохорного процесса --- \textbf{закон Шарля}: \( \frac{p}{T} = \text{const} \).
\end{itemize}
Изопроцессы широко используются для анализа поведения идеальных газов и других термодинамических систем.



% 1.13
\section*{1.13 Насыщенные и ненасыщенные пары}

\vspace{-9pt}
\subsection*{Определения:}
\vspace{-3pt}
\begin{enumerate} [itemsep=0pt, topsep=0pt, parsep=0pt]
    \item \textbf{Пар} --- это газообразное состояние вещества, находящееся при температуре ниже его критической температуры. Пар может быть как насыщенным, так и ненасыщенным, в зависимости от условий, в которых он находится.
    \item \textbf{Насыщенный пар} — это пар, находящийся в динамическом равновесии с жидкой или твёрдой фазой того же вещества. При данной температуре давление насыщенного пара является максимальным, и дальнейшее испарение жидкости или сублимация твёрдого тела прекращаются. Насыщенный пар характеризуется тем, что количество молекул, покидающих поверхность жидкости (или твёрдого тела), равно количеству молекул, возвращающихся обратно.
    \item \textbf{Ненасыщенный пар} — это пар, давление которого ниже давления насыщенного пара при данной температуре. В таком состоянии вещество может продолжать испаряться, так как количество молекул, покидающих жидкость, превышает количество молекул, возвращающихся обратно.
    \item \textbf{Динамическое равновесие} — это состояние системы, при котором скорость прямого процесса (например, испарения) равна скорости обратного процесса (например, конденсации). В случае насыщенного пара динамическое равновесие означает, что количество молекул, переходящих из жидкости в пар, равно количеству молекул, переходящих из пара в жидкость, при этом макроскопические параметры системы (давление, температура, объём) остаются постоянными.
\end{enumerate}


% 1.14
\newpage
\section*{1.14 Влажность воздуха}
\vspace{-9pt}
\subsection*{Определения:}
\vspace{-3pt}
\begin{enumerate}
    \item \textbf{Влажность воздуха} --- это величина, характеризующая содержание водяного пара в воздухе. Она может быть выражена как абсолютная или относительная влажность.
    \item \textbf{Абсолютная влажность} --- это масса водяного пара, содержащегося в единице объёма воздуха. Обозначается символом $\rho$ и измеряется в $\text{г/м}^3$.
        $$ \rho = \frac{m_{\text{пара}}}{V_{\text{воздуха}}} $$
    \item \textbf{Относительная влажность} — это отношение абсолютной влажности воздуха к максимально возможной абсолютной влажности при данной температуре, выраженное в процентах. Обозначается символом $\varphi$.
        $$ \varphi = \frac{\rho}{\rho_{\text{max}}} \times 100\% $$
\end{enumerate}


% 1.15
\section*{1.15 Изменение агрегатных состояний вещества}

\vspace{-9pt}
\subsection*{Испарение:}
\vspace{-3pt}
Переход вещества из жидкого состояния в газообразное (происходит при любой температуре).

\vspace{-9pt}
\subsection*{Конденсация:}
\vspace{-3pt}
Переход вещества из газообразного состояния в жидкое.

\vspace{-9pt}
\subsection*{Кипение:}
\vspace{-3pt}
Интенсивное испарение, происходящее по всему объему жидкости при определенной температуре (температуре кипения). Температура кипения зависит от давления.

\vspace{-9pt}
\subsection*{Плавление:}
\vspace{-3pt}
Переход вещества из твердого состояния в жидкое (происходит при определенной температуре – температуре плавления).

\vspace{-9pt}
\subsection*{Кристаллизация:}
\vspace{-3pt}
Переход вещества из жидкого состояния в твердое (происходит при той же температуре, что и плавление).

\vspace{-9pt}
\subsection*{Примечание:}
\vspace{-3pt}
Во время плавления или кристаллизации температура вещества не меняется, вся подводимая (отводимая) энергия идет на изменение агрегатного состояния.


% 1.16
\newpage
\section*{1.16 Изменение энергии в фазовых переходах}

\vspace{-9pt}
\subsection*{Энергия при фазовых переходах:}
\vspace{-3pt}
При фазовых переходах (плавление, кристаллизация, испарение, конденсация) происходит поглощение или выделение энергии (теплоты) без изменения температуры вещества.

\vspace{-9pt}
\subsection*{Константы:}
\vspace{-3pt}
\begin{enumerate} [itemsep=0pt, topsep=0pt, parsep=0pt]
    \item \textbf{Удельная теплота плавления ($\lambda$)} --- количество теплоты, необходимое для плавления 1 кг вещества при температуре плавления.
    \item \textbf{Удельная теплота парообразования ($L$)} --- количество теплоты, необходимое для превращения 1 кг жидкости в пар при температуре кипения.
\end{enumerate}


\end{document}