\documentclass[a4paper,12pt]{article}
\usepackage{fancyhdr}
\usepackage{graphicx} 
\usepackage{lipsum}
\usepackage{geometry}
\usepackage{mathtext}
\usepackage[T2A]{fontenc}
\usepackage[utf8]{inputenc}
\usepackage[english, russian]{babel}
\usepackage{titlesec}
\usepackage{xcolor}
\usepackage{graphicx}
\usepackage{caption}
\usepackage{subcaption}
\usepackage{wrapfig}
\usepackage{gensymb}
\usepackage{amsmath}
\usepackage{amssymb}
\usepackage{tikz}
\usepackage{circuitikz} 
\usepackage{physics}

\usepackage{enumitem}

\geometry{top=2cm, bottom=2cm, left=1cm, right=1cm}

\pagestyle{fancy}
\fancyhf{} 
\fancyfoot[C]{\rule[20pt]{\textwidth}{0.4pt}}
\fancyhead{} 
\fancyfoot[L]{Page \thepage} 
\fancyfoot[R]{\copyright\ Бот для подготовки к ЕГЭ по физике. 2025г} 

\titleformat{\section}[block]{\normalfont\Large\bfseries}{\thesection}{3em}{}
\setlist[itemize]{label=\scriptsize\textbullet}


\begin{document}

\begin{center}
\scalebox{2}{\textbf{Термодинамика}}
\end{center}

\vspace{-2.5em}

%%%%%%%%%%%%%%%%%
% \section*{}


% \vspace{-9pt}
% \subsection*{}
% \vspace{-3pt}


% \begin{enumerate} [itemsep=0pt, topsep=0pt, parsep=0pt]
%   \item $p$ – импульс тела;
%   \item $m$ – масса тела;
%   \item $v$ – скорость тела.
% \end{enumerate}


% [itemsep=0pt, topsep=0pt, parsep=0pt]
%%%%%%%%%%%%%%%%%


% 2.1
\section*{2.1 Внутренняя энергия}

\vspace{-9pt}
\subsection*{Определение:}
\vspace{-3pt}
\textbf{Внутренняя энергия ($U$) системы} --- это полная энергия всех микрочастиц (молекул, атомов, ионов), составляющих систему. Она включает в себя:
\begin{enumerate} [itemsep=0pt, topsep=0pt, parsep=0pt]
    \item \textbf{Кинетическую энергию} хаотического теплового движения этих частиц (поступательного, вращательного и колебательного).
    \item \textbf{Потенциальную энергию} взаимодействия между этими частицами (межмолекулярные силы, внутриатомные силы).
\end{enumerate}

\vspace{-9pt}
\subsection*{Свойства:}
\vspace{-3pt}
Внутренняя энергия $U$ является одной из ключевых термодинамических величин и обладает следующими свойствами:
\begin{enumerate} [itemsep=0pt, topsep=0pt, parsep=0pt]
    \item \textbf{Аддитивность}:
        $$ U_{\text{системы}} = \sum_{i} U_{i}, $$
        где $U_{i}$ — внутренняя энергия $i$-й подсистемы.
    \item \textbf{Функция состояния}:
        $$\Delta U = U_{2} - U_{1},$$
        где $\Delta U$ зависит только от начального и конечного состояний системы, а не от пути перехода.
    \item \textbf{Зависимость от температуры и объёма}:
        $$U = U(T, V),$$
        где $T$ — температура, $V$ — объём системы.
    \item \textbf{Связь с первым началом термодинамики}:
        $$\Delta U = Q - W,$$
        где $Q$ — количество теплоты, подведённое к системе, $W$ — работа, совершённая системой.
    \item \textbf{Для идеального газа}:
        $$U = \frac{3}{2} nRT \quad (\text{для одноатомного газа}),$$
        где $n$ — количество вещества, $R$ — универсальная газовая постоянная, $T$ — температура.
    \item \textbf{Для реальных систем}:
        Внутренняя энергия зависит от межмолекулярных взаимодействий и может быть выражена через уравнение состояния.
\end{enumerate}
\vspace{5pt}
% REMAKE
\par
Эти свойства позволяют описывать и анализировать термодинамические процессы в различных системах.
\par
Для многоатомного идеального газа формула несколько усложняется, но всё также внутренняя энергия пропорциональна температуре: $$U = \frac{i}{2}\nu RT,$$ где $i$ - число степеней свободы.
\par
\textbf{Изменение внутренней энергии} ($\Delta U$): $\Delta U = U_2 - U_1$.
\par
При нагревании газа внутренняя энергия увеличивается, при охлаждении - уменьшается.


\newpage
% ставишь только тогда, когда видно, что страница реально кончилась "ОК"

% 2.2
\section*{2.2 Тепловое равновесие}
\vspace{-9pt}
\subsection*{Определение:}
\vspace{-3pt}
\textbf{Тепловое равновесие} — это состояние системы, при котором все ее части имеют одинаковую температуру, и не происходит теплообмена между ними.

\vspace{-9pt}
\subsection*{Условие:}
\vspace{-3pt}
Система изолирована от внешней среды или находится с ней в равновесии.

\vspace{-9pt}
\subsection*{Процесс:}
\vspace{-3pt}
\begin{enumerate}[itemsep=0pt, topsep=0pt, parsep=3pt]
    \item Если тела с разными температурами приводят в контакт, то тепловая энергия передается от более нагретого к менее нагретому.
    \item Этот процесс продолжается до тех пор, пока температуры всех тел не сравняются.
\end{enumerate}

\vspace{-9pt}
\subsection*{Макроскопическое описание:}
\vspace{-3pt}
Система в тепловом равновесии характеризуется постоянными макроскопическими параметрами (давление, температура, объем), при этом на микроскопическом уровне частицы продолжают хаотическое тепловое движение.


% 2.3
\section*{2.3 Теплопередача}
\vspace{-9pt}
\subsection*{Определение:}
\vspace{-3pt}
\textbf{Теплопередача} — это процесс переноса тепловой энергии от более нагретых тел (или областей тела) к менее нагретым.

\vspace{-9pt}
\subsection*{Виды теплопередачи:}
\vspace{-3pt}
\begin{enumerate}[itemsep=0pt, topsep=0pt, parsep=3pt]
    \item \textbf{Теплопроводность} --- Передача энергии путем прямого взаимодействия (столкновения) между частицами, имеющими разную кинетическую энергию. Преобладает в твердых телах (особенно в металлах).
    \par
    \textbf{Закон Фурье}: описывает теплопроводность:
    \par
    $$ \frac{Q}{t} = -k \cdot A \cdot \frac{\Delta T}{\Delta x} $$
    где:
    \begin{itemize} 
        \item $\frac{Q}{t}$ — поток тепла (количество теплоты, проходящее через сечение в единицу времени).
        \item $k$ — коэффициент теплопроводности (характеристика материала).
        \item $A$ — площадь поперечного сечения, через которое проходит теплота.
        \item $\frac{\Delta T}{\Delta x}$ — температурный градиент (изменение температуры на единицу длины).
    \end{itemize}
    \vspace{-0.05em} % откуда это тут?
    \textbf{Примечание}: Теплопроводность происходит от более нагретой области к менее нагретой.

    \item \textbf{Конвекция} --- Передача энергии за счет перемещения потоков жидкости или газа, содержащих нагретые частицы. Характерно только для жидкостей и газов.
    \par
    \textbf{Бывает:}
    \vspace{-0.05 em}
    \begin{itemize}
        \item Естественной (за счет разности плотностей)
        \item вынужденной (с помощью насосов, вентиляторов).
    \end{itemize}
\newpage

    \item \textbf{Излучение} --- передача энергии в виде электромагнитных волн (инфракрасное излучение, видимый свет и др.). Может происходить в вакууме.
    \par
    \textbf{Закон Стефана-Больцмана}: описывает мощность излучения абсолютно черного тела:
    \vspace{-0.05em}
    $$ P = \sigma A T^4 $$
    где:
    \begin{itemize}
        \item $P$ — мощность излучения.
        \item $\sigma$ — постоянная Стефана-Больцмана.
        \item $A$ — площадь поверхности тела.
        \item $T$ — температура тела (в Кельвинах).
    \end{itemize}
    \vspace{-0.05em}
    \textbf{Примечание}: Все тела, имеющие температуру выше абсолютного нуля, излучают электромагнитную энергию.
\end{enumerate}

% 2.4
\section*{2.4 Количество теплоты. Удельная теплоемкость вещества}
\vspace{-9pt}
\subsection*{Количество теплоты ($Q$):}
\vspace{-3pt}
\textbf{Количество теплоты ($Q$)} — мера энергии, переданной или полученной системой в результате теплообмена (теплопередачи).

\vspace{-9pt}
\subsection*{Формулы:}
\vspace{-3pt}
\begin{enumerate}[itemsep=0pt, topsep=0pt, parsep=3pt]
    \item \textbf{Нагревание/охлаждение}:
    \vspace{-0.05em}
    $$ Q = cm \Delta T $$
    где:
    \begin{itemize}
        \item $c$ — удельная теплоемкость вещества (количество теплоты, необходимое для изменения температуры 1 кг вещества на 1 градус Цельсия (или Кельвин)).
        \item $m$ — масса тела.
        \item $\Delta T$ — изменение температуры $(T_2 - T_1)$.
    \end{itemize}
    \vspace{-0.05em}
    \par
    Примечание: Эта формула справедлива, если нет фазовых переходов.

    \item \textbf{Плавление/кристаллизация}:
    \vspace{-0.05em}
    $$ Q = \pm \lambda m $$
    где:
    \begin{itemize}
        \item $\lambda$ — удельная теплота плавления (количество теплоты, необходимое для плавления 1 кг вещества при температуре плавления).
        \item $m$ — масса вещества.
        \item Знак "$+$" — при плавлении (теплота поглощается), знак "$–$" — при кристаллизации (теплота выделяется).
    \end{itemize}

    \item \textbf{Испарение/конденсация}:
    \vspace{-0.05em}
    $$ Q = \pm Lm $$
    где:
    \begin{itemize}
        \item $L$ — удельная теплота парообразования (количество теплоты, необходимое для превращения 1 кг жидкости в пар при температуре кипения).
        \item $m$ — масса вещества.
        \item Знак "$+$" — при парообразовании (теплота поглощается), знак "$–$" — при конденсации (теплота выделяется).
    \end{itemize}
\end{enumerate}
\vspace{-9pt}
\newpage

\subsection*{Удельная теплоемкость ($c$):} 
\textbf{Удельная теплоемкость} --- Характеристика вещества, показывающая, насколько сильно изменяется его температура при передаче ему определенного количества теплоты.
\vspace{-3pt}
\begin{itemize}
    \item Зависит от агрегатного состояния вещества и его температуры.
\end{itemize}

% 2.5
\section*{2.5 Работа в термодинамике}

\vspace{-9pt}
\subsection*{Определение:}
\vspace{-3pt}
\textbf{Работа ($A$)} в термодинамике — это мера энергии, переданной системе (или полученной от нее) в результате изменения объема системы под действием внешнего давления.

\vspace{-9pt}
\subsection*{Формула (для квазистатического процесса):}
\vspace{-3pt}
\begin{enumerate}[itemsep=0pt, topsep=0pt, parsep=3pt]
    \item Работа:
    \vspace{-0.05em}
    $$ A = \int p \, dV $$
    где:
    \begin{itemize}
        \item $p$ — давление газа, которое может меняться в процессе.
        \item $dV$ — бесконечно малое изменение объема.
        \item Интеграл берется по пути изменения объема.
    \end{itemize}
\end{enumerate}

\vspace{-9pt}
\subsection*{Формула для изобарного процесса:}
\vspace{-3pt}

$$ A = p \Delta V = p(V_2 - V_1) \quad \text{(давление постоянно).} $$.


\vspace{-9pt}
\subsection*{Геометрический смысл:}
\vspace{-3pt}
Работа равна площади под кривой на PV-диаграмме.

\vspace{-9pt}
\subsection*{Знаки работы:}
\vspace{-3pt}
\begin{itemize}
    \item $A > 0$: Если система совершает работу (например, газ расширяется, выталкивая поршень).
    \item $A < 0$: Если над системой совершается работа (например, газ сжимается внешними силами).
\end{itemize}


% 2.6
\section*{2.6 Уравнение теплового баланса}
\vspace{-9pt}
\subsection*{Формулировка:}
\vspace{-3pt}
В изолированной системе (системе, не обменивающейся энергией с окружающей средой) суммарное количество теплоты, отданное одними телами, равно суммарному количеству теплоты, полученному другими телами:
\vspace{-0.05em}
$$ \sum{Q_{\text{отданное}}} = \sum{Q_{\text{полученное}}} $$

\vspace{-9pt}
\subsection*{Применение:}
\vspace{-3pt}
Используется для расчета температур при теплообмене в изолированных системах (например, в калориметре).
\newpage
\vspace{-9pt}
\subsection*{Условие справедливости:}
\vspace{-3pt}
Система должна быть \textit{замкнутой} и \textit{изолированной}, т.е. не должна обмениваться теплом и веществом с окружающей средой.

\vspace{-9pt}
\subsection*{Пример:}
\vspace{-3pt}
Смешивание воды разной температуры в калориметре.


% 2.7

\section*{2.7 Первый закон термодинамики}
\vspace{-9pt}
\subsection*{Формулировка:}
\vspace{-3pt}
Изменение внутренней энергии системы ($\Delta U$) равно сумме количества теплоты ($Q$), переданного системе, и работы ($A$), совершенной над системой внешними силами:
\begin{enumerate}[itemsep=0pt, topsep=0pt, parsep=3pt]
    \item $\Delta U = Q + A$ (работа над системой, от внешних сил).
    \item $Q = \Delta U + A$ (работа, совершенная системой против внешних сил).
\end{enumerate}

\vspace{-9pt}
\subsection*{Связь с законом сохранения энергии:}
\vspace{-3pt}
Первый закон термодинамики является математической формулировкой закона сохранения энергии для термодинамических процессов.

\vspace{-9pt}
\subsection*{Знаки величин:}
\vspace{-3pt}
\begin{itemize}
    \item $Q > 0$: Теплота подводится к системе.
    \item $Q < 0$: Теплота отводится от системы.
    \item $A > 0$: Работа совершается системой над внешними телами.
    \item $A < 0$: Работа совершается над системой внешними силами.
\end{itemize}

\vspace{-9pt}
\subsection*{Применение:}
\vspace{-3pt}
Позволяет анализировать энергетические балансы при различных процессах: изотермическом, адиабатном, изобарном, изохорном.



% 2.8
\section*{2.8 Второй закон термодинамики}
\vspace{-9pt}
\subsection*{Формулировки:}
\vspace{-3pt}
\begin{enumerate}[itemsep=0pt, topsep=0pt, parsep=3pt]
    \item \textbf{Формулировка Клаузиуса:} Невозможен процесс, при котором теплота самопроизвольно переходила бы от холодного тела к более нагретому.
    \item \textbf{Формулировка Томсона (Кельвина):} Невозможен периодический процесс, единственным результатом которого было бы превращение теплоты, полученной от теплового резервуара, полностью в работу (не существует "вечного двигателя второго рода").
    \item \textbf{Формулировка с помощью энтропии:} В изолированной системе энтропия не уменьшается, а в идеале растет (при необратимых процессах).
\end{enumerate}

\vspace{-9pt}
\subsection*{Суть:}
\vspace{-3pt}
Второй закон указывает на то, что все реальные процессы в природе необратимы и сопровождаются увеличением энтропии (меры беспорядка).
\newpage
\vspace{-9pt}
\subsection*{Следствия:}
\vspace{-3pt}
\begin{itemize}
    \item Существует направление развития процессов (теплота сама не переходит от холодного к горячему).
    \item КПД тепловых двигателей не может быть 100\%.
    \item Природа стремится к беспорядку.
\end{itemize}


% 2.9
\section*{2.9 КПД тепловой машины}
\vspace{-9pt}
\subsection*{Определение:}
\vspace{-3pt}
\textbf{КПД (коэффициент полезного действия)} тепловой машины — это отношение полезной работы ($A$), совершенной машиной, к количеству теплоты ($Q_{\text{нагр}}$), полученному от нагревателя.

\vspace{-9pt}
\subsection*{Формула:}
\vspace{-3pt}
\textbf{КПД}:
    $$ \eta = \frac{Q_{\text{нагр}} - Q_{\text{холод}}}{Q_{\text{нагр}}} \quad \text{или} \quad \eta = 1 - \frac{Q_{\text{холод}}}{Q_{\text{нагр}}} $$
    где:
\begin{itemize}
    \item $Q_{\text{нагр}}$ — количество теплоты, полученное от нагревателя.
    \item $Q_{\text{холод}}$ — количество теплоты, отданное холодильнику.
\end{itemize}
\vspace{-0.05em}
\textbf{Примечание}: КПД выражается либо в долях единицы, либо в процентах.


\vspace{-9pt}
\subsection*{Ограничения:}
\vspace{-3pt}
КПД не может быть 100\% (второй закон термодинамики).

\vspace{-9pt}
\subsection*{Идеальная тепловая машина (Цикл Карно):}
\vspace{-3pt}
Наибольший КПД, теоретически возможный для тепловой машины, работающей при заданных температурах нагревателя ($T_{\text{нагр}}$) и холодильника ($T_{\text{холод}}$):
\vspace{-0.05em}
$$ \eta_{\text{Карно}} = 1 - \frac{T_{\text{холод}}}{T_{\text{нагр}}} $$


% 2.10
\section*{2.10 Принципы действия тепловых машин}

\vspace{-9pt}
\subsection*{Принцип:}
\vspace{-3pt}
Тепловые машины преобразуют тепловую энергию в механическую работу за счет циклического изменения состояния рабочего тела (газа или пара).

\vspace{-9pt}
\subsection*{Основные элементы:}
\vspace{-3pt}
\begin{itemize}
    \item \textbf{Нагреватель:} Источник теплоты с высокой температурой (например, сгорание топлива).
    \item \textbf{Рабочее тело:} Вещество (обычно газ или пар), совершающее работу.
    \item \textbf{Холодильник:} Тело с более низкой температурой, которому рабочее тело отдает избыточную теплоту.
\end{itemize}
\newpage
\vspace{-9pt}
\subsection*{Цикл:}
\vspace{-3pt}
Тепловой двигатель работает по циклу, который состоит из нескольких термодинамических процессов:
\begin{enumerate}[itemsep=0pt, topsep=0pt, parsep=3pt]
    \item Рабочее тело получает тепло от нагревателя.
    \item Рабочее тело совершает работу, расширяясь.
    \item Рабочее тело отдает избыточную теплоту холодильнику.
    \item Рабочее тело возвращается в начальное состояние.
\end{enumerate}

\vspace{-9pt}
\subsection*{Примеры:}
\vspace{-3pt}
\begin{itemize}
    \item Двигатель внутреннего сгорания (ДВС).
    \item Паровая турбина.
    \item Холодильная машина (работает в обратном цикле, т.е. совершает работу для переноса тепла от холодного тела к более нагретому).
\end{itemize}


% здесь хочу сделать что то в таком формате
% https://myslide.ru/documents_4/186a0fd46c45e9c3beaf3522148824e2/img5.jpg
% Цикл: Тепловой двигатель работает по циклу, который состоит из нескольких термодинамических процессов:
%     1.  Рабочее тело получает тепло от нагревателя.
%     2.  Рабочее тело совершает работу, расширяясь.
%     3.  Рабочее тело отдает избыточную теплоту холодильнику.
%     4.  Рабочее тело возвращается в начальное состояние.


\end{document}