\documentclass[a4paper,12pt]{article}
\usepackage{fancyhdr}
\usepackage{graphicx} 
\usepackage{lipsum}
\usepackage{geometry}
\usepackage{mathtext}
\usepackage[T2A]{fontenc}
\usepackage[utf8]{inputenc}
\usepackage[english, russian]{babel}
\usepackage{titlesec}
\usepackage{xcolor}
\usepackage{graphicx}
\usepackage{caption}
\usepackage{subcaption}
\usepackage{wrapfig}
\usepackage{gensymb}
\usepackage{amsmath}
\usepackage{amssymb}
\usepackage{tikz}
\usepackage{circuitikz} 
\usepackage{physics}

\usepackage{enumitem}

\geometry{top=2cm, bottom=2cm, left=1cm, right=1cm}

\pagestyle{fancy}
\fancyhf{} 
\fancyfoot[C]{\rule[20pt]{\textwidth}{0.4pt}}
\fancyhead{} 
\fancyfoot[L]{Page \thepage} 
\fancyfoot[R]{\copyright\ Бот для подготовки к ЕГЭ по физике. 2025г} 

\titleformat{\section}[block]{\normalfont\Large\bfseries}{\thesection}{3em}{}
\setlist[itemize]{label=\scriptsize\textbullet}


\begin{document}

\begin{center}
\scalebox{2}{\textbf{Электромагнитная индукция}}
\end{center}

\vspace{-2.5em}

%%%%%%%%%%%%%%%%%
% \section*{}


% \vspace{-9pt}
% \subsection*{}
% \vspace{-3pt}


% \begin{enumerate} [itemsep=0pt, topsep=0pt, parsep=0pt]
%   \item $p$ – импульс тела;
%   \item $m$ – масса тела;
%   \item $v$ – скорость тела.
% \end{enumerate}


% [itemsep=0pt, topsep=0pt, parsep=0pt]
%%%%%%%%%%%%%%%%%

\section*{4.1 Явление электромагнитной индукции}
\vspace{-9pt}
\subsection*{Электромагнитная индукция:}
\vspace{-3pt}
Явление возникновения электрического тока в проводящем контуре при изменении магнитного потока через этот контур.

\vspace{-9pt}
\subsection*{Условие:}
\vspace{-3pt}
Изменение магнитного потока может быть вызвано движением магнита относительно контура, изменением силы тока в проводнике, изменением площади контура в магнитном поле и др.

\section*{4.2 Магнитный поток}
\vspace{-9pt}
\subsection*{Магнитный поток ($\Phi$):}
\vspace{-3pt}
Физическая величина, характеризующая количество линий магнитного поля, пронизывающих данную поверхность.

\vspace{-9pt}
\subsection*{Формула:}
\vspace{-3pt}
\begin{enumerate}[itemsep=0pt, topsep=0pt, parsep=3pt]
  \item 
  \vspace{-0.05em}
  $$ \Phi = B \cdot S \cdot \cos(\alpha) $$
  где:
  \begin{itemize}
    \item $\Phi$ – магнитный поток,
    \item $B$ – индукция магнитного поля,
    \item $S$ – площадь поверхности,
    \item $\alpha$ – угол между нормалью к поверхности и вектором магнитной индукции.
  \end{itemize}
\end{enumerate}

\vspace{-9pt}
\subsection*{Единица измерения:}
\vspace{-3pt}
Вебер (Вб).

\section*{4.3 Закон электромагнитной индукции Фарадея}
\vspace{-9pt}
\subsection*{Формулировка:}
\vspace{-3pt}
ЭДС индукции в замкнутом контуре равна скорости изменения магнитного потока через этот контур, взятой с обратным знаком.

\vspace{-9pt}
\subsection*{Формула:}
\vspace{-3pt}
\begin{enumerate}[itemsep=0pt, topsep=0pt, parsep=3pt]
  \item 
  \vspace{-0.05em}
  $$ \epsilon_{инд} = -\frac{\Delta \Phi}{\Delta t} $$
  где:
  \begin{itemize}
    \item $\epsilon_{инд}$ – ЭДС индукции,
    \item $\Delta \Phi$ – изменение магнитного потока,
    \item $\Delta t$ – время, за которое произошло изменение магнитного потока.
  \end{itemize}
\end{enumerate}

\section*{4.4 Правило Ленца}
\vspace{-9pt}
\subsection*{Формулировка:}
\vspace{-3pt}
Индукционный ток имеет такое направление, что его магнитное поле противодействует изменению магнитного потока, вызвавшего этот ток.

\vspace{-9pt}
\subsection*{Значение:}
\vspace{-3pt}
Правило Ленца определяет направление индукционного тока.

\vspace{-9pt}
\subsection*{Влияние:}
\vspace{-3pt}
Индукционный ток стремится сохранить магнитный поток через контур неизменным.

\section*{4.5 Самоиндукция}
\vspace{-9pt}
\subsection*{Самоиндукция:}
\vspace{-3pt}
Явление возникновения ЭДС (электродвижущей силы) в проводнике при изменении силы тока в этом же проводнике.

\vspace{-9pt}
\subsection*{Суть явления:}
\vspace{-3pt}
Когда ток в проводнике (например, в катушке индуктивности) меняется, создаваемое им магнитное поле тоже меняется. Это изменение магнитного поля, в свою очередь, индуцирует ЭДС в самом проводнике, который и создал это поле. Эта индуцированная ЭДС называется ЭДС самоиндукции.

\vspace{-9pt}
\subsection*{Направление ЭДС самоиндукции:}
\vspace{-3pt}
Согласно правилу Ленца, ЭДС самоиндукции всегда направлена так, чтобы препятствовать изменению тока, вызвавшему её.
\begin{itemize}
  \item Если ток в цепи возрастает, ЭДС самоиндукции направлена против тока, стремясь уменьшить его нарастание.
  \item Если ток в цепи убывает, ЭДС самоиндукции направлена в ту же сторону, что и ток, стремясь замедлить его убывание.
\end{itemize}

\vspace{-9pt}
\subsection*{Индуктивность ($L$):}
\vspace{-3pt}
Это физическая величина, характеризующая способность проводника (катушки) создавать ЭДС самоиндукции при изменении тока в нем. Индуктивность измеряется в Генри (Гн).
\begin{itemize}
  \item Чем больше индуктивность, тем больше ЭДС самоиндукции возникает при заданном изменении тока.
  \item Индуктивность зависит от геометрических размеров и формы проводника (катушки), а также от магнитной проницаемости среды, в которой он находится.
\end{itemize}

\vspace{-9pt}
\subsection*{Формула для ЭДС самоиндукции:}
\vspace{-3pt}
\begin{enumerate}[itemsep=0pt, topsep=0pt, parsep=3pt]
  \item 
  \vspace{-0.05em}
  $$ \epsilon = -L \cdot \frac{\Delta I}{\Delta t} $$
  где:
  \begin{itemize}
    \item $\epsilon$ - ЭДС самоиндукции (В),
    \item $L$ - индуктивность (Гн),
    \item $\Delta I$ - изменение силы тока (А),
    \item $\Delta t$ - изменение времени (с).
  \end{itemize}
\end{enumerate}

\vspace{-9pt}
\subsection*{Энергия магнитного поля катушки индуктивности:}
\vspace{-3pt}
Катушка с током накапливает энергию в своем магнитном поле. Эта энергия может быть вычислена по формуле:
\begin{enumerate}[itemsep=0pt, topsep=0pt, parsep=3pt]
  \item 
  \vspace{-0.05em}
  $$ W = \frac{L \cdot I^2}{2} $$
  где:
  \begin{itemize}
    \item $W$ - энергия магнитного поля (Дж),
    \item $L$ - индуктивность (Гн),
    \item $I$ - сила тока (А).
  \end{itemize}
\end{enumerate}

\vspace{-9pt}
\subsection*{Применение самоиндукции:}
\vspace{-3pt}
\begin{itemize}
  \item Катушки индуктивности в электрических цепях: Используются для сглаживания пульсаций тока, создания колебательных контуров, фильтрации частот и других целей.
  \item Трансформаторы: Работают на основе явления взаимной индукции, которое тесно связано с самоиндукцией.
  \item Системы зажигания в автомобилях: Для создания высоковольтного импульса для поджига топливно-воздушной смеси.
  \item Индукционные нагреватели: Для бесконтактного нагрева металлических деталей.
\end{itemize}

\newpage

\section*{4.6 Индуктивность}
\vspace{-9pt}
\subsection*{Индуктивность ($L$):}
\vspace{-3pt}
Физическая величина, характеризующая способность контура создавать магнитное поле при прохождении через него электрического тока.

\vspace{-9pt}
\subsection*{Формула:}
\vspace{-3pt}
\begin{enumerate}[itemsep=0pt, topsep=0pt, parsep=3pt]
  \item 
  \vspace{-0.05em}
  $$ L = \frac{\Phi}{I} $$
  где:
  \begin{itemize}
    \item $L$ – индуктивность,
    \item $\Phi$ – магнитный поток,
    \item $I$ – сила тока.
  \end{itemize}
\end{enumerate}

\vspace{-9pt}
\subsection*{Единица измерения:}
\vspace{-3pt}
Генри (Гн). 1 Гн = 1 Вб/А.

\vspace{-9pt}
\subsection*{ЭДС самоиндукции:}
\vspace{-3pt}
\vspace{-0.05em}
$$ \epsilon_{самоинд} = -L \cdot \frac{\Delta I}{\Delta t} $$

\section*{4.7 Энергия магнитного поля}
\vspace{-9pt}
\subsection*{Энергия магнитного поля ($W_м$):}
\vspace{-3pt}
Энергия, запасенная в магнитном поле катушки с током.

\vspace{-9pt}
\subsection*{Формула:}
\vspace{-3pt}
\begin{enumerate}[itemsep=0pt, topsep=0pt, parsep=3pt]
  \item 
  \vspace{-0.05em}
  $$ W_м = \frac{L \cdot I^2}{2} $$
  где:
  \begin{itemize}
    \item $W_м$ – энергия магнитного поля,
    \item $L$ – индуктивность,
    \item $I$ – сила тока.
  \end{itemize}
\end{enumerate}



\end{document}