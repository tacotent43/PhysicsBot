\documentclass[a4paper,12pt]{article}
\usepackage{fancyhdr}
\usepackage{graphicx} 
\usepackage{lipsum}
\usepackage{geometry}
\usepackage{mathtext}
\usepackage[T2A]{fontenc}
\usepackage[utf8]{inputenc}
\usepackage[english, russian]{babel}
\usepackage{titlesec}
\usepackage{xcolor}
\usepackage{graphicx}
\usepackage{caption}
\usepackage{subcaption}
\usepackage{wrapfig}
\usepackage{gensymb}
\usepackage{amsmath}
\usepackage{amssymb}
\usepackage{tikz}
\usepackage{circuitikz} 
\usepackage{physics}

\usepackage{enumitem}

\geometry{top=2cm, bottom=2cm, left=1cm, right=1cm}

\pagestyle{fancy}
\fancyhf{} 
\fancyfoot[C]{\rule[20pt]{\textwidth}{0.4pt}}
\fancyhead{} 
\fancyfoot[L]{Page \thepage} 
\fancyfoot[R]{\copyright\ Бот для подготовки к ЕГЭ по физике. 2025г} 

\titleformat{\section}[block]{\normalfont\Large\bfseries}{\thesection}{3em}{}
\setlist[itemize]{label=\scriptsize\textbullet}


\begin{document}

\begin{center}
\scalebox{2}{\textbf{Электромагнитные колебания и волны}}
\end{center}

\vspace{-2.5em}

%%%%%%%%%%%%%%%%%
% \section*{}


% \vspace{-9pt}
% \subsection*{}
% \vspace{-3pt}


% \begin{enumerate} [itemsep=0pt, topsep=0pt, parsep=0pt]
%   \item $p$ – импульс тела;
%   \item $m$ – масса тела;
%   \item $v$ – скорость тела.
% \end{enumerate}


% [itemsep=0pt, topsep=0pt, parsep=0pt]
%%%%%%%%%%%%%%%%%

\section*{3.6.1 Прямолинейное распространение света}
\vspace{-9pt}
\subsection*{Закон:}
\vspace{-3pt}
В однородной среде свет распространяется по прямой.

\vspace{-9pt}
\subsection*{Примеры:}
\vspace{-3pt}
\begin{itemize}
    \item Образование тени.
    \item Солнечные и лунные затмения.
\end{itemize}


\section*{3.6.2 Закон отражения света}
\vspace{-9pt}
\subsection*{Формулировка:}
\vspace{-3pt}
Угол падения равен углу отражения.

\vspace{-9pt}
\subsection*{Дополнение:}
\vspace{-3pt}
Падающий луч, отраженный луч и нормаль к поверхности лежат в одной плоскости.


\section*{3.6.3 Построение изображений в плоском зеркале}
\vspace{-9pt}
\subsection*{Изображение:}
\vspace{-3pt}
Мнимое, прямое, равное по размеру предмету, симметричное относительно плоскости зеркала.

\vspace{-9pt}
\subsection*{Правило:}
\vspace{-3pt}
Изображение точки находится на том же расстоянии от зеркала, что и предмет, но с противоположной стороны.


\section*{3.6.4 Закон преломления света}
\vspace{-9pt}
\subsection*{Формулировка:}
\vspace{-3pt}
Отношение синуса угла падения к синусу угла преломления есть величина постоянная для двух данных сред:
\vspace{-0.05em}
$$ \frac{\sin(\alpha)}{\sin(\beta)} = \frac{n_2}{n_1} = n_{21} $$
где:
\begin{itemize}
    \item $n_{21}$ — относительный показатель преломления среды 2 относительно среды 1.
    \item $n$ — абсолютный показатель преломления среды.
\end{itemize}

\vspace{-9pt}
\subsection*{Дополнение:}
\vspace{-3pt}
Падающий луч, преломленный луч и нормаль к поверхности лежат в одной плоскости.


\section*{3.6.5 Полное внутреннее отражение}
\vspace{-9pt}
\subsection*{Условие:}
\vspace{-3pt}
Свет падает из среды с большим показателем преломления в среду с меньшим показателем преломления под углом, большим критического угла.

\vspace{-9pt}
\subsection*{Критический угол:}
\vspace{-3pt}
\subsubsection*{Формула:}
\vspace{-0.05em}
$$ \sin(\alpha_{\text{кр}}) = \frac{n_2}{n_1} $$

\vspace{-9pt}
\subsection*{Явление:}
\vspace{-3pt}
Свет полностью отражается от границы раздела двух сред, не выходя во вторую среду.

\vspace{-9pt}
\subsection*{Примеры:}
\vspace{-3pt}
\begin{itemize}
    \item Волоконная оптика.
    \item Алмазы.
\end{itemize}


\section*{3.6.6 Линзы. Оптическая сила линзы}
\vspace{-9pt}
\subsection*{Линза:}
\vspace{-3pt}
Прозрачное тело, ограниченное двумя сферическими поверхностями.

\vspace{-9pt}
\subsection*{Собирающая линза:}
\vspace{-3pt}
Линза, у которой лучи света, проходящие через нее, сходятся в фокусе.
\begin{itemize}
    \item Фокусное расстояние ($F$) > 0.
\end{itemize}

\vspace{-9pt}
\subsection*{Рассеивающая линза:}
\vspace{-3pt}
Линза, у которой лучи света, проходящие через нее, расходятся.
\begin{itemize}
    \item Фокусное расстояние ($F$) < 0.
\end{itemize}

\vspace{-9pt}
\subsection*{Оптическая сила линзы ($D$):}
\vspace{-3pt}
Величина, обратная фокусному расстоянию линзы.
\subsubsection*{Формула:}
\vspace{-0.05em}
$$ D = \frac{1}{F} $$
\subsubsection*{Единица измерения:} диоптрия (дптр). 1 дптр = 1 $м^{-1}$.

\section*{3.6.7 Формула тонкой линзы}
\vspace{-9pt}
\subsubsection*{Формула:}
\vspace{-0.05em}
$$ \frac{1}{d} + \frac{1}{f} = \frac{1}{F} $$
где:
\begin{itemize}
    \item $d$ — расстояние от предмета до линзы.
    \item $f$ — расстояние от линзы до изображения.
    \item $F$ — фокусное расстояние линзы.
\end{itemize}

\vspace{-9pt}
\subsection*{Линейное увеличение линзы:}
\vspace{-3pt}
\subsubsection*{Формула:}
\vspace{-0.05em}
$$ Г = \frac{f}{d} $$

\section*{3.6.8 Построение изображений в линзах}
\vspace{-9pt}
\subsection*{Изображения:}
\vspace{-3pt}
Могут быть действительными (пересечение лучей) или мнимыми (пересечение продолжений лучей), прямыми или перевернутыми, увеличенными или уменьшенными.

\vspace{-9pt}
\subsection*{Правила:}
\vspace{-3pt}
\begin{enumerate}[itemsep=0pt, topsep=0pt, parsep=3pt]
    \item Луч, идущий параллельно главной оптической оси, после преломления проходит через фокус.
    \item Луч, проходящий через оптический центр линзы, не меняет своего направления.
    \item Луч, проходящий через фокус, после преломления идет параллельно главной оптической оси.
\end{enumerate}


\section*{3.6.9 Оптические приборы. Глаз как оптическая система}
\vspace{-9pt}
\subsection*{Оптические приборы:}
\vspace{-3pt}
Устройства, использующие линзы для получения изображения (лупа, микроскоп, телескоп, фотоаппарат).

\vspace{-9pt}
\subsection*{Глаз:}
\vspace{-3pt}
Сложная оптическая система, фокусирующая изображение на сетчатке.
\begin{itemize}
    \item Хрусталик глаза является линзой.
    \item Дефекты зрения (близорукость, дальнозоркость) корректируются линзами.
\end{itemize}

\newpage
\section*{3.6.10 Интерференция света}
\vspace{-9pt}
\subsection*{Интерференция света:}
\vspace{-3pt}
Сложение световых волн, при котором наблюдается усиление или ослабление результирующей амплитуды.

\vspace{-9pt}
\subsection*{Условие для максимума:}
\vspace{-3pt}
Разность хода волн должна быть равна целому числу длин волн:
\vspace{-0.05em}
$$ \Delta d = m\lambda $$

\vspace{-9pt}
\subsection*{Условие для минимума:}
\vspace{-3pt}
Разность хода волн должна быть равна полуцелому числу длин волн:
\vspace{-0.05em}
$$ \Delta d = \left(m + \frac{1}{2}\right)\lambda $$


\section*{3.6.11 Дифракция света}
\vspace{-9pt}
\subsection*{Дифракция света:}
\vspace{-3pt}
Огибание световыми волнами препятствий.

\vspace{-9pt}
\subsection*{Наблюдение:}
\vspace{-3pt}
Проявляется при прохождении света через узкие щели или отверстия, а также при обходе краев препятствий.


\section*{3.6.12 Дифракционная решетка}
\vspace{-9pt}
\subsection*{Дифракционная решетка:}
\vspace{-3pt}
Оптический прибор, представляющий собой совокупность большого числа параллельных равноотстоящих щелей.

\vspace{-9pt}
\subsubsection*{Формула:}
\vspace{-0.05em}
$$ d \cdot \sin(\alpha) = m\lambda $$
где:
\begin{itemize}
    \item $d$ — период (постоянная) решетки.
    \item $\alpha$ — угол дифракции.
    \item $m$ — порядок дифракционного максимума.
    \item $\lambda$ — длина волны света.
\end{itemize}
\newpage
\section*{3.6.13 Дисперсия света}
\vspace{-9pt}
\subsection*{Дисперсия света:}
\vspace{-3pt}
Зависимость показателя преломления вещества от частоты света.

\vspace{-9pt}
\subsection*{Призма:}
\vspace{-3pt}
Разложение белого света на спектр при прохождении через призму (разные длины волн отклоняются на разные углы).

\end{document}