\documentclass[a4paper,12pt]{article}
\usepackage{fancyhdr}
\usepackage{graphicx} 
\usepackage{lipsum}
\usepackage{geometry}
\usepackage{mathtext}
\usepackage[T2A]{fontenc}
\usepackage[utf8]{inputenc}
\usepackage[english, russian]{babel}
\usepackage{titlesec}
\usepackage{xcolor}
\usepackage{graphicx}
\usepackage{caption}
\usepackage{subcaption}
\usepackage{wrapfig}
\usepackage{gensymb}
\usepackage{amsmath}
\usepackage{amssymb}
\usepackage{tikz}
\usepackage{circuitikz} 
\usepackage{physics}

\usepackage{enumitem}

\geometry{top=2cm, bottom=2cm, left=1cm, right=1cm}

\pagestyle{fancy}
\fancyhf{} 
\fancyfoot[C]{\rule[20pt]{\textwidth}{0.4pt}}
\fancyhead{} 
\fancyfoot[L]{Page \thepage} 
\fancyfoot[R]{\copyright\ Бот для подготовки к ЕГЭ по физике. 2025г} 

\titleformat{\section}[block]{\normalfont\Large\bfseries}{\thesection}{3em}{}
\setlist[itemize]{label=\scriptsize\textbullet}


\begin{document}

\begin{center}
\scalebox{2}{\textbf{Электрическое поле}}
\end{center}

\vspace{-2.5em}

%%%%%%%%%%%%%%%%%
% \section*{}


% \vspace{-9pt}
% \subsection*{}
% \vspace{-3pt}


% \begin{enumerate} [itemsep=0pt, topsep=0pt, parsep=0pt]
%   \item $p$ – импульс тела;
%   \item $m$ – масса тела;
%   \item $v$ – скорость тела.
% \end{enumerate}


% [itemsep=0pt, topsep=0pt, parsep=0pt]
%%%%%%%%%%%%%%%%%



\section*{3.3.1 Взаимодействие магнитов}
\vspace{-9pt}
\subsection*{Магниты:}
\vspace{-3pt}
Тела, обладающие способностью создавать магнитное поле и взаимодействовать с другими магнитами.

\vspace{-9pt}
\subsection*{Магнитные полюса:}
\vspace{-3pt}
Северный ($N$) и южный ($S$).
\begin{itemize}
    \item Одноименные полюса отталкиваются, разноименные притягиваются.
\end{itemize}

\vspace{-9pt}
\subsection*{Магнитное поле:}
\vspace{-3pt}
Особая форма материи, существующая вокруг движущихся электрических зарядов и магнитов, и проявляющаяся в воздействии на другие движущиеся заряды и магниты.

\vspace{-9pt}
\subsection*{Магнитные линии:}
\vspace{-3pt}
Линии, вдоль которых располагается вектор магнитной индукции.

\section*{3.3.2 Магнитное поле проводника с током}
\vspace{-9pt}
\subsection*{Основные моменты:}
\vspace{-3pt}
\begin{itemize}
    \item Ток создает магнитное поле: Это фундаментальный принцип электромагнетизма. Любое движение электрических зарядов (т.е. электрический ток) создает вокруг себя магнитное поле.
    \item Форма магнитного поля: Форма магнитного поля зависит от формы проводника с током.
\end{itemize}

\vspace{-9pt}
\subsection*{Форма магнитного поля:}
\vspace{-3pt}
\begin{enumerate}[itemsep=0pt, topsep=0pt, parsep=3pt]
    \item Прямой проводник с током:
    \vspace{-0.05em}
    \begin{itemize}
        \item Магнитное поле вокруг прямого проводника с током имеет форму концентрических окружностей, расположенных в плоскости, перпендикулярной проводнику.
        \item Проводник находится в центре этих окружностей.
        \item Направление линий магнитного поля определяется правилом буравчика (правилом правой руки): Если направить большой палец правой руки по направлению тока в проводнике, то остальные пальцы, обхватывающие проводник, укажут направление линий магнитного поля.
    \end{itemize}
    \item Круговой виток с током:
    \vspace{-0.05em}
    \begin{itemize}
        \item Магнитное поле кругового витка с током представляет собой комбинацию полей, создаваемых каждой малой частью витка.
        \item В центре витка магнитное поле направлено перпендикулярно плоскости витка.
        \item Вдали от витка магнитное поле становится похожим на поле магнитного диполя (как у небольшого магнита).
        \item Направление магнитного поля определяется тем же правилом буравчика.
    \end{itemize}
    \newpage
    
    \item Соленоид (катушка) с током:
    \vspace{-0.05em}
    \begin{itemize}
        \item Соленоид - это катушка, намотанная из проволоки в форме спирали.
        \item Магнитное поле внутри соленоида практически однородно и направлено вдоль оси соленоида.
        \item Вне соленоида магнитное поле слабее и похоже на поле полосового магнита.
        \item Соленоид с током можно рассматривать как электромагнит.
        \item Направление магнитного поля определяется правилом буравчика: Если обхватить соленоид правой рукой так, чтобы пальцы указывали направление тока в витках, то большой палец покажет направление магнитного поля внутри соленоида.
    \end{itemize}
\end{enumerate}

\vspace{-9pt}
\subsection*{Магнитная индукция ($B$):}
\vspace{-3pt}
\begin{itemize}
    \item Это векторная величина, характеризующая магнитное поле в данной точке пространства.
    \item Единица измерения магнитной индукции - Тесла (Тл).
    \item Модуль магнитной индукции определяется силой, действующей на движущийся заряд в магнитном поле (сила Лоренца).
\end{itemize}

\vspace{-9pt}
\subsection*{Расчет магнитной индукции:}
\vspace{-3pt}
\begin{enumerate}[itemsep=0pt, topsep=0pt, parsep=3pt]
    \item Закон Био-Савара-Лапласа: Это основной закон, позволяющий рассчитать магнитную индукцию, создаваемую элементом тока. Он применяется для проводников любой формы.
    \item Для прямого проводника:
    \vspace{-0.05em}
    $$ B = \frac{\mu_0 \cdot I}{2\pi r} $$
    где:
    \begin{itemize}
        \item $B$ - магнитная индукция
        \item $\mu_0$ - магнитная постоянная ($4\pi \cdot 10^{-7}$ Гн/м)
        \item $I$ - сила тока
        \item $r$ - расстояние от проводника до точки, в которой измеряется магнитная индукция.
    \end{itemize}
    \item Для центра кругового витка:
    \vspace{-0.05em}
    $$ B = \frac{\mu_0 \cdot I}{2R} $$
    где $R$ - радиус витка.
    \item Для соленоида (внутри):
    \vspace{-0.05em}
    $$ B = \mu_0 \cdot n \cdot I $$
    где $n$ - количество витков на единицу длины соленоида.
\end{enumerate}

\vspace{-9pt}
\subsection*{Взаимодействие магнитных полей:}
\vspace{-3pt}
\begin{itemize}
    \item Проводники с током взаимодействуют друг с другом посредством своих магнитных полей.
    \item Если токи в двух параллельных проводниках направлены в одну сторону, то проводники притягиваются.
    \item Если токи направлены в противоположные стороны, то проводники отталкиваются.
\end{itemize}

\vspace{-9pt}
\subsection*{Практическое применение:}
\vspace{-3pt}
\begin{itemize}
    \item Электродвигатели: Принцип работы основан на взаимодействии магнитных полей, создаваемых током в обмотках.
    \item Генераторы: Преобразуют механическую энергию в электрическую энергию, используя явление электромагнитной индукции, которое тесно связано с магнитным полем проводника с током.
    \item Трансформаторы: Изменяют напряжение переменного тока, используя явление электромагнитной индукции в двух или более обмотках, связанных магнитным полем.
    \item Электромагниты: Используются в различных устройствах, таких как реле, подъемные краны, и т.д.
    \item Магнитные записывающие устройства: Жесткие диски и магнитные ленты используют магнитное поле для записи и хранения информации.
\end{itemize}

\vspace{-9pt}
\subsection*{Основные выводы:}
\vspace{-3pt}
\begin{itemize}
    \item Электрический ток создает вокруг себя магнитное поле.
    \item Форма магнитного поля зависит от формы проводника с током.
    \item Направление магнитного поля определяется правилом буравчика.
    \item Магнитная индукция - это векторная величина, характеризующая магнитное поле.
    \item Проводники с током взаимодействуют друг с другом посредством своих магнитных полей.
\end{itemize}

\newpage
\section*{3.3.3 Сила Ампера}
\vspace{-9pt}
\subsection*{Сила Ампера:}
\vspace{-3pt}
Сила, действующая на проводник с током в магнитном поле.

\vspace{-9pt}
\subsection*{Направление:}
\vspace{-3pt}
Определяется правилом левой руки (если расположить левую руку так, чтобы линии магнитного поля входили в ладонь, а четыре пальца указывали направление тока, то отогнутый на 90 градусов большой палец покажет направление силы Ампера).

\vspace{-9pt}
\subsection*{Формула:}
\vspace{-3pt}

\vspace{-0.05em}
$$ F_А = I \cdot B \cdot l \cdot \sin(\alpha) $$
где:
\begin{itemize}
    \item $F_А$ – сила Ампера,
    \item $I$ – сила тока в проводнике,
    \item $B$ – индукция магнитного поля,
    \item $l$ – длина проводника, находящегося в магнитном поле,
    \item $\alpha$ – угол между направлением тока и вектором магнитной индукции.
\end{itemize}


\section*{3.3.4 Сила Лоренца}
\vspace{-9pt}
\subsection*{Сила Лоренца:}
\vspace{-3pt}
Сила, действующая на движущийся электрический заряд в магнитном поле.

\vspace{-9pt}
\subsection*{Направление:}
\vspace{-3pt}
Определяется правилом левой руки (если расположить левую руку так, чтобы линии магнитного поля входили в ладонь, а четыре пальца указывали направление движения положительного заряда, то отогнутый на 90 градусов большой палец покажет направление силы Лоренца). Для отрицательного заряда направление силы противоположное.

\vspace{-9pt}
\subsection*{Формула:}
\vspace{-3pt} 
\vspace{-0.05em}
$$ F_Л = q \cdot v \cdot B \cdot \sin(\alpha) $$
где:
\begin{itemize}
    \item $F_Л$ – сила Лоренца,
    \item $q$ – величина заряда,
    \item $v$ – скорость движения заряда,
    \item $B$ – индукция магнитного поля,
    \item $\alpha$ – угол между направлением скорости и вектором магнитной индукции.
\end{itemize}

\vspace{-9pt}
\subsection*{Движение заряженной частицы в магнитном поле:}
\vspace{-3pt}
Если скорость частицы перпендикулярна магнитному полю, то частица движется по окружности.



\end{document}