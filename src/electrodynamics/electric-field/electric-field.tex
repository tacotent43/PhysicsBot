\documentclass[a4paper,12pt]{article}
\usepackage{fancyhdr}
\usepackage{graphicx} 
\usepackage{lipsum}
\usepackage{geometry}
\usepackage{mathtext}
\usepackage[T2A]{fontenc}
\usepackage[utf8]{inputenc}
\usepackage[english, russian]{babel}
\usepackage{titlesec}
\usepackage{xcolor}
\usepackage{graphicx}
\usepackage{caption}
\usepackage{subcaption}
\usepackage{wrapfig}
\usepackage{gensymb}
\usepackage{amsmath}
\usepackage{amssymb}
\usepackage{tikz}
\usepackage{circuitikz} 
\usepackage{physics}

\usepackage{enumitem}

\geometry{top=2cm, bottom=2cm, left=1cm, right=1cm}

\pagestyle{fancy}
\fancyhf{} 
\fancyfoot[C]{\rule[20pt]{\textwidth}{0.4pt}}
\fancyhead{} 
\fancyfoot[L]{Page \thepage} 
\fancyfoot[R]{\copyright\ Бот для подготовки к ЕГЭ по физике. 2025г} 

\titleformat{\section}[block]{\normalfont\Large\bfseries}{\thesection}{3em}{}
\setlist[itemize]{label=\scriptsize\textbullet}


\begin{document}

\begin{center}
\scalebox{2}{\textbf{Электрическое поле}}
\end{center}

\vspace{-2.5em}

%%%%%%%%%%%%%%%%%
% \section*{}


% \vspace{-9pt}
% \subsection*{}
% \vspace{-3pt}


% \begin{enumerate} [itemsep=0pt, topsep=0pt, parsep=0pt]
%   \item $p$ – импульс тела;
%   \item $m$ – масса тела;
%   \item $v$ – скорость тела.
% \end{enumerate}


% [itemsep=0pt, topsep=0pt, parsep=0pt]
%%%%%%%%%%%%%%%%%

% 1.1
\section*{1.1 Электризация тел}
\vspace{-9pt}
\subsection*{Определение:}
\vspace{-3pt}
\textbf{Электризация} — это процесс, при котором тела приобретают электрический заряд в результате перераспределения зарядов между ними.

\vspace{-9pt}
\subsection*{Способы электризации:}
\vspace{-3pt}
\begin{enumerate} [itemsep=0pt, topsep=0pt, parsep=3pt]
  \item \textbf{Электризация трением:} При трении двух тел часть электронов переходит с одного тела на другое, в результате чего тела приобретают заряды противоположных знаков.
  \item \textbf{Электризация соприкосновением:} При контакте заряженного тела с нейтральным заряд перераспределяется между ними.
  \item \textbf{Электризация через влияние (электростатическая индукция):} Под действием внешнего заряженного тела в нейтральном теле происходит перераспределение зарядов, и оно приобретает разноименные заряды на разных концах.
\end{enumerate}

\vspace{-9pt}
\subsection*{Закон сохранения заряда:}
\vspace{-3pt}
В замкнутой системе алгебраическая сумма электрических зарядов остается постоянной:
\vspace{-0.05em}
$$ q_1 + q_2 + \dots + q_n = \text{const} $$

\vspace{-18pt}
\subsection*{Типы зарядов:}
\vspace{-3pt}
\begin{itemize} [itemsep=0pt, topsep=0pt, parsep=0pt]
  \item \textbf{Положительный заряд:} Обусловлен недостатком электронов.
  \item \textbf{Отрицательный заряд:} Обусловлен избытком электронов.
\end{itemize}

\vspace{-18pt}
\subsection*{Взаимодействие зарядов:}
\vspace{-3pt}
\begin{itemize} [itemsep=0pt, topsep=0pt, parsep=0pt]
  \item Одноименные заряды отталкиваются.
  \item Разноименные заряды притягиваются.
\end{itemize}

\vspace{-18pt}
\subsection*{Электрическое поле:}
\vspace{-3pt}
\begin{itemize}
  \item Заряженные тела создают вокруг себя электрическое поле, которое действует на другие заряженные тела.
  \item Напряженность электрического поля ($E$) определяется силой, действующей на единичный положительный заряд:
  \vspace{-0.05em}
  $$ E = \frac{F}{q} $$
  где:
  \begin{itemize} [itemsep=0pt, topsep=0pt, parsep=0pt]
    \item $F$ — сила, действующая на заряд.
    \item $q$ — величина заряда.
  \end{itemize}
\end{itemize}

\vspace{-18pt}
\subsection*{Примеры электризации:}
\vspace{-3pt}
\begin{itemize}
  \item Электризация волос при расчесывании.
  \item Электризация воздушного шарика при трении о шерсть.
  \item Электризация через влияние в электроскопе.
\end{itemize}

\newpage

% 1.4
\section*{1.4 Закон Кулона}

\vspace{-9pt}
\subsection*{Формулировка:}
\vspace{-3pt}
Сила взаимодействия между двумя точечными зарядами прямо пропорциональна произведению модулей зарядов и обратно пропорциональна квадрату расстояния между ними.

\vspace{-9pt}
\subsection*{Формула:}
\vspace{-3pt}
\begin{enumerate}[itemsep=0pt, topsep=0pt, parsep=3pt]
  \item Закон Кулона:
  \vspace{-0.05em}
  \vspace{-6pt}
  $$ F = k \cdot \frac{|q_1 \cdot q_2|}{r^2} $$
  где:
  \begin{itemize}
    \item $F$ — сила Кулона.
    \item $k$ — коэффициент пропорциональности (электрическая постоянная). $k = 9 \cdot 10^9$ $\frac{Н \cdot м^2}{Кл^2}$.
    \item $q_1, q_2$ — величины точечных зарядов.
    \item $r$ — расстояние между зарядами.
  \end{itemize}
\end{enumerate}

\vspace{-9pt}
\subsection*{Векторная форма:}
\vspace{-3pt}
\begin{enumerate}[itemsep=0pt, topsep=0pt, parsep=3pt]
  \item Векторная форма закона Кулона:
  \vspace{-0.05em}
  $$ \vec{F}_{12} = k \cdot \frac{q_1 \cdot q_2}{r^2} \cdot \frac{\vec{r}_{12}}{r} $$
  где $\vec{r}_{12}$ — вектор, направленный от заряда $q_1$ к $q_2$.
\end{enumerate}

\vspace{-9pt}
\subsection*{Сила Кулона:}
\vspace{-3pt}
\begin{itemize}
  \item Сила отталкивания, если заряды одноименные.
  \item Сила притяжения, если заряды разноименные.
  \item Действует вдоль прямой, соединяющей заряды.
\end{itemize}

\newpage

% 1.5
\section*{1.5 Действие электрического поля на электрические заряды}

\vspace{-9pt}
\subsection*{Электрическое поле:}
\vspace{-3pt}
Особая форма материи, существующая вокруг электрических зарядов и проявляющаяся в воздействии на другие заряды, помещенные в это поле.

\vspace{-9pt}
\subsection*{Силовое воздействие:}
\vspace{-3pt}
Электрическое поле оказывает силовое воздействие на заряды.

\vspace{-9pt}
\subsection*{Характер действия:}
\vspace{-3pt}
\begin{itemize}
  \item Положительный заряд движется в направлении силовых линий электрического поля.
  \item Отрицательный заряд движется против силовых линий электрического поля.
\end{itemize}

\vspace{-9pt}
\subsection*{Сила, действующая на заряд в электрическом поле:}
\vspace{-3pt}
\vspace{-0.05em}
$$ F = q \cdot E, $$
где:
\begin{itemize}
  \item $F$ — сила, действующая на заряд.
  \item $q$ — величина заряда.
  \item $E$ — напряженность электрического поля.
\end{itemize}

\newpage

% 1.6
\section*{1.6 Напряженность электрического поля}

\vspace{-9pt}
\subsection*{Определение:}
\vspace{-3pt}
Векторная физическая величина, характеризующая электрическое поле в данной точке пространства. Напряженность электрического поля равна отношению силы, действующей на пробный положительный заряд, помещенный в эту точку, к величине этого заряда.

\vspace{-9pt}
\subsection*{Формула:}
\vspace{-3pt}
\begin{enumerate}[itemsep=0pt, topsep=0pt, parsep=3pt]
  \item Напряженность электрического поля:
  \vspace{-0.05em}
  $$ E = \frac{F}{q_0} $$
  где:
  \begin{itemize} [itemsep=0pt, topsep=0pt, parsep=0pt]
    \item $E$ — напряженность электрического поля.
    \item $F$ — сила, действующая на пробный заряд $q_0$.
    \item $q_0$ — пробный положительный заряд.
  \end{itemize}
\end{enumerate}

\vspace{-9pt}
\subsection*{Направление:}
\vspace{-3pt}
Направление напряженности совпадает с направлением силы, действующей на положительный заряд.

\vspace{-9pt}
\subsection*{Напряженность поля точечного заряда:}
\vspace{-3pt}
\begin{enumerate}[itemsep=0pt, topsep=0pt, parsep=3pt]
  \item Формула:
  \vspace{-0.05em}
  $$ E = k \cdot \frac{|q|}{r^2} $$
  где:
  \begin{itemize} [itemsep=0pt, topsep=0pt, parsep=0pt]
    \item $k$ — коэффициент пропорциональности ($k = 9 \cdot 10^9$ $\frac{Н \cdot м^2}{Кл^2}$).
    \item $q$ — величина точечного заряда.
    \item $r$ — расстояние от заряда до точки, в которой определяется напряженность.
  \end{itemize}
\end{enumerate}

\newpage
\vspace{-9pt}
\subsection*{Единица измерения:}
\vspace{-3pt}
\begin{itemize} [itemsep=0pt, topsep=0pt, parsep=0pt]
  \item В/м (вольт на метр) или Н/Кл (ньютон на кулон).
\end{itemize}

% 1.7
\section*{1.7 Принцип суперпозиции электрических полей}
\vspace{-9pt}
\subsection*{Формулировка:}
\vspace{-3pt}
Напряженность электрического поля, созданного несколькими зарядами, равна векторной сумме напряженностей полей, созданных каждым из зарядов в отдельности.

\vspace{-9pt}
\subsection*{Математически:}
\vspace{-3pt}
\begin{enumerate}[itemsep=0pt, topsep=0pt, parsep=3pt]
  \item Формула:
  \vspace{-0.05em}
  $$ \vec{E}_{\text{рез}} = \vec{E}_1 + \vec{E}_2 + \dots + \vec{E}_n $$
\end{enumerate}

\vspace{-9pt}
\subsection*{Применение:}
\vspace{-3pt}
Позволяет рассчитать электрическое поле, созданное сложной системой зарядов.

\section*{1.8 Потенциальность электростатического поля}
\vspace{-9pt}
\subsection*{Потенциальное поле (консервативное поле):}
\vspace{-3pt}
Работа сил электрического поля при перемещении заряда не зависит от формы траектории и определяется только начальным и конечным положениями заряда.

\vspace{-9pt}
\subsection*{Электростатическое поле:}
\vspace{-3pt}
Электрическое поле, создаваемое неподвижными зарядами, является потенциальным полем.

\vspace{-9pt}
\subsection*{Работа электрического поля:}
\vspace{-3pt}
\begin{enumerate}[itemsep=0pt, topsep=0pt, parsep=3pt]
  \item Формула:
  \vspace{-0.05em}
  $$ A = q(\phi_1 - \phi_2) $$
  где:
  \begin{itemize} [itemsep=0pt, topsep=0pt, parsep=0pt]
    \item $\phi_1$ и $\phi_2$ — потенциалы в начальной и конечной точках.
  \end{itemize}
\end{enumerate}

\section*{1.9 Потенциал электрического поля. Разность потенциалов}
\vspace{-9pt}
\subsection*{Потенциал электрического поля ($\phi$):}
\vspace{-3pt}
Скалярная физическая величина, характеризующая энергетические свойства электростатического поля в данной точке. Потенциал равен отношению потенциальной энергии пробного заряда, помещенного в данную точку, к величине этого заряда:
\vspace{-0.05em}
$$ \phi = \frac{E_p}{q_0} $$

\vspace{-9pt}
\subsection*{Потенциал поля точечного заряда:}
\vspace{-3pt}
\begin{enumerate}[itemsep=0pt, topsep=0pt, parsep=3pt]
  \item Формула:
  \vspace{-0.05em}
  $$ \phi = k \cdot \frac{q}{r} $$
\end{enumerate}

\vspace{-9pt}
\subsection*{Единица измерения:}
\vspace{-3pt}
Вольт (В).

\vspace{-9pt}
\subsection*{Разность потенциалов (напряжение, $U$):}
\vspace{-3pt}
Разность потенциалов между двумя точками электрического поля:
\vspace{-0.05em}
$$ U = \phi_1 - \phi_2 $$

\vspace{-9pt}
\subsection*{Связь между напряженностью и разностью потенциалов:}
\vspace{-3pt}
\begin{enumerate}[itemsep=0pt, topsep=0pt, parsep=3pt]
  \item Формула:
  \vspace{-0.05em}
  $$ E = - \frac{\Delta \phi}{\Delta x} $$
  (связь для однородного поля)
  \item Для однородного электрического поля:
  \vspace{-0.05em}
  $$ U = E \cdot d $$
  где $d$ — расстояние между точками с разностью потенциалов $U$.
\end{enumerate}

\section*{1.10 Проводники в электрическом поле}
\vspace{-9pt}
\subsection*{Проводники:}
\vspace{-3pt}
Вещества, содержащие свободные заряды (электроны), способные перемещаться под действием электрического поля.

\vspace{-9pt}
\subsection*{Поведение в электрическом поле:}
\vspace{-3pt}
\begin{itemize} [itemsep=0pt, topsep=0pt, parsep=0pt]
  \item Свободные заряды в проводнике перемещаются под действием электрического поля, пока поле внутри проводника не станет равным нулю (перераспределение зарядов).
  \item На поверхности проводника возникают индуцированные заряды, которые компенсируют внешнее электрическое поле внутри проводника.
  \item Поле внутри проводника равно нулю.
  \item Поверхность проводника является эквипотенциальной поверхностью (потенциал одинаков во всех точках).
\end{itemize}


\section*{1.11 Диэлектрики в электрическом поле}
\vspace{-9pt}
\subsection*{Диэлектрики:}
\vspace{-3pt}
Вещества, в которых отсутствуют свободные заряды, но есть электрические диполи (молекулы, у которых центры положительного и отрицательного зарядов не совпадают).

\vspace{-9pt}
\subsection*{Поляризация диэлектриков:}
\vspace{-3pt}
Под действием электрического поля диполи диэлектрика ориентируются вдоль поля, что приводит к ослаблению поля внутри диэлектрика (диэлектрическая проницаемость).
\begin{itemize} [itemsep=0pt, topsep=0pt, parsep=0pt]
  \item Процесс наведения диполей называют поляризацией.
  \item Поляризация ведет к появлению связанных зарядов на границе диэлектрика.
\end{itemize}

\vspace{-9pt}
\subsection*{Диэлектрическая проницаемость ($\epsilon$):}
\vspace{-3pt}
Характеризует способность диэлектрика ослаблять электрическое поле.
\begin{enumerate}[itemsep=0pt, topsep=0pt, parsep=3pt]
  \item Напряженность поля в диэлектрике:
  \vspace{-0.05em}
  $$ E_{\text{диэл}} = \frac{E_{\text{вакуум}}}{\epsilon} $$
\end{enumerate}

\vspace{-9pt} 
\subsection*{Ослабление поля:}
\vspace{-3pt}
Электрическое поле внутри диэлектрика становится меньше, чем внешнее электрическое поле.

\section*{1.12 Электрическая емкость. Конденсатор}
\vspace{-9pt}
\subsection*{Электрическая емкость ($C$):}
\vspace{-3pt}
Мера способности тела накапливать электрический заряд.

\vspace{-9pt}
\subsection*{Формула:}
\vspace{-3pt}
\begin{enumerate}[itemsep=0pt, topsep=0pt, parsep=3pt]
  \item Формула:
  \vspace{-0.05em}
  $$ C = \frac{q}{U} $$
  где:
  \begin{itemize} [itemsep=0pt, topsep=0pt, parsep=0pt]
    \item $C$ — электрическая емкость.
    \item $q$ — заряд тела.
    \item $U$ — потенциал тела или разность потенциалов.
  \end{itemize}
\end{enumerate}

\vspace{-9pt}
\subsection*{Единица измерения:}
\vspace{-3pt}
Фарад (Ф). 1 Ф = 1 Кл/В.

\vspace{-9pt}
\subsection*{Конденсатор:}
\vspace{-3pt}
Устройство, предназначенное для накопления электрического заряда и энергии.
\begin{itemize} [itemsep=0pt, topsep=0pt, parsep=0pt]
  \item Состоит из двух проводников (обкладок), разделенных диэлектриком.
\end{itemize}

\vspace{-9pt}
\subsection*{Емкость плоского конденсатора:}
\vspace{-3pt}
\begin{enumerate}[itemsep=0pt, topsep=0pt, parsep=3pt]
  \item Формула:
  \vspace{-0.05em}
  $$ C = \frac{\epsilon \epsilon_0 \cdot S}{d} $$
  где:
  \begin{itemize} [itemsep=0pt, topsep=0pt, parsep=0pt]
    \item $\epsilon$ — диэлектрическая проницаемость вещества между обкладками.
    \item $\epsilon_0$ — электрическая постоянная.
    \item $S$ — площадь обкладок.
    \item $d$ — расстояние между обкладками.
  \end{itemize}
\end{enumerate}

\newpage
\section*{1.13 Энергия электрического поля конденсатора}

\vspace{-9pt}
\subsection*{Энергия заряженного конденсатора ($W$):}
\vspace{-3pt}
Энергия, запасенная в электрическом поле, созданном зарядами на обкладках конденсатора.

\vspace{-9pt}
\subsection*{Формулы:}
\vspace{-3pt}
\begin{enumerate}[itemsep=0pt, topsep=0pt, parsep=3pt]
  \item Формула 1:
  \vspace{-0.05em}
  $$ W = \frac{qU}{2} $$
  \item Формула 2:
  \vspace{-0.05em}
  $$ W = \frac{CU^2}{2} $$
  \item Формула 3:
  \vspace{-0.05em}
  $$ W = \frac{q^2}{2C} $$
  где:
  \begin{itemize} [itemsep=0pt, topsep=0pt, parsep=0pt]
    \item $W$ - Энергия на обкладках конденсатора
    \item $C$ — электрическая емкость.
    \item $q$ — заряд тела.
    \item $U$ — потенциал тела или разность потенциалов.
  \end{itemize}
\end{enumerate}

\end{document}