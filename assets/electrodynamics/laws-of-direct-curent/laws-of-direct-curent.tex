\documentclass[a4paper,12pt]{article}
\usepackage{fancyhdr}
\usepackage{graphicx} 
\usepackage{lipsum}
\usepackage{geometry}
\usepackage{mathtext}
\usepackage[T2A]{fontenc}
\usepackage[utf8]{inputenc}
\usepackage[english, russian]{babel}
\usepackage{titlesec}
\usepackage{xcolor}
\usepackage{graphicx}
\usepackage{caption}
\usepackage{subcaption}
\usepackage{wrapfig}
\usepackage{gensymb}
\usepackage{amsmath}
\usepackage{amssymb}
\usepackage{tikz}
\usepackage{circuitikz} 
\usepackage{physics}

\usepackage{enumitem}

\geometry{top=2cm, bottom=2cm, left=1cm, right=1cm}

\pagestyle{fancy}
\fancyhf{} 
\fancyfoot[C]{\rule[20pt]{\textwidth}{0.4pt}}
\fancyhead{} 
\fancyfoot[L]{Page \thepage} 
\fancyfoot[R]{\copyright\ Бот для подготовки к ЕГЭ по физике. 2025г} 

\titleformat{\section}[block]{\normalfont\Large\bfseries}{\thesection}{3em}{}
\setlist[itemize]{label=\scriptsize\textbullet}


\begin{document}

\begin{center}
\scalebox{2}{\textbf{Законы постоянного тока}}
\end{center}

\vspace{-2.5em}

\section*{3.2.1 Постоянный электрический ток. Сила тока}
\vspace{-9pt}
\subsection*{Постоянный электрический ток:}
\vspace{-3pt}
Упорядоченное движение электрических зарядов (носителей тока) под действием электрического поля, при котором сила тока не меняется со временем.

\vspace{-9pt}
\subsection*{Направление тока:}
\vspace{-3pt}
Принято считать, что направление тока совпадает с направлением движения положительных зарядов (противоположно направлению движения электронов).

\vspace{-9pt}
\subsection*{Сила тока ($I$):}
\vspace{-3pt}
Физическая величина, равная отношению количества электрического заряда ($\Delta q$), прошедшего через поперечное сечение проводника за время ($\Delta t$), к этому промежутку времени.

\vspace{-9pt}
\subsection*{Формула:}
\vspace{-3pt}
\vspace{-0.05em}
$$ I = \frac{\Delta q}{\Delta t} $$
где:
\begin{itemize}
    \item $I$ - сила тока,
    \item $\Delta q$ - величина заряда, прошедшего через сечение проводника,
    \item $\Delta t$ - время прохождения заряда.
\end{itemize}

\vspace{-9pt}
\subsection*{Единица измерения:}
\vspace{-3pt}
Ампер (А). 1 = 1 kл /с.
\vspace{-18pt}

\newpage
\section*{3.2.2 Постоянный электрический ток. Напряжение}
\vspace{-9pt}
\subsection*{Напряжение ($U$):}
\vspace{-3pt}
Физическая величина, характеризующая работу электрического поля по перемещению единичного положительного заряда на участке цепи.

\vspace{-9pt}
\subsection*{Формула:}
\vspace{-3pt}
\vspace{-0.05em}
$$ U = \frac{A}{q} $$
где:
\begin{itemize}
    \item $U$ - напряжение,
    \item $A$ - работа электрического поля по перемещению заряда $q$,
    \item $q$ - величина переносимого заряда.
\end{itemize}

\vspace{-9pt}
\subsection*{Единица измерения:}
\vspace{-9pt}
Вольт (В). 1 В = 1 Дж/Кл.
\vspace{-9pt}
\subsection*{Разность потенциалов:}
\vspace{-3pt}
Напряжение между двумя точками электрической цепи равно разности потенциалов между этими точками:
\vspace{-0.05em}
$$ U = \phi_1 - \phi_2 $$

\section*{3.2.3 Закон Ома для участка цепи}
\vspace{-9pt}
\subsection*{Формулировка:}
\vspace{-3pt}
Сила тока в участке цепи прямо пропорциональна напряжению на этом участке и обратно пропорциональна его электрическому сопротивлению.

\vspace{-9pt}
\subsection*{Формула:}
\vspace{-3pt}
\vspace{-0.05em}
$$ I = \frac{U}{R} $$
где:
\begin{itemize}
    \item $I$ - сила тока,
   \item $U$ - напряжение на участке цепи,
    \item $R$ - электрическое сопротивление участка цепи.
\end{itemize}


\vspace{-9pt}
\subsection*{Зависимость:}
\vspace{-3pt}
Закон Ома справедлив для металлических проводников и некоторых других материалов при постоянной температуре.

\newpage
\section*{3.2.4 Электрическое сопротивление}
\vspace{-9pt}
\subsection*{Электрическое сопротивление ($R$):}
\vspace{-3pt}
Физическая величина, характеризующая свойство проводника препятствовать прохождению электрического тока.

\vspace{-9pt}
\subsection*{Формула (зависимость от геометрических параметров):}
\vspace{-3pt}
\vspace{-0.05em}
$$ R = \rho \cdot \frac{l}{S} $$
где:
\begin{itemize}
    \item $R$ - сопротивление проводника,
    \item $\rho$ - удельное сопротивление материала проводника (зависит от материала и температуры),
    \item $l$ - длина проводника,
    \item $S$ - площадь поперечного сечения проводника.
\end{itemize}

\vspace{-9pt}
\subsection*{Единица измерения:}
\vspace{-3pt}
Ом (Ом). 1 Ом = 1 В/А.

\vspace{-9pt}
\subsection*{Удельное сопротивление:}
\vspace{-3pt}
Сопротивление проводника длиной 1 метр и поперечным сечением 1 м².

\vspace{-9pt}
\subsection*{Зависимость сопротивления от температуры:}
\vspace{-3pt}
\vspace{-0.05em}
$$ R = R_0(1+\alpha\Delta T) $$
где:
\begin{itemize}
    \item $R_0$ - сопротивление при начальной температуре,
    \item $\alpha$ - температурный коэффициент сопротивления.
\end{itemize}

\newpage
\section*{3.2.5 Электродвижущая сила. Внутреннее сопротивление источника тока}
\vspace{-9pt}
\subsection*{Электродвижущая сила (ЭДС) ($\epsilon$):}
\vspace{-3pt}
Работа, совершаемая сторонними силами при перемещении единичного положительного заряда внутри источника тока (от отрицательного полюса к положительному).

\vspace{-9pt}
\subsection*{Формула:}
\vspace{-3pt}
\vspace{-0.05em}
$$ \epsilon = \frac{A_{ст}}{q} $$
где:
\begin{itemize}
    \item $\epsilon$ - ЭДС источника тока,
    \item $A_{ст}$ - работа сторонних сил,
    \item $q$ - величина переносимого заряда.
\end{itemize}

\vspace{-9pt}
\subsection*{Единица измерения:}
\vspace{-3pt}
Вольт (В).

\vspace{-9pt}
\subsection*{Внутреннее сопротивление источника тока ($r$):}
\vspace{-3pt}
Сопротивление, которое оказывает сам источник тока прохождению тока внутри себя.

\vspace{-9pt}
\subsection*{Наличие внутреннего сопротивления:}
\vspace{-3pt}
Поток зарядов встречает сопротивление не только во внешней цепи, но и внутри самого источника.

\newpage
\section*{3.2.6 Закон Ома для полной электрической цепи}
\vspace{-9pt}
\subsection*{Формулировка:}
\vspace{-3pt}
Сила тока в полной электрической цепи прямо пропорциональна ЭДС источника тока и обратно пропорциональна сумме внешнего сопротивления и внутреннего сопротивления источника.

\vspace{-9pt}
\subsection*{Формула:}
\vspace{-3pt}
\vspace{-0.05em}
$$ I = \frac{\epsilon}{R + r} $$
где:
\begin{itemize}
    \item $I$ - сила тока в цепи,
    \item $\epsilon$ - ЭДС источника тока,
    \item $R$ - внешнее сопротивление цепи,
    \item $r$ - внутреннее сопротивление источника тока.
\end{itemize}

\vspace{-9pt}
\subsection*{Напряжение на клеммах источника:}
\vspace{-3pt}
\vspace{-0.05em}
$$ U = \epsilon - Ir $$

\vspace{-9pt}
\subsection*{Ток короткого замыкания:}
\vspace{-3pt}
\vspace{-0.05em}
$$ I_{кз}= \frac{\epsilon}{r} $$

\newpage
\section*{3.2.7 Параллельное и последовательное соединение проводников}
\vspace{-9pt}
\subsection*{Последовательное соединение:}
\vspace{-3pt}
Проводники соединены "друг за другом", через них проходит один и тот же ток.

\vspace{-9pt}
\subsection*{Общее сопротивление:}
\vspace{-3pt}
\vspace{-0.05em}
$$ R_{общ} = R_1 + R_2 + ... + R_n $$

\vspace{-9pt}
\subsection*{Сила тока:}
\vspace{-3pt}
\vspace{-0.05em}
$$ I = I_1 = I_2 = ... = I_n $$

\vspace{-9pt}
\subsection*{Напряжение:}
\vspace{-3pt}
\vspace{-0.05em}
$$ U = U_1 + U_2 + ... + U_n $$

\vspace{-9pt}
\subsection*{Параллельное соединение:}
\vspace{-3pt}
Проводники соединены так, что начала всех проводников соединены в одну точку, и концы всех проводников - в другую точку. Напряжение на всех проводниках одинаково.

\vspace{-9pt}
\subsection*{Общее сопротивление:}
\vspace{-3pt}
\vspace{-0.05em}
$$ \frac{1}{R_{общ}} = \frac{1}{R_1} + \frac{1}{R_2} + ... + \frac{1}{R_n} $$

\vspace{-9pt}
\subsection*{Сила тока:}
\vspace{-3pt}
\vspace{-0.05em}
$$ I = I_1 + I_2 + ... + I_n $$

\vspace{-9pt}
\subsection*{Напряжение:}
\vspace{-3pt}
\vspace{-0.05em}
$$ U = U_1 = U_2 = ... = U_n $$

\section*{3.2.8 Смешанное соединение проводников}
\vspace{-9pt}
\subsection*{Определение:}
\vspace{-3pt}
Соединение, в котором присутствуют как последовательно, так и параллельно соединенные участки цепи.

\vspace{-9pt}
\subsection*{Расчет:}
\vspace{-3pt}
Для расчета общего сопротивления необходимо последовательно упрощать схему, заменяя параллельные и последовательные участки их эквивалентными сопротивлениями.

\newpage
\section*{3.2.12 Полупроводники. Собственная и примесная проводимость полупроводников}
\vspace{-9pt}
\subsection*{Полупроводники:}
\vspace{-3pt}
Вещества, имеющие проводимость между проводниками и диэлектриками (кремний, германий).

\vspace{-9pt}
\subsection*{Собственная проводимость:}
\vspace{-3pt}
Проводимость чистого полупроводника.

\vspace{-9pt}
\subsection*{Носители тока:}
\vspace{-3pt}
Электроны и дырки (положительные вакансии, образованные при уходе электронов).

\vspace{-9pt}
\subsection*{Число электронов и дырок:}
\vspace{-3pt}
Одинаково.

\vspace{-9pt}
\subsection*{Примесная проводимость:}
\vspace{-3pt}
Проводимость полупроводника с добавлением примесей.

\vspace{-9pt}
\subsection*{n-тип (электронная проводимость):}
\vspace{-3pt}
При добавлении примесей, отдающих электроны (например, фосфор в кремний), число свободных электронов увеличивается, электроны становятся основными носителями тока.

\vspace{-9pt}
\subsection*{p-тип (дырочная проводимость):}
\vspace{-3pt}
При добавлении примесей, захватывающих электроны (например, бор в кремний), число дырок увеличивается, дырки становятся основными носителями тока.

\end{document}