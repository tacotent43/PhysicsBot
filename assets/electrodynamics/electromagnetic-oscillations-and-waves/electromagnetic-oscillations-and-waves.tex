\documentclass[a4paper,12pt]{article}
\usepackage{fancyhdr}
\usepackage{graphicx} 
\usepackage{lipsum}
\usepackage{geometry}
\usepackage{mathtext}
\usepackage[T2A]{fontenc}
\usepackage[utf8]{inputenc}
\usepackage[english, russian]{babel}
\usepackage{titlesec}
\usepackage{xcolor}
\usepackage{graphicx}
\usepackage{caption}
\usepackage{subcaption}
\usepackage{wrapfig}
\usepackage{gensymb}
\usepackage{amsmath}
\usepackage{amssymb}
\usepackage{tikz}
\usepackage{circuitikz} 
\usepackage{physics}

\usepackage{enumitem}

\geometry{top=2cm, bottom=2cm, left=1cm, right=1cm}

\pagestyle{fancy}
\fancyhf{} 
\fancyfoot[C]{\rule[20pt]{\textwidth}{0.4pt}}
\fancyhead{} 
\fancyfoot[L]{Page \thepage} 
\fancyfoot[R]{\copyright\ Бот для подготовки к ЕГЭ по физике. 2025г} 

\titleformat{\section}[block]{\normalfont\Large\bfseries}{\thesection}{3em}{}
\setlist[itemize]{label=\scriptsize\textbullet}


\begin{document}

\begin{center}
\scalebox{2}{\textbf{Электромагнитные колебания и волны}}
\end{center}

\vspace{-2.5em}

%%%%%%%%%%%%%%%%%
% \section*{}


% \vspace{-9pt}
% \subsection*{}
% \vspace{-3pt}


% \begin{enumerate} [itemsep=0pt, topsep=0pt, parsep=0pt]
%   \item $p$ – импульс тела;
%   \item $m$ – масса тела;
%   \item $v$ – скорость тела.
% \end{enumerate}


% [itemsep=0pt, topsep=0pt, parsep=0pt]
%%%%%%%%%%%%%%%%%

\section*{3.5.1 Свободные электромагнитные колебания. Колебательный контур}
\vspace{-9pt}
\subsection*{Свободные электромагнитные колебания:}
\vspace{-3pt}
Колебания электрического заряда и тока в колебательном контуре без внешнего воздействия.

\vspace{-9pt}
\subsection*{Колебательный контур:}
\vspace{-3pt}
Простейшая электрическая цепь, состоящая из последовательно соединенных конденсатора ($C$) и катушки индуктивности ($L$).

\vspace{-9pt}
\subsection*{Преобразование энергии:}
\vspace{-3pt}
В колебательном контуре происходит взаимное преобразование энергии электрического поля конденсатора в энергию магнитного поля катушки и обратно.

\vspace{-9pt}
\subsection*{Период собственных колебаний (период Томсона):}
\vspace{-3pt}

\subsubsection*{Формула:}
\vspace{-0.05em}
$$ T = 2\pi \sqrt{LC} $$
где:
\begin{itemize}
    \item $L$ — индуктивность.
    \item $C$ — ёмкость.
\end{itemize}


\vspace{-9pt}
\subsection*{Частота собственных колебаний:}
\vspace{-3pt}
\subsubsection*{Формула:}
\vspace{-0.05em}
$$ \nu = \frac{1}{T} $$


\section*{3.5.2 Вынужденные электромагнитные колебания. Резонанс}
\vspace{-9pt}
\subsection*{Вынужденные электромагнитные колебания:}
\vspace{-3pt}
Колебания электрического заряда и тока в контуре под действием внешней переменной ЭДС.

\vspace{-9pt}
\subsection*{Резонанс:}
\vspace{-3pt}
Явление резкого возрастания амплитуды вынужденных колебаний, когда частота внешней силы близка к собственной частоте колебательного контура.

\vspace{-9pt}
\subsection*{Условие резонанса:}
\vspace{-3pt}
Частота вынужденной ЭДС равна собственной частоте контура.

\newpage
\section*{3.5.3 Гармонические электромагнитные колебания}
\vspace{-9pt}
\subsection*{Гармонические колебания:}
\vspace{-3pt}
Колебания, при которых заряд, ток и напряжение изменяются со временем по закону синуса или косинуса.

\vspace{-9pt}
\subsection*{Описание:}
\vspace{-3pt}
Аналогичны механическим гармоническим колебаниям.

\vspace{-9pt}
\subsection*{Уравнение колебаний:}
\vspace{-3pt}
\subsubsection*{Формула:}
\vspace{-0.05em}
$$ q(t) = q_{\text{max}} \cdot \cos(\omega t + \phi) $$
где:
\begin{itemize}
    \item $q_{\text{max}}$ — амплитуда заряда.
    \item $\omega$ — циклическая частота.
    \item $\phi$ — начальная фаза.
\end{itemize}

\newpage
\section*{3.5.4 Переменный ток. Производство, передача и потребление электрической энергии}
\vspace{-9pt}
\subsection*{Переменный ток:}
\vspace{-3pt}
Электрический ток, сила и направление которого периодически изменяются со временем.

\vspace{-9pt}
\subsection*{Производство:}
\vspace{-3pt}
Генераторы переменного тока (электромагнитная индукция).

\vspace{-9pt}
\subsection*{Передача:}
\vspace{-3pt}
По линиям электропередачи (высокое напряжение, низкий ток для уменьшения потерь на нагрев).

\vspace{-9pt}
\subsection*{Потребление:}
\vspace{-3pt}
Электрические приборы, трансформаторы для преобразования напряжения.

\vspace{-9pt}
\subsection*{Действующее значение силы тока и напряжения:}
\vspace{-3pt}
\subsubsection*{Формула:}
\vspace{-0.05em}
$$ I_{\text{действ}} = \frac{I_{\text{макс}}}{\sqrt{2}}, \quad U_{\text{действ}} = \frac{U_{\text{макс}}}{\sqrt{2}} $$

\vspace{-9pt}
\subsection*{Мощность переменного тока:}
\vspace{-3pt}
\subsubsection*{Формула:}
\vspace{-0.05em}
$$ P = I_{\text{действ}} \cdot U_{\text{действ}} \cdot \cos(\phi) $$
где $\phi$ — сдвиг фаз между напряжением и током.


\vspace{-9pt}
\subsection*{Трансформатор:}
\vspace{-3pt}
Устройство для преобразования напряжения переменного тока.
\begin{itemize}
    \item Отношение напряжений и токов в трансформаторе:
    \vspace{-0.05em}
    $$ \frac{U_1}{U_2} = \frac{I_2}{I_1} = \frac{n_1}{n_2} $$
    где $n$ — число витков обмотки трансформатора.
\end{itemize}


\section*{3.5.5 Электромагнитное поле}
\vspace{-9pt}
\subsection*{Электромагнитное поле:}
\vspace{-3pt}
Особая форма материи, представляющая собой совокупность электрического и магнитного полей, взаимосвязанных друг с другом.

\vspace{-9pt}
\subsection*{Изменение одного поля:}
\vspace{-3pt}
Изменение электрического поля порождает магнитное поле, и наоборот (вихревое поле).

\vspace{-9pt}
\subsection*{Уравнения Максвелла:}
\vspace{-3pt}
Описывают электромагнитное поле.


\section*{3.5.6 Свойства электромагнитных волн}
\vspace{-9pt}
\subsection*{Электромагнитные волны:}
\vspace{-3pt}
Распространение электромагнитного поля в пространстве (поперечные волны).

\vspace{-9pt}
\subsection*{Свойства:}
\vspace{-3pt}
\begin{itemize}
    \item Распространяются в вакууме со скоростью света ($c \approx 3 \cdot 10^8$ м/с).
    \item Переносят энергию.
    \item Отражаются, преломляются, интерферируют, дифрагируют.
    \item Поляризация.
    \item Несут импульс.
\end{itemize}

\vspace{-9pt}
\subsection*{Скорость электромагнитных волн:}
\vspace{-3pt}
\subsubsection*{Формула:}
\vspace{-0.05em}
$$ v = \lambda \nu $$
где:
\begin{itemize}
    \item $\lambda$ — длина волны.
    \item $\nu$ — частота.
\end{itemize}

\section*{3.5.7 Различные виды электромагнитных излучений и их применение}
\vspace{-9pt}
\subsection*{Шкала электромагнитных волн (по убыванию длины волны/увеличению частоты):}
\vspace{-3pt}
\begin{itemize}
    \item Радиоволны (радиосвязь, телевидение).
    \item Инфракрасное излучение (тепловидение, пульты ДУ).
    \item Видимый свет (зрение, освещение).
    \item Ультрафиолетовое излучение (медицина, стерилизация).
    \item Рентгеновское излучение (медицинская диагностика).
    \item Гамма-излучение (ядерные реакции, радиотерапия).
\end{itemize}



\end{document}