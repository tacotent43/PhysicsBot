\documentclass[a4paper,12pt]{article}
\usepackage{fancyhdr}   % Для настройки заголовков и футеров
\usepackage{graphicx}    % Для работы с изображениями (например, логотип)
\usepackage{lipsum}      % Для примера текста
\usepackage{geometry}    % Для настройки полей страницы
\usepackage{mathtext}
\usepackage[T2A]{fontenc}
\usepackage[utf8]{inputenc}
\usepackage[english, russian]{babel}
\usepackage{titlesec}
\usepackage{xcolor}
\usepackage{graphicx}
\usepackage{caption}
\usepackage{subcaption}
\usepackage{wrapfig}
\usepackage{gensymb}
\usepackage{amsmath}
\usepackage{amssymb}
\usepackage{tikz}
\usepackage{circuitikz} 
\usepackage{physics}

\usepackage{enumitem}

\geometry{top=2cm, bottom=2cm, left=1cm, right=1cm}

% Настройка заголовков и футеров
\pagestyle{fancy}
\fancyhf{} % Убираем все стандартные элементы
\fancyfoot[C]{\rule[20pt]{\textwidth}{0.4pt}}
\fancyhead{} % Убираем верхнюю линию заголовка
\fancyfoot[L]{Page \thepage}  % Нумерация страниц по центру
\fancyfoot[R]{\copyright\ Бот для подготовки к ЕГЭ по физике. 2025г}  % Копирайт в правом нижнем углу


\titleformat{\section}[block]{\normalfont\Large\bfseries}{\thesection}{3em}{}


%~%~%~%~%~%~%~%~%~%~%~%~%~%~%~%~%~%~%~%~%
%          document starts here         %
%~%~%~%~%~%~%~%~%~%~%~%~%~%~%~%~%~%~%~%~%

\begin{document}

\begin{center}
\scalebox{2}{\textbf{Законы сохранения}}
\end{center}

\vspace{-2.5em}

\section*{Импульс тела}

\vspace{-9pt}
\subsection*{Определение}
\vspace{-3pt}
\begin{itemize}[itemsep=0pt, topsep=0pt, parsep=3pt]
  \item \textbf{Импульс тела} ($p$): Векторная физическая величина, являющаяся мерой механического движения тела. Определяется как произведение массы тела на его скорость.
\end{itemize}

\vspace{-9pt}
\subsection*{Формула}
\vspace{-3pt}
\begin{itemize}[itemsep=0pt, topsep=0pt, parsep=3pt]
  \item \textbf{Формула}: \[ p = m \cdot v \]
    \begin{itemize}[itemsep=0pt, topsep=0pt, parsep=3pt]
      \item $p$ – импульс тела,
      \item $m$ – масса тела,
      \item $v$ – скорость тела.
    \end{itemize}
\end{itemize}



\section*{Импульс системы тел}

\vspace{-9pt}
\subsection*{Определение}
\vspace{-3pt}
\begin{itemize}[itemsep=0pt, topsep=0pt, parsep=3pt]
  \item \textbf{Импульс системы тел} ($P$): Векторная сумма импульсов всех тел, входящих в систему.
\end{itemize}

\vspace{-9pt}
\subsection*{Формула}
\vspace{-3pt}
\[ P = p_1 + p_2 + ... + p_n = m_1v_1 + m_2v_2 + ... + m_nv_n \]
\begin{enumerate}[itemsep=0pt, topsep=0pt, parsep=3pt]
  \item $P$ – импульс системы тел,
  \item $p_n$ – импульс $n$-го тела,
  \item $m_n$ – масса $n$-го тела,
  \item $v_n$ – скорость $n$-го тела.
\end{enumerate}


\vspace{-9pt}
\subsection*{Определение}
\vspace{-3pt}
\begin{itemize}[itemsep=0pt, topsep=0pt, parsep=3pt]
  \item \textbf{Закон сохранения импульса}: В замкнутой системе (системе, на которую не действуют внешние силы или сумма внешних сил равна нулю) полный импульс системы остается постоянным с течением времени.
\end{itemize}

\vspace{-9pt}
\subsection*{Формула}
\vspace{-3pt}
\begin{itemize}[itemsep=0pt, topsep=0pt, parsep=3pt]
  \item \textbf{Формула}: \[ P_{\text{начальное}} = P_{\text{конечное}} \] \[ m_1v_1 + m_2v_2 + ... + m_nv_n = m_1u_1 + m_2u_2 + ... + m_nu_n \]
    \begin{itemize}[itemsep=0pt, topsep=0pt, parsep=3pt]
      \item $P_{\text{начальное}}$ – импульс системы до взаимодействия,
      \item $P_{\text{конечное}}$ – импульс системы после взаимодействия,
      \item $v_i$ – скорости тел до взаимодействия,
      \item $u_i$ – скорости тел после взаимодействия.
    \end{itemize}
\end{itemize}

\vspace{-9pt}
\subsection*{Важно}
\vspace{-3pt}
\begin{itemize}[itemsep=0pt, topsep=0pt, parsep=3pt]
  \item Закон выполняется только для замкнутых систем.
  \item Применяется для анализа столкновений, реактивного движения и других процессов.
\end{itemize}








\end{document}