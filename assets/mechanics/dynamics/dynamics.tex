\documentclass[a4paper,12pt]{article}
\usepackage{fancyhdr}   % Для настройки заголовков и футеров
\usepackage{graphicx}    % Для работы с изображениями (например, логотип)
\usepackage{lipsum}      % Для примера текста
\usepackage{geometry}    % Для настройки полей страницы
\usepackage{mathtext}
\usepackage[T2A]{fontenc}
\usepackage[utf8]{inputenc}
\usepackage[english, russian]{babel}
\usepackage{titlesec}
\usepackage{xcolor}
\usepackage{graphicx}
\usepackage{caption}
\usepackage{subcaption}
\usepackage{wrapfig}
\usepackage{gensymb}
\usepackage{amsmath}
\usepackage{amssymb}
\usepackage{tikz}
\usepackage{circuitikz} 
\usepackage{physics}

\usepackage{enumitem}

\geometry{top=2cm, bottom=2cm, left=1cm, right=1cm}

% Настройка заголовков и футеров
\pagestyle{fancy}
\fancyhf{} % Убираем все стандартные элементы
\fancyfoot[C]{\rule[20pt]{\textwidth}{0.4pt}}
\fancyhead{} % Убираем верхнюю линию заголовка
\fancyfoot[L]{Page \thepage}  % Нумерация страниц по центру
\fancyfoot[R]{\copyright\ Бот для подготовки к ЕГЭ по физике. 2025г}  % Копирайт в правом нижнем углу


\titleformat{\section}[block]{\normalfont\Large\bfseries}{\thesection}{3em}{}


%~%~%~%~%~%~%~%~%~%~%~%~%~%~%~%~%~%~%~%~%
%          document starts here         %
%~%~%~%~%~%~%~%~%~%~%~%~%~%~%~%~%~%~%~%~%

\begin{document}

\begin{center}
\scalebox{2}{\textbf{Динамика}}
\end{center}

% 1.2.1
\section*{1.2.1 Инерциальные системы отсчета. Первый закон Ньютона}
\vspace{-9pt}
\textbf{Инерциальная система отсчета (ИСО):} Система отсчета, в которой тело, на которое не действуют силы (или действие сил скомпенсировано), находится в состоянии покоя или равномерного прямолинейного движения.
\vspace{-3pt}

\textbf{Первый закон Ньютона (закон инерции):} Существуют такие системы отсчета, называемые инерциальными, относительно которых материальная точка сохраняет состояние покоя или равномерного прямолинейного движения, если на нее не действуют другие тела или действие других тел скомпенсировано.
\vspace{-3pt}

Формула: Этот закон не имеет математической формулы, но его суть выражается в том, что если $\sum{\vec{F}} = 0$, то $a = 0$.

% 1.2.2
\section*{1.2.2 Принцип относительности Галилея}
\vspace{-9pt}
\textbf{Принцип относительности Галилея:} В любых инерциальных системах отсчета все механические явления протекают одинаково, то есть никаким механическим опытом нельзя установить, покоится ли система отсчета или движется равномерно и прямолинейно.
\vspace{-3pt}

\textbf{Трансформации Галилея:} Выражают связь координат и времени между двумя ИСО, движущимися друг относительно друга равномерно и прямолинейно.
\begin{enumerate} [itemsep=0pt, topsep=0pt, parsep=3pt]
    \item \textbf{Координаты:} $x' = x - vt$, $y' = y$, $z' = z$ (если движение вдоль оси x).
    \item \textbf{Время:} $t' = t$.
    \item $(x, y, z, t)$ - координаты и время в одной ИСО.
    \item $(x', y', z', t')$ - координаты и время в другой ИСО.
    \item $v$ - скорость движения второй ИСО относительно первой.
\end{enumerate}

% 1.2.3
\section*{1.2.3 Масса тела}
\vspace{-9pt}
\textbf{Масса ($m$):} Физическая величина, являющаяся мерой инертности тела и его способности сохранять свою скорость. Также масса является мерой гравитационных свойств тела.
\vspace{-3pt}

\textbf{Свойства массы:}
\begin{enumerate} [itemsep=0pt, topsep=0pt, parsep=3pt]
    \item Масса – скалярная величина.
    \item Масса – аддитивная величина: масса системы тел равна сумме масс отдельных тел.
    \item Масса – инвариантна: не меняется при переходе из одной ИСО в другую (в классической механике).
\end{enumerate}
\textbf{Единица измерения:} килограмм (кг).

% 1.2.4
\section*{1.2.4 Плотность вещества}
\vspace{-9pt}
\textbf{Плотность ($\rho$):} Физическая величина, равная отношению массы тела к его объему.
\vspace{-3pt}

Формула: $\rho = \frac{m}{V}$, где
\begin{enumerate} [itemsep=0pt, topsep=0pt, parsep=3pt]
    \item $\rho$ - плотность,
    \item $m$ - масса,
    \item $V$ - объем.
\end{enumerate}
Единица измерения: $\frac{кг}{м^3}$.


% new page
\newpage


% 1.2.5
\section*{1.2.5 Сила}
\vspace{-9pt}
\textbf{Сила ($F$):} Векторная физическая величина, являющаяся мерой механического воздействия на тело, в результате которого тело получает ускорение или деформируется.
\vspace{-3pt}

\textbf{Характеристики силы:}
\begin{enumerate} [itemsep=0pt, topsep=0pt, parsep=3pt]
    \item Величина (модуль).
    \item Направление.
    \item Точка приложения.
\end{enumerate}
Единица измерения: ньютон (Н).

% 1.2.6
\section*{1.2.6 Принцип суперпозиции сил}
\vspace{-9pt}
\textbf{Принцип суперпозиции сил:} Если на тело одновременно действует несколько сил, то их действие эквивалентно действию одной силы, равной векторной сумме всех действующих сил.
\vspace{-3pt}

Формула: $\sum{\vec{F_R}} =\vec{F_1} + \vec{F_2} + ... + \vec{F_n}$.

% 1.2.7
\section*{1.2.7 Второй закон Ньютона}
\vspace{-9pt}
\textbf{Второй закон Ньютона:} Ускорение, которое получает тело, прямо пропорционально равнодействующей силе, действующей на тело, и обратно пропорционально его массе.
\vspace{-3pt}

Формула: $\vec{F_R} = m\vec{a}$, где
\begin{enumerate} [itemsep=0pt, topsep=0pt, parsep=3pt]
    \item $\vec{F_R}$ – векторная равнодействующая сила,
    \item $m$ – масса тела,
    \item $a$ – вектор ускорения тела.
\end{enumerate}

В импульсной форме: $F_R\Delta{t} = \Delta{p}$, где
\begin{enumerate} [itemsep=0pt, topsep=0pt, parsep=3pt]
    \item $\Delta{p}$ - изменение импульса тела,
    \item $\Delta{t}$ - время действия силы.
\end{enumerate}

\newpage




\end{document}
