\documentclass[a4paper,12pt]{article}
\usepackage{fancyhdr}
\usepackage{graphicx} 
\usepackage{lipsum}
\usepackage{geometry}
\usepackage{mathtext}
\usepackage[T2A]{fontenc}
\usepackage[utf8]{inputenc}
\usepackage[english, russian]{babel}
\usepackage{titlesec}
\usepackage{xcolor}
\usepackage{graphicx}
\usepackage{caption}
\usepackage{subcaption}
\usepackage{wrapfig}
\usepackage{gensymb}
\usepackage{amsmath}
\usepackage{amssymb}
\usepackage{tikz}
\usepackage{circuitikz} 
\usepackage{physics}

\usepackage{enumitem}

\geometry{top=2cm, bottom=2cm, left=1cm, right=1cm}

\pagestyle{fancy}
\fancyhf{} 
\fancyfoot[C]{\rule[20pt]{\textwidth}{0.4pt}}
\fancyhead{} 
\fancyfoot[L]{Page \thepage} 
\fancyfoot[R]{\copyright\ Бот для подготовки к ЕГЭ по физике. 2025г} 

\titleformat{\section}[block]{\normalfont\Large\bfseries}{\thesection}{3em}{}
\setlist[itemize]{label=\scriptsize\textbullet}

\begin{document}

\begin{center}
\scalebox{2}{\textbf{Механические колебания и волны}}
\end{center}

\vspace{-2.5em}

% 1.5.1
\section*{1.5.1 Гармонические колебания}

\vspace{-9pt}
\subsection*{Определение:}
\vspace{-3pt}
\textbf{Гармонические колебания} - колебания, при которых физическая величина (например, смещение, скорость) изменяется со временем по закону синуса или косинуса. Это простейший вид периодических колебаний.

\vspace{-9pt}
\subsection*{Описание:}
\vspace{-3pt}
График гармонических колебаний представляет собой синусоиду.

\vspace{-9pt}
\subsection*{Уравнение гармонических колебаний (смещение):}
\vspace{-3pt}
\begin{itemize}[itemsep=0pt, topsep=0pt, parsep=0pt]
    \item $x(t) = A \cdot \cos(\omega t + \phi)$ или $x(t) = A \cdot \sin(\omega t + \phi)$, где
    \begin{itemize}
        \item $x(t)$ - смещение тела от положения равновесия в момент времени t
        \item $A$ - амплитуда колебаний
        \item $\omega$ - циклическая частота
        \item $t$ - время
        \item $\phi$ - начальная фаза
    \end{itemize}
\end{itemize}

% 1.5.2
\section*{1.5.2 Амплитуда и фаза колебаний}

\vspace{-9pt}
\subsection*{Амплитуда (A):}
\vspace{-3pt}
Максимальное отклонение колеблющегося тела от положения равновесия.
\begin{itemize}[itemsep=0pt, topsep=0pt, parsep=0pt]
    \item Характеризует "размах" колебаний
    \item Измеряется в единицах смещения (например, метрах)
\end{itemize}

\vspace{-9pt}
\subsection*{Фаза колебаний ($\omega t + \phi$):}
\vspace{-3pt}
Определяет состояние колебательной системы в конкретный момент времени.

\vspace{-9pt}
\subsection*{Начальная фаза ($\phi$):}
\vspace{-3pt}
Значение фазы в начальный момент времени (t=0).
\begin{itemize}[itemsep=0pt, topsep=0pt, parsep=0pt]
    \item Определяет положение тела в момент начала отсчета времени
    \item Измеряется в радианах
\end{itemize}

% 1.5.3
\newpage
\section*{1.5.3 Период колебаний}

\vspace{-9pt}
\subsection*{Определение:}
\vspace{-3pt}
\textbf{Период колебаний (T)} - время, за которое совершается одно полное колебание.

\vspace{-9pt}
\subsection*{Формулы:}
\vspace{-3pt}
\begin{itemize}[itemsep=0pt, topsep=0pt, parsep=0pt]
    \item $T = \frac{1}{\nu}$
    \item $T = \frac{2\pi}{\omega}$
\end{itemize}
где:
\begin{itemize}
    \item $\nu$ - частота колебаний
    \item $\omega$ - циклическая частота
\end{itemize}

\vspace{-9pt}
\subsection*{Единица измерения:}
\vspace{-3pt}
секунда (с).

% 1.5.4
\section*{1.5.4 Частота колебаний}

\vspace{-9pt}
\subsection*{Определение:}
\vspace{-3pt}
\textbf{Частота колебаний ($\nu$)} - количество полных колебаний, совершаемых в единицу времени.

\vspace{-9pt}
\subsection*{Формулы:}
\vspace{-3pt}
\begin{itemize}[itemsep=0pt, topsep=0pt, parsep=0pt]
    \item $\nu = \frac{1}{T}$
    \item $\nu = \frac{\omega}{2\pi}$
\end{itemize}
где:
\begin{itemize}
    \item $T$ - период колебаний
    \item $\omega$ - циклическая частота
\end{itemize}

\vspace{-9pt}
\subsection*{Циклическая частота:}
\vspace{-3pt}
$\omega = 2\pi\nu$

\vspace{-9pt}
\subsection*{Единица измерения:}
\vspace{-3pt}
герц (Гц). 1 Гц = 1 колебание в секунду.

% 1.5.5
\newpage
\section*{1.5.5 Свободные колебания (математический и пружинный маятники)}

\vspace{-9pt}
\subsection*{Определение:}
\vspace{-3pt}
\textbf{Свободные колебания} - колебания, происходящие под действием внутренних сил системы после однократного вывода системы из состояния равновесия.

\vspace{-9pt}
\subsection*{Математический маятник:}
\vspace{-3pt}
Идеализированная система, состоящая из материальной точки, подвешенной на невесомой нерастяжимой нити, совершающей колебания под действием силы тяжести.

\vspace{-9pt}
\subsection*{Формула периода:}
\vspace{-3pt}
$T = 2\pi \cdot \sqrt{\frac{l}{g}}$, где:
\begin{itemize}[itemsep=0pt, topsep=0pt, parsep=0pt]
    \item $l$ - длина нити
    \item $g$ - ускорение свободного падения
\end{itemize}

\vspace{-9pt}
\subsection*{Примечание:}
\vspace{-3pt}
Эта формула справедлива только для малых углов отклонения.

\vspace{-9pt}
\subsection*{Пружинный маятник:}
\vspace{-3pt}
Система, состоящая из тела, прикрепленного к пружине, которая совершает колебания под действием силы упругости.

\vspace{-9pt}
\subsection*{Формула периода:}
\vspace{-3pt}
$T = 2\pi \cdot \sqrt{\frac{m}{k}}$, где:
\begin{itemize}[itemsep=0pt, topsep=0pt, parsep=0pt]
    \item $m$ - масса тела
    \item $k$ - жесткость пружины
\end{itemize}

% 1.5.6
\section*{1.5.6 Вынужденные колебания}

\vspace{-9pt}
\subsection*{Определение:}
\vspace{-3pt}
\textbf{Вынужденные колебания} - колебания, которые происходят под действием внешней периодической силы.

\vspace{-9pt}
\subsection*{Пример:}
\vspace{-3pt}
Колебания качелей, которые кто-то постоянно подталкивает.

\vspace{-9pt}
\subsection*{Отличие от свободных:}
\vspace{-3pt}
Внешняя сила компенсирует потери энергии на трение. Частота вынужденных колебаний равна частоте внешней силы.

% 1.5.7
\newpage
\section*{1.5.7 Резонанс}

\vspace{-9pt}
\subsection*{Определение:}
\vspace{-3pt}
\textbf{Резонанс} - явление резкого возрастания амплитуды вынужденных колебаний, когда частота внешней силы близка к собственной частоте колебательной системы.

\vspace{-9pt}
\subsection*{Условие резонанса:}
\vspace{-3pt}
Частота внешней силы ($\nu_{\text{внеш}}$) $\approx$ собственной частоте ($\nu_{\text{соб}}$).

\vspace{-9pt}
\subsection*{Примеры:}
\vspace{-3pt}
Раскачивание качелей, разрушение мостов под действием ветра, работа радиоприемников.

\vspace{-9pt}
\subsection*{Опасность:}
\vspace{-3pt}
Резонанс может привести к разрушению системы, если амплитуда колебаний становится слишком большой.

% 1.5.8
\section*{1.5.8 Длина волны}

\vspace{-9pt}
\subsection*{Определение:}
\vspace{-3pt}
\textbf{Волна} - распространение колебаний в пространстве.

\vspace{-9pt}
\subsection*{Длина волны ($\lambda$):}
\vspace{-3pt}
Расстояние между двумя ближайшими точками волны, колеблющимися в одинаковой фазе (например, между двумя соседними гребнями или впадинами).

\vspace{-9pt}
\subsection*{Формулы:}
\vspace{-3pt}
\begin{itemize}[itemsep=0pt, topsep=0pt, parsep=0pt]
    \item $\lambda = v \cdot T$
    \item $\lambda = \frac{v}{\nu}$
\end{itemize}
где:
\begin{itemize}
    \item $v$ - скорость распространения волны
    \item $T$ - период колебаний
    \item $\nu$ - частота колебаний
\end{itemize}

\vspace{-9pt}
\subsection*{Единица измерения:}
\vspace{-3pt}
метр (м).

% 1.5.9
\newpage
\section*{1.5.9 Звук}

\vspace{-9pt}
\subsection*{Определение:}
\vspace{-3pt}
\textbf{Звук} - механические колебания, распространяющиеся в упругой среде (газе, жидкости, твердом теле) и воспринимаемые органами слуха.

\vspace{-9pt}
\subsection*{Характеристики звука:}
\vspace{-3pt}
\begin{itemize}[itemsep=0pt, topsep=0pt, parsep=0pt]
    \item \textbf{Частота:} Определяет высоту звука (высокие частоты - высокий звук, низкие частоты - низкий звук)
    \item \textbf{Амплитуда:} Определяет громкость звука (большая амплитуда - громкий звук, малая амплитуда - тихий звук)
    \item \textbf{Скорость:} Зависит от среды, в которой распространяется звук. В воздухе при нормальных условиях $\approx$ 340 м/с
\end{itemize}

\vspace{-9pt}
\subsection*{Звуковые волны:}
\vspace{-3pt}
Продольные волны (колебания частиц среды происходят вдоль направления распространения волны).

\end{document}