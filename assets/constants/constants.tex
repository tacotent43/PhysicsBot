\documentclass[a4paper,12pt]{article}
\usepackage{fancyhdr}
\usepackage{graphicx} 
\usepackage{lipsum}
\usepackage{geometry}
\usepackage{mathtext}
\usepackage[T2A]{fontenc}
\usepackage[utf8]{inputenc}
\usepackage[english, russian]{babel}
\usepackage{titlesec}
\usepackage{xcolor}
\usepackage{graphicx}
\usepackage{caption}
\usepackage{subcaption}
\usepackage{wrapfig}
\usepackage{gensymb}
\usepackage{amsmath}
\usepackage{amssymb}
\usepackage{tikz}
\usepackage{circuitikz} 
\usepackage{physics}

\usepackage{enumitem}

\geometry{top=2cm, bottom=2cm, left=1cm, right=1cm}

\pagestyle{fancy}
\fancyhf{} 
\fancyfoot[C]{\rule[20pt]{\textwidth}{0.4pt}}
\fancyhead{} 
\fancyfoot[L]{Page \thepage} 
\fancyfoot[R]{\copyright\ Бот для подготовки к ЕГЭ по физике. 2025г} 

\begin{document}

\begin{center}
\Large\textbf{Справочные материалы по физике}
\end{center}

\section*{Плотность веществ}

\subsection*{Твердые тела}
\begin{tabular}{@{}llll@{}}
Алюминий & 2,7  & Олово & 7,3 \\
Германий & 5,4  & Свинец & 11,3 \\
Кремний & 2,4   & Серебро & 10,5 \\
Лед & 0,9      & Сталь & 7,8 \\
Медь & 8,9     & Хром & 7,2 \\
Нихром & 8,4   & & \\
\end{tabular}

\subsection*{Жидкости}
\begin{tabular}{@{}llll@{}}
Бензин & 0,70 & Нефть & 0,80 \\
Вода & 1,0    & Ртуть & 13,60 \\
Керосин & 0,80 & Спирт & 0,79 \\
\end{tabular}

\subsection*{Газы (при нормальных условиях)}
\begin{tabular}{@{}llll@{}}
Азот & 1,25 & Воздух & 1,29 \\
Водород & 0,09 & Кислород & 1,43 \\
\end{tabular}

\section*{Тепловые свойства веществ}

\subsection*{Твердые тела}
\begin{tabular}{@{}lrrr@{}}
\toprule
Вещество & $c$, \si{\kilo\joule\per(\kg\kelvin)} & $t_{пл}$, \si{\celsius} & $\lambda$, \si{\kilo\joule\per\kg} \\
\midrule
Алюминий & 0,89 & 660 & 380 \\
Лед & 2,1 & 0 & 334 \\
Медь & 0,38 & 1083 & 214 \\
Олово & 0,23 & 232 & 59 \\
Свинец & 0,13 & 327 & 23 \\
Серебро & 0,23 & 961 & 87 \\
Сталь & 0,46 & 1400 & 82 \\
\bottomrule
\end{tabular}

\subsection*{Жидкости}
\begin{tabular}{@{}lrrr@{}}
\toprule
Вещество & $c$, \si{\kilo\joule\per(\kg\kelvin)} & $t_{кип}$, \si{\celsius} & $L$, \si{\mega\joule\per\kg} \\
\midrule
Вода & 4,19 & 100 & 2,3 \\
Ртуть & 0,14 & 357 & 0,29 \\
Спирт & 2,4 & 78 & 0,85 \\
\bottomrule
\end{tabular}

\subsection*{Газы}
\begin{tabular}{@{}lrr@{}}
\toprule
Вещество & $c_p$, \si{\kilo\joule\per(\kg\kelvin)} & $t_{конд}$, \si{\celsius} \\
\midrule
Азот & 1,05 & -196 \\
Водород & 14,3 & -253 \\
Воздух & 1,01 & — \\
Гелий & 5,29 & -269 \\
Кислород & 0,913 & -183 \\
\bottomrule
\end{tabular}

\section*{Коэффициент поверхностного натяжения}
\begin{tabular}{@{}llll@{}}
\toprule
Вода & 73 & Молоко & 46 \\
Бензин & 21 & Нефть & 30 \\
Керосин & 24 & Ртуть & 510 \\
Мыльный раствор & 40 & Спирт & 22 \\
\bottomrule
\end{tabular}

\section*{Удельная теплота сгорания топлива}
\begin{tabular}{@{}llll@{}}
\toprule
Бензин & 44 & Порох & 3,8 \\
Дерево & 10 & Спирт & 29 \\
Дизельное топливо & 42 & Авиатопливо & 43 \\
Каменный уголь & 29 & Условное топливо & 29 \\
Керосин & 46 & & \\
\bottomrule
\end{tabular}

\section*{Давление и плотность насыщенного водяного пара}
\begin{tabular}{@{}rrr|rrr@{}}
\toprule
$t$, \si{\celsius} & $p$, \si{\kilo\pascal} & $\rho$, \si{\gram\per\cubic\meter} & 
$t$, \si{\celsius} & $p$, \si{\kilo\pascal} & $\rho$, \si{\gram\per\cubic\meter} \\
\midrule
-5 & 0,40 & 3,2 & 11 & 1,33 & 10,0 \\
0 & 0,61 & 4,8 & 12 & 1,40 & 10,7 \\
1 & 0,65 & 5,2 & 13 & 1,49 & 11,4 \\
2 & 0,71 & 5,6 & 14 & 1,60 & 12,1 \\
3 & 0,76 & 6,0 & 15 & 1,71 & 12,8 \\
4 & 0,81 & 6,4 & 16 & 1,81 & 13,6 \\
5 & 0,88 & 6,8 & 17 & 1,93 & 14,5 \\
6 & 0,93 & 7,3 & 18 & 2,07 & 15,4 \\
7 & 1,0 & 7,8 & 19 & 2,20 & 16,3 \\
8 & 1,06 & 8,3 & 20 & 2,33 & 17,3 \\
9 & 1,14 & 8,8 & 25 & 3,17 & 23,0 \\
10 & 1,23 & 9,4 & 50 & 12,3 & 83,0 \\
\bottomrule
\end{tabular}

\section*{Удельное сопротивление и температурный коэффициент}
\begin{tabular}{@{}lrr@{}}
\toprule
Вещество & $\rho$, \SI{e-8}{\ohm\meter} & $\alpha$, \si{\per\kelvin} \\
\midrule
Алюминий & 2,8 & 0,0042 \\
Вольфрам & 5,5 & 0,0048 \\
Латунь & 7,1 & 0,001 \\
Медь & 1,7 & 0,0043 \\
Никелин & 42 & 0,0001 \\
Нихром & 110 & 0,0001 \\
Свинец & 21 & 0,0037 \\
Серебро & 1,6 & 0,004 \\
Сталь & 12 & 0,006 \\
Константан & 50 & 0,00003 \\
\bottomrule
\end{tabular}

\section*{Электрохимические эквиваленты}
\begin{tabular}{@{}llll@{}}
\toprule
Алюминий (Al\textsuperscript{3+}) & 0,093 & Никель (Ni\textsuperscript{2+}) & 0,30 \\
Водород (H\textsuperscript{+}) & 0,0104 & Серебро (Ag\textsuperscript{+}) & 1,12 \\
Кислород (O\textsuperscript{2-}) & 0,083 & Хром (Cr\textsuperscript{3+}) & 0,18 \\
Медь (Cu\textsuperscript{2+}) & 0,33 & Цинк (Zn\textsuperscript{2+}) & 0,34 \\
Олово (Sn\textsuperscript{2+}) & 0,62 & & \\
\bottomrule
\end{tabular}

\section*{Работа выхода электронов}
\begin{tabular}{@{}lrrlrr@{}}
\toprule
Вещество & \si{\electronvolt} & \si{\atto\joule} & Вещество & \si{\electronvolt} & \si{\atto\joule} \\
\midrule
Вольфрам & 4,5 & 0,72 & Платина & 5,3 & 0,85 \\
Калий & 2,2 & 0,35 & Серебро & 4,3 & 0,69 \\
Литий & 2,4 & 0,38 & Цезий & 1,8 & 0,29 \\
Оксид бария & 1,0 & 0,16 & Цинк & 4,2 & 0,67 \\
\bottomrule
\end{tabular}

\section*{Показатель преломления}
\begin{tabular}{@{}llll@{}}
\toprule
Алмаз & 2,42 & Сероуглерод & 1,63 \\
Вода & 1,33 & Спирт этиловый & 1,36 \\
Воздух & 1,00029 & Стекло & 1,60 \\
\bottomrule
\end{tabular}

\section*{Физические постоянные}

\subsection*{Основные константы}
\begin{tabular}{@{}ll@{}}
\toprule
Элементарный заряд & $e = \num{1,60219e-19}$ Кл \\
Масса электрона & $m_e = \num{9,1095e-31}$ кг \\
Масса протона & $m_p = \num{1,6726e-27}$ кг \\
Масса нейтрона & $m_n = \num{1,6749e-27}$ кг \\
Скорость света & $c = \num{2,9979e8}$ м/с \\
Гравитационная постоянная & $G = \num{6,672e-11}$ Н·м\textsuperscript{2}/кг\textsuperscript{2} \\
Электрическая постоянная & $\varepsilon_0 = \num{8,854e-12}$ Ф/м \\
Постоянная Авогадро & $N_A = \num{6,022e23}$ моль\textsuperscript{-1} \\
Постоянная Больцмана & $k = \num{1,3807e-23}$ Дж/К \\
Постоянная Планка & $h = \num{6,626e-34}$ Дж·с \\
\bottomrule
\end{tabular}

\subsection*{Производные константы}
\begin{tabular}{@{}ll@{}}
\toprule
Энергия покоя электрона & $m_ec^2 = 0,511$ МэВ \\
Энергия покоя протона & $m_pc^2 = 938,26$ МэВ \\
Энергия покоя нейтрона & $m_nc^2 = 939,55$ МэВ \\
Отношение $e/m_e$ & \num{1,759e11} Кл/кг \\
Постоянная Фарадея & $F = eN_A = \num{9,648e4}$ Кл/моль \\
Молярная газовая постоянная & $R = kN_A = 8,314$ Дж/(моль·К) \\
\bottomrule
\end{tabular}

\section*{Приставки СИ}
\begin{tabular}{@{}lll|lll@{}}
\toprule
\multicolumn{3}{c|}{Кратные} & \multicolumn{3}{c}{Дольные} \\
Приставка & Обозначение & Множитель & Приставка & Обозначение & Множитель \\
\midrule
экса & Э & $10^{18}$ & атто & а & $10^{-18}$ \\
пета & П & $10^{15}$ & фемто & ф & $10^{-15}$ \\
тера & Т & $10^{12}$ & пико & п & $10^{-12}$ \\
гига & Г & $10^9$ & нано & н & $10^{-9}$ \\
мега & М & $10^6$ & микро & мк & $10^{-6}$ \\
кило & к & $10^3$ & милли & м & $10^{-3}$ \\
гекто & г & $10^2$ & санти & с & $10^{-2}$ \\
дека & да & $10^1$ & деци & д & $10^{-1}$ \\
\bottomrule
\end{tabular}

\end{document}

\begin{document}

\section**{ПРИЛОЖЕНИЯ}

\subsection**{1. Плотность веществ}

\textbf{Т в е р д ы е т е л а} \\
$10^3 \, \text{кг/м}^3$ \\
Алюминий ...... 2,7 \quad Олово ...... 7,3 \\
Германий ...... 5,4 \quad Свинец ...... 11,3 \\
Кремний ...... 2,4 \quad Серебро ...... 10,5 \\
Лед ...... 0,9 \quad Сталь ...... 7,8 \\
Медь ...... 8,9 \quad Хром ...... 7,2 \\
Нихром ...... 8,4 \\

\textbf{Ж и д к о с т и} \\
$10^3 \, \text{кг/м}^3$ \\
Бензин ...... 0,70 \quad Нефть ...... 0,80 \\
Вода ...... 1,0 \quad Ртуть ...... 13,60 \\
Керосин ...... 0,80 \quad Спирт ...... 0,79 \\

\textbf{Г а з ы} \\
(при нормальных условиях) \\
$\text{кг/м}^3$ \\
Азот ...... 1,25 \quad Воздух ...... 1,29 \\
Водород ...... 0,09 \quad Кислород ...... 1,43 \\

\subsection**{2. Тепловые свойства веществ}
\textbf{Т в е р д ы е т е л а}

\begin{tabular}{|l|c|c|c|}
\hline
\textbf{Вещество} & \textbf{Удельная теплоемкость, кДж/(кг·К)} & \textbf{Температура плавления, °С} & \textbf{Удельная теплота плавления, кДж/кг} \\
\hline
Алюминий & 0,89 & 660 & 380 \\
Лед & 2,1 & 0 & 334 \\
Медь & 0,38 & 1083 & 214 \\
Олово & 0,23 & 232 & 59 \\
Свинец & 0,13 & 327 & 23 \\
Серебро & 0,23 & 961 & 87 \\
Сталь & 0,46 & 1400 & 82 \\
\hline
\end{tabular}

\textbf{Ж и д к о с т и}

\begin{tabular}{|l|c|c|c|}
\hline
\textbf{Вещество} & \textbf{Удельная теплоёмкость, кДж/(кг·К)} & \textbf{Температура кипения, °С} & \textbf{Удельная теплота парообразования$^1$, МДж/кг} \\
\hline
Вода & 4,19 & 100 & 2,3 \\
Ртуть & 0,14 & 357 & 0,29 \\
Спирт & 2,4 & 78 & 0,85 \\
\hline
\end{tabular}

$^1$ При нормальном давлении.

\subsection**{Газы}

\begin{tabular}{|l|c|c|}
\hline
\textbf{Вещество} & \textbf{Удельная теплоёмкость$^1$, кДж/(кг·К)} & \textbf{Температура конденсации$^2$, °С} \\
\hline
Азот & 1,05 & -196 \\
Водород & 14,3 & -253 \\
Воздух & 1,01 & — \\
Гелий & 5,29 & -269 \\
Кислород & 0,913 & -183 \\
\hline
\end{tabular}

\subsection**{3. Коэффициент поверхностного натяжения жидкостей, мН/м (При 20°С)}

\begin{tabular}{|l|c|l|c|}
\hline
Вода & 73 & Молоко & 46 \\
Бензин & 21 & Нефть & 30 \\
Керосин & 24 & Ртуть & 510 \\
Мыльный раствор & 40 & Спирт & 22 \\
\hline
\end{tabular}

\subsection**{4. Удельная теплота сгорания топлива, МДж/кг}

\begin{tabular}{|l|c|l|c|}
\hline
Бензин & 44 & Порох & 3,8 \\
Дерево & 10 & Спирт & 29 \\
Дизельное топливо & 42 & Топливо для реактивных самолётов & 43 \\
Каменный уголь & 29 & Условное топливо & 29 \\
Керосин & 46 & & \\
\hline
\end{tabular}

\subsection**{5. Зависимость давления $ p $ и плотности $ \rho $ насыщенного водяного пара от температуры}

\begin{tabular}{|c|c|c|c|c|c|}
\hline
$ t, °С $ & $ p $, кПа & $ \rho $, г/м$^3$ & $ t, °С $ & $ p $, кПа & $ \rho $, г/м$^3$ \\
\hline
-5 & 0,40 & 3,2 & 11 & 1,33 & 10,0 \\
0 & 0,61 & 4,8 & 12 & 1,40 & 10,7 \\
1 & 0,65 & 5,2 & 13 & 1,49 & 11,4 \\
2 & 0,71 & 5,6 & 14 & 1,60 & 12,1 \\
3 & 0,76 & 6,0 & 15 & 1,71 & 12,8 \\
4 & 0,81 & 6,4 & 16 & 1,81 & 13,6 \\
5 & 0,88 & 6,8 & 17 & 1,93 & 14,5 \\
6 & 0,93 & 7,3 & 18 & 2,07 & 15,4 \\
7 & 1,0 & 7,8 & 19 & 2,20 & 16,3 \\
8 & 1,06 & 8,3 & 20 & 2,33 & 17,3 \\
9 & 1,14 & 8,8 & 25 & 3,17 & 23,0 \\
10 & 1,23 & 9,4 & 50 & 12,3 & 83,0 \\
\hline
\end{tabular}

$^1$ При постоянном давлении. \\
$^2$ При нормальном давлении.

\subsection**{9. Удельное сопротивление $\rho$ (при 20 °C) и температурный коэффициент сопротивления $\alpha$ металлов и сплавов}

\begin{tabular}{|l|c|c|}
\hline
\textbf{Вещество} & $\rho$, $\times10^{-8}$ Ом $\cdot$ м или $\times10^{-2}$ Ом $\cdot$ мм$^2$/м & $\alpha$, К$^{-1}$ \\
\hline
Алюминий & 2,8 & 0,0042 \\
Вольфрам & 5,5 & 0,0048 \\
Латунь & 7,1 & 0,001 \\
Медь & 1,7 & 0,0043 \\
Никелин & 42 & 0,0001 \\
Нихром & 110 & 0,0001 \\
Свинец & 21 & 0,0037 \\
Серебро & 1,6 & 0,004 \\
Сталь & 12 & 0,006 \\
Константан & 50 & 0,00003 \\
\hline
\end{tabular}

\subsection**{10. Электрохимические эквиваленты, мг/Кл ($10^{-6}$ кг/Кл)}

\begin{tabular}{|l|c|l|c|}
\hline
Алюминий (А$\ell^{3+}$) & 0,093 & Никель (Ni$^{2+}$) & 0,30 \\
Водород (Н$^+$) & 0,0104 & Серебро (Ag$^+$) & 1,12 \\
Кислород (О$^{2-}$) & 0,083 & Хром (Cr$^{3+}$) & 0,18 \\
Медь (Cu$^{2+}$) & 0,33 & Цинк (Zn$^{2+}$) & 0,34 \\
Олово (Sn$^{2+}$) & 0,62 & & \\
\hline
\end{tabular}

\subsection**{11. Работа выхода электронов}

\begin{tabular}{|l|c|c|l|c|c|}
\hline
\textbf{Вещество} & \textbf{эВ} & \textbf{аДж} & \textbf{Вещество} & \textbf{эВ} & \textbf{аДж} \\
\hline
Вольфрам & 4,5 & 0,72 & Платина & 5,3 & 0,85 \\
Калий & 2,2 & 0,35 & Серебро & 4,3 & 0,69 \\
Литий & 2,4 & 0,38 & Цезий & 1,8 & 0,29 \\
Оксид бария & 1,0 & 0,16 & Цинк & 4,2 & 0,67 \\
\hline
\end{tabular}

\subsection**{12. Показатель преломления (средний для видимых лучей)}

\begin{tabular}{|l|c|l|c|}
\hline
Алмаз & 2,42 & Сероуглерод & 1,63 \\
Вода & 1,33 & Спирт этиловый & 1,36 \\
Воздух & 1,00029 & Стекло & 1,60 \\
\hline
\end{tabular}

\subsection**{15. Физические постоянные}
\textbf{О с н о в н ы е к о н с т а н т ы}

\begin{itemize}
\item элементарный заряд — $ e = 1,60219 \cdot 10^{-19} \, \text{Кл} $
\item Масса электрона — $ m_e = 9,1095 \cdot 10^{-31} \, \text{кг} = 5,486 \cdot 10^{-4} \, \text{a. е. м.} $
\item Масса протона — $ m_p = 1,6726 \cdot 10^{-27} \, \text{кг} = 1,00728 \, \text{a. е. м.} $
\item Масса нейтрона — $ m_n = 1,6749 \cdot 10^{-27} \, \text{кг} = 1,00867 \, \text{a. е. м.} $
\item Скорость света в вакууме — $ c = 2,9979 \cdot 10^8 \, \text{м/с} $
\item Гравитационная постоянная — $ G = 6,672 \cdot 10^{-11} \, \text{H} \cdot \text{m}^2/\text{кг}^2 $
\item Электрическая постоянная — $ \varepsilon_0 = 8,854 \cdot 10^{-12} \, \text{Ф/м} $
\item Постоянная Авогадро — $ N_A = 6,022 \cdot 10^{23} \, \text{моль}^{-1} $
\item Постоянная Больцмана — $ k = 1,3807 \cdot 10^{-23} \, \text{Дж/К} $
\item Постоянная Планка — $ h = 6,626 \cdot 10^{-34} \, \text{Дж} \cdot \text{c} = 4,136 \cdot 10^{-15} \, \text{эВ} \cdot \text{c}, $ 
\[ \hbar = \frac{h}{2\pi} = 1,055 \cdot 10^{-34} \, \text{Дж} \cdot \text{c} = 6,59 \cdot 10^{-16} \, \text{эВ} \cdot \text{c} \]
\end{itemize}

\textbf{П р о и з в о д н ы е о т о с н о в н ы х к о н с т а н т}

\begin{itemize}
\item Коэффициент взаимосвязи массы и энергии — 
\[ c^2 = \frac{E}{m} = 8,9874 \cdot 10^{16} \, \text{Дж/кг} = 931,5 \, \text{МэВ/a. е. м.} \]
(1 a. е. м. = 1,66057 \cdot 10^{-27} \, \text{кг}; 1 \, \text{МэВ} = 1,60219 \cdot 10^{-13} \, \text{Дж})
\item Энергия покоя электрона — $ E_{0e} = m_e c^2 = 8,187 \cdot 10^{-14} \, \text{Дж} = 0,511 \, \text{МэВ} $
\item Энергия покоя протона — $ E_{0p} = m_p c^2 = 1,503 \cdot 10^{-10} \, \text{Дж} = 938,26 \, \text{МэВ} $
\item Энергия покоя нейтрона — $ E_{0n} = m_n c^2 = 1,505 \cdot 10^{-10} \, \text{Дж} = 939,55 \, \text{МэВ} $
\item Отношение заряда электрона к его массе — $ \frac{e}{m_e} = 1,759 \cdot 10^{11} \, \text{Кл/кг} $
\item Постоянная Фарадея — $ F = eN_A = 9,648 \cdot 10^4 \, \text{Кл/моль} $
\item Молярная газовая постоянная — $ R = kN_A = 8,314 \, \text{Дж/(моль} \cdot \text{K)} $
\end{itemize}

\subsection**{16. Приставки для образования десятичных кратных и дольных единиц}

\begin{tabular}{|l|l|c|l|l|c|}
\hline
\multicolumn{3}{|c|}{\textbf{Кратные}} & \multicolumn{3}{c|}{\textbf{Дольные}} \\
\hline
приставка & обозначение & множитель & приставка & обозначение & множитель \\
\hline
экса & Э & $10^{18}$ & атто & а & $10^{-18}$ \\
пета & П & $10^{15}$ & фемто & ф & $10^{-15}$ \\
тера & Т & $10^{12}$ & пико & п & $10^{-12}$ \\
гига & Г & $10^9$ & нано & н & $10^{-9}$ \\
мега & М & $10^6$ & микро & мк & $10^{-6}$ \\
кило & к & $10^3$ & милли & м & $10^{-3}$ \\
гекто & г & $10^2$ & санти & с & $10^{-2}$ \\
дека & да & $10^1$ & деци & д & $10^{-1}$ \\
\hline
\end{tabular}

\end{document}