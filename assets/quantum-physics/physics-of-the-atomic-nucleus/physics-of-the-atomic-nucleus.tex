\documentclass[a4paper,12pt]{article}
\usepackage{fancyhdr}
\usepackage{graphicx} 
\usepackage{lipsum}
\usepackage{geometry}
\usepackage{mathtext}
\usepackage[T2A]{fontenc}
\usepackage[utf8]{inputenc}
\usepackage[english, russian]{babel}
\usepackage{titlesec}
\usepackage{xcolor}
\usepackage{graphicx}
\usepackage{caption}
\usepackage{subcaption}
\usepackage{wrapfig}
\usepackage{gensymb}
\usepackage{amsmath}
\usepackage{amssymb}
\usepackage{tikz}
\usepackage{circuitikz} 
\usepackage{physics}

\usepackage{enumitem}

\geometry{top=2cm, bottom=2cm, left=1cm, right=1cm}

\pagestyle{fancy}
\fancyhf{} 
\fancyfoot[C]{\rule[20pt]{\textwidth}{0.4pt}}
\fancyhead{} 
\fancyfoot[L]{Page \thepage} 
\fancyfoot[R]{\copyright\ Бот для подготовки к ЕГЭ по физике. 2025г} 

\titleformat{\section}[block]{\normalfont\Large\bfseries}{\thesection}{3em}{}
\setlist[itemize]{label=\scriptsize\textbullet}

\begin{document}

\begin{center}
    \scalebox{2}{\textbf{Физика атомного ядра}}
\end{center}

\vspace{-2em}

%%%%%%%%%%%%%%%%%
% \section*{}


% \vspace{-9pt}
% \subsection*{}
% \vspace{-3pt}


% \begin{enumerate} [itemsep=0pt, topsep=0pt, parsep=0pt]
%   \item $p$ – импульс тела;
%   \item $m$ – масса тела;
%   \item $v$ – скорость тела.
% \end{enumerate}


% [itemsep=0pt, topsep=0pt, parsep=0pt]
%%%%%%%%%%%%%%%%%


\section*{5.3.1 Радиоактивность. Альфа-распад. Бетта-распад. Гамма-излучение}
\vspace{-9pt}
\subsection*{Определение:}
\vspace{-3pt}
\begin{itemize}[itemsep=0pt, topsep=0pt, parsep=3pt]
    \item Радиоактивность: Явление самопроизвольного распада атомных ядер.
    \item Альфа-распад: Испускание ядром альфа-частицы (ядро гелия).
    \item Бетта-распад: Испускание ядром электрона или позитрона.
    \item Гамма-излучение: Испускание ядром фотонов высокой энергии.
\end{itemize}

\vspace{-9pt}
\subsection*{Виды излучения:}
\vspace{-3pt}
\begin{itemize}[itemsep=0pt, topsep=0pt, parsep=3pt]
    \item Альфа-частицы: положительно заряженные ядра гелия.
    \item Бетта-частицы: электроны или позитроны.
    \item Гамма-кванты: фотоны с высокой энергией.
\end{itemize}

\section*{5.3.2 Закон радиоактивного распада}
\vspace{-9pt}
\subsection*{Формула:}
\vspace{-3pt}
\vspace{-0.05em}
$$ N(t) = N_0 \cdot e^{-\lambda t} $$где:
\begin{itemize}[itemsep=0pt, topsep=0pt, parsep=3pt]
        \item $N(t)$ – число нераспавшихся ядер в момент времени t,
        \item $N_0$ – начальное число ядер,
        \item $\lambda$ – постоянная распада,
        \item $t$ - время распада.
\end{itemize}


\vspace{-9pt}
\subsection*{Период полураспада:}
\vspace{-3pt}
\begin{itemize}[itemsep=0pt, topsep=0pt, parsep=3pt]
    \item Период полураспада ($T_{\frac{1}{2}}$): время, за которое распадается половина начального числа ядер.
    \vspace{-0.05em}
    $$ T_{\frac{1}{2}}= \frac{ln2}{\lambda} $$
\end{itemize}

\section*{5.3.3 Нуклонная модель ядра. Заряд ядра. Массовое число ядра}
\vspace{-9pt}
\subsection*{Определение:}
\vspace{-3pt}
\begin{itemize}[itemsep=0pt, topsep=0pt, parsep=3pt]
    \item Нуклонная модель: Ядро состоит из протонов и нейтронов (нуклонов).
    \item Протоны: Положительно заряженные частицы (заряд +e). Число протонов определяет заряд ядра и порядковый номер элемента в таблице Менделеева.
    \item Нейтроны: Нейтральные частицы.
    \item Заряд ядра ($Z$): Число протонов в ядре.
    \item Массовое число ядра ($A$): Общее число протонов и нейтронов в ядре.
\end{itemize}

\section*{5.3.4 Энергия связи нуклонов в ядре. Ядерные силы}
\vspace{-9pt}
\subsection*{Определение:}
\vspace{-3pt}
\begin{itemize}[itemsep=0pt, topsep=0pt, parsep=3pt]
    \item Энергия связи: Энергия, необходимая для полного расщепления ядра на отдельные нуклоны.
    \item Дефект массы ($\Delta m$): Разность между суммой масс свободных нуклонов и массой ядра.
    \item Формула:
     \vspace{-0.05em}
    $$ E_{св} = \Delta mc^2 $$
    \item Ядерные силы: Сильные взаимодействия, удерживающие нуклоны в ядре.
\end{itemize}

\section*{5.3.5 Ядерные реакции. Деление и синтез ядер}
\vspace{-9pt}
\subsection*{Определение:}
\vspace{-3pt}
\begin{itemize}[itemsep=0pt, topsep=0pt, parsep=3pt]
    \item Ядерные реакции: Преобразование атомных ядер при взаимодействии с другими частицами или ядрами.
    \item Деление ядер: Распад тяжелого ядра на два или более легких ядра с выделением энергии (используется в ядерных реакторах).
    \item Синтез ядер: Слияние легких ядер в более тяжелые ядра с выделением энергии (происходит в звездах и термоядерных реакторах).
\end{itemize}

\end{document}