\documentclass[a4paper,12pt]{article}
\usepackage{fancyhdr}
\usepackage{graphicx} 
\usepackage{lipsum}
\usepackage{geometry}
\usepackage{mathtext}
\usepackage[T2A]{fontenc}
\usepackage[utf8]{inputenc}
\usepackage[english, russian]{babel}
\usepackage{titlesec}
\usepackage{xcolor}
\usepackage{graphicx}
\usepackage{caption}
\usepackage{subcaption}
\usepackage{wrapfig}
\usepackage{gensymb}
\usepackage{amsmath}
\usepackage{amssymb}
\usepackage{tikz}
\usepackage{circuitikz} 
\usepackage{physics}

\usepackage{enumitem}

\geometry{top=2cm, bottom=2cm, left=1cm, right=1cm}

\pagestyle{fancy}
\fancyhf{} 
\fancyfoot[C]{\rule[20pt]{\textwidth}{0.4pt}}
\fancyhead{} 
\fancyfoot[L]{Page \thepage} 
\fancyfoot[R]{\copyright\ Бот для подготовки к ЕГЭ по физике. 2025г} 

\titleformat{\section}[block]{\normalfont\Large\bfseries}{\thesection}{3em}{}
\setlist[itemize]{label=\scriptsize\textbullet}


\begin{document}

\begin{center}
\scalebox{2}{\textbf{Физика атома}}
\end{center}

\vspace{-2.5em}

%%%%%%%%%%%%%%%%%
% \section*{}


% \vspace{-9pt}
% \subsection*{}
% \vspace{-3pt}


% \begin{enumerate} [itemsep=0pt, topsep=0pt, parsep=0pt]
%   \item $p$ – импульс тела;
%   \item $m$ – масса тела;
%   \item $v$ – скорость тела.
% \end{enumerate}


% [itemsep=0pt, topsep=0pt, parsep=0pt]
%%%%%%%%%%%%%%%%%


\section*{2.1 Планетарная модель атома}
\vspace{-9pt}
\subsection*{Резерфорд:}
\vspace{-3pt}
Атом состоит из положительно заряженного ядра, в котором сосредоточена основная масса атома, и вращающихся вокруг него электронов (по аналогии с планетами вокруг Солнца).

\vspace{-9pt}
\subsection*{Несостоятельность:}
\vspace{-3pt}
Модель не объясняет устойчивость атома (электроны, вращаясь, должны были бы излучать и падать на ядро) и линейчатые спектры излучения.


\section*{2.2 Постулаты Бора}
\vspace{-9pt}
\subsection*{Постулаты:}
\vspace{-3pt}
\begin{enumerate}[itemsep=0pt, topsep=0pt, parsep=3pt]
    \item Атом может находиться только в определенных (стационарных) состояниях, которым соответствуют определенные значения энергии.
    \item Излучение и поглощение света происходят при переходах электронов между стационарными состояниями.
    \item \textbf{Правило частот Бора:}
    \vspace{-0.05em}
    $$ h\nu = |E_n - E_m| $$
    где $E_n$ и $E_m$ — энергии электрона на разных орбитах.
\end{enumerate}


\section*{2.3 Линейчатые спектры}
\vspace{-9pt}
\subsection*{Линейчатые спектры:}
\vspace{-3pt}
Спектры излучения и поглощения, состоящие из отдельных линий (характерных для каждого элемента).

\section*{2.4 Лазер}
\vspace{-9pt}
\subsection*{Определение:}
\vspace{-3pt}
Устройство, генерирующее когерентное (с одинаковой фазой), монохроматическое (одной длины волны) и узконаправленное излучение.

\vspace{-9pt}
\subsection*{Принцип работы:}
\vspace{-3pt}
Основан на вынужденном излучении (атомы, находясь в возбужденном состоянии, испускают фотоны под действием других фотонов).

\vspace{-9pt}
\subsection*{Применение:}
\vspace{-3pt}
\begin{itemize}[itemsep=0pt, topsep=0pt, parsep=3pt]
    \item Медицина
    \item Связь
    \item Промышленность
    \item Наука
\end{itemize}


\end{document}\
