\documentclass[a4paper,12pt]{article}
\usepackage{fancyhdr}
\usepackage{graphicx} 
\usepackage{lipsum}
\usepackage{geometry}
\usepackage{mathtext}
\usepackage[T2A]{fontenc}
\usepackage[utf8]{inputenc}
\usepackage[english, russian]{babel}
\usepackage{titlesec}
\usepackage{xcolor}
\usepackage{graphicx}
\usepackage{caption}
\usepackage{subcaption}
\usepackage{wrapfig}
\usepackage{gensymb}
\usepackage{amsmath}
\usepackage{amssymb}
\usepackage{tikz}
\usepackage{circuitikz} 
\usepackage{physics}

\usepackage{enumitem}

\geometry{top=2cm, bottom=2cm, left=1cm, right=1cm}

\pagestyle{fancy}
\fancyhf{} 
\fancyfoot[C]{\rule[20pt]{\textwidth}{0.4pt}}
\fancyhead{} 
\fancyfoot[L]{Page \thepage} 
\fancyfoot[R]{\copyright\ Бот для подготовки к ЕГЭ по физике. 2025г} 

\titleformat{\section}[block]{\normalfont\Large\bfseries}{\thesection}{3em}{}
\setlist[itemize]{label=\scriptsize\textbullet}

\begin{document}

\begin{center}
\scalebox{2}{\textbf{Корпускулярно-волновой дуализм}}
\end{center}

\vspace{-2em}

%%%%%%%%%%%%%%%%%
% \section*{}


% \vspace{-9pt}
% \subsection*{}
% \vspace{-3pt}


% \begin{enumerate} [itemsep=0pt, topsep=0pt, parsep=0pt]
%   \item $p$ – импульс тела;
%   \item $m$ – масса тела;
%   \item $v$ – скорость тела.
% \end{enumerate}


% [itemsep=0pt, topsep=0pt, parsep=0pt]
%%%%%%%%%%%%%%%%%

\vspace{-9pt}
\section*{1.1 Гипотеза М. Планка о квантах}
\vspace{-3pt}
\begin{itemize}[itemsep=0pt, topsep=0pt, parsep=3pt]
    \item \textbf{Гипотеза:} Энергия электромагнитного излучения поглощается и испускается порциями (квантами).
    \item \textbf{Квант энергии ($E$):}
    \vspace{-0.05em}
    $$ E = h \cdot \nu $$
    где:
    \begin{itemize}[itemsep=0pt, topsep=0pt, parsep=3pt]
        \item $h$ — постоянная Планка (приблизительно $6.626 \cdot 10^{-34}$ Дж$\cdot$с).
        \item $\nu$ — частота излучения.
    \end{itemize}
\end{itemize}


\section*{1.2 Фотоэффект}
\vspace{-9pt}
\subsection*{Фотоэффект:}
\vspace{-3pt}
Явление испускания электронов веществом под действием света.

\vspace{-9pt}
\subsection*{Виды:}
\vspace{-3pt}
\begin{itemize}[itemsep=0pt, topsep=0pt, parsep=3pt]
    \item Внешний (электроны покидают поверхность металла).
    \item Внутренний (электроны возбуждаются внутри вещества).
    \item Вентильный (возникновение тока при освещении полупроводниковых контактов).
\end{itemize}

\vspace{-9pt}
\subsection*{Красная граница фотоэффекта:}
\vspace{-3pt}
Минимальная частота или максимальная длина волны, при которых еще возможен фотоэффект.


\section*{1.3 Опыты А.Г. Столетова}
\vspace{-9pt}
\subsection*{Установка:}
\vspace{-3pt}
Опыты по изучению фотоэффекта с применением вакуумного фотоэлемента.

\vspace{-9pt}
\subsection*{Результаты:}
\vspace{-3pt}
\begin{itemize}[itemsep=0pt, topsep=0pt, parsep=3pt]
    \item Сила фототока прямо пропорциональна интенсивности света.
    \item Максимальная кинетическая энергия фотоэлектронов зависит от частоты света и не зависит от интенсивности.
    \item Существует красная граница фотоэффекта.
\end{itemize}

\newpage
\section*{1.4 Уравнение Эйнштейна для фотоэффекта}
\vspace{-9pt}
\subsection*{Формула:}
\vspace{-3pt}

\vspace{-0.05em}
$$ h\nu = A_{\text{вых}} + E_{k_{\text{макс}}} $$
где:
\begin{itemize}[itemsep=0pt, topsep=0pt, parsep=3pt]
    \item $h\nu$ — энергия фотона.
    \item $A_{\text{вых}}$ — работа выхода электрона из металла.
    \item $E_{k_{\text{макс}}}$ — максимальная кинетическая энергия фотоэлектрона.
\end{itemize}

\vspace{-9pt}
\subsection*{Пояснение:}
\vspace{-3pt}
Уравнение устанавливает связь между энергией фотона, работой выхода и кинетической энергией фотоэлектронов.


\section*{1.5 Фотоны}
\vspace{-9pt}
\subsection*{Фотон:}
\vspace{-3pt}
Квант электромагнитного излучения.

\vspace{-9pt}
\subsection*{Свойства:}
\vspace{-3pt}
\begin{itemize}[itemsep=0pt, topsep=0pt, parsep=3pt]
    \item Движется со скоростью света.
    \item Обладает энергией и импульсом.
    \item Электрически нейтрален.
\end{itemize}


\section*{1.6 Энергия фотона}
\vspace{-9pt}
\subsection*{Формула:}
\vspace{-3pt}
\vspace{-0.05em}
$$ E_{\phi} = h\nu = \frac{hc}{\lambda} $$
где:
\begin{itemize}[itemsep=0pt, topsep=0pt, parsep=3pt]
    \item $E_{\phi}$ — энергия фотона.
    \item $h$ — постоянная Планка.
    \item $\nu$ — частота излучения.
    \item $c$ — скорость света.
    \item $\lambda$ — длина волны излучения.
\end{itemize}

\section*{1.7 Импульс фотона}
\vspace{-9pt}
\subsection*{Формула:}
\vspace{-3pt}
\vspace{-0.05em}
$$ p_{\phi} = \frac{E_{\phi}}{c} = \frac{h}{\lambda} $$
где:
\begin{itemize}[itemsep=0pt, topsep=0pt, parsep=3pt]
    \item $p_{\phi}$ — импульс фотона.
    \item $E_{\phi}$ — энергия фотона.
    \item $c$ — скорость света.
    \item $h$ — постоянная Планка.
    \item $\lambda$ — длина волны излучения.
\end{itemize}

\newpage
\section*{1.8 Гипотеза де Бройля о волновых свойствах частиц. Корпускулярно-волновой дуализм}
\vspace{-9pt}
\subsection*{Гипотеза:}
\vspace{-3pt}
Любая частица материи обладает волновыми свойствами, а любой волне соответствуют корпускулярные свойства.

\vspace{-9pt}
\subsection*{Длина волны де Бройля:}
\vspace{-3pt}
\vspace{-0.05em}
$$ \lambda = \frac{h}{p} $$
где:
\begin{itemize}[itemsep=0pt, topsep=0pt, parsep=3pt]
    \item $\lambda$ — длина волны де Бройля.
    \item $h$ — постоянная Планка.
    \item $p$ — импульс частицы.
\end{itemize}

\vspace{-9pt}
\subsection*{Корпускулярно-волновой дуализм:}
\vspace{-3pt}
Явление, при котором материальные объекты проявляют свойства и частиц, и волн.


\section*{1.9 Дифракция электронов}
\vspace{-9pt}
\subsection*{Эксперименты:}
\vspace{-3pt}
Наблюдение дифракции электронов при прохождении через кристаллическую решетку, подтверждают волновые свойства электронов.

\end{document}